%\input{../header}

\section{Draftparts}

\begin{definition}
	A \emph{graph} $G$ is a finite $1$-dimensional CW complex. The set of edges is denoted by $E(G)$, the set of vertices by  $V(G)$ and the set of half edges by  $H(G)$.
	We call a graph \emph{connected} if the CW complex is connected in the topological sense.
	A graph is said to be \emph{$n$-valent} if every vertex has valency $n$ i.e. for every vertex the number of edges incident is  $n$.

	Lastly a \emph{tree} is a graph which contains no loops and a \emph{forest} is a collection of disjoint trees.
\end{definition}

On trivalent connected graphs we call an \emph{orientation} a choice of cyclic orders of all vertices up to an even number of changes.

\begin{definition}
	A \emph{forested graph} is a pair $(G, \Phi)$, where $G$ is a finite connected trivalent graph and $\Phi$ is an oriented forest which contains all vertices of $G$. 
\end{definition}

\begin{definition}
Let $(G,\Phi)$ be a forested graph and let  $e \in \Phi$. Moreover let $(G_{e},\Phi_{e})$ be the graph where $e$ has been collapsed.
Then there exist exactly two other graphs whose edge collapse results in $(G_{e},\Phi_{e})$. This is visualised in the figure below.
Where  $1,2,3,4$ represent the rest of the graph.

\ctikzfig{./tikzit/IHXRelator}
Now the vector 
\[
	(G,\Phi) + (G',\Phi') + (G'',\Phi'')
\]
is called the \emph{basic IHX relator} associated to $(G,\Phi,e)$.
\end{definition}

Denote by $\widehat{\fg}_{k}$ the vector space spanned by all forested graphs containing $k$ trees modulo the relations $(G,\Phi) = -(G,-\Phi)$.
Moreover let $\fg_{k}$ be the quotient of $\widehat{\fg}_{k}$ modulo the subspace spanned by all basic IHX relators.


\begin{definition}
	Let $\widehat{\partial}_E(G,\Phi): \widehat{\fg}_{k} \to \widehat{\fg}_{k-1}$ be given by
\begin{align*}
	\widehat{\partial}_{E}(G,\Phi) = \sum (G, \Phi \cup e)
.\end{align*}
where the sum is over all edges $e$ of $G \setminus \Phi$ such that $\Phi \cup e$ is still a forest.
Notice that this only happens if the two vertices of  $e$ lie in different trees of $\Phi$. Thus $\Phi \cup e$ has $k-1$ components.
The orientation of $\Phi \cup e$ is determined by ordering the edges of $\Phi$ with labels  $1,\ldots,k$ consistent with its orientation
and then labeling the new edge $e$ with  $k+1$.

Now let the boundary map $\partial_{E}: \fg_{k} \to \fg_{k-1}$ be the map induced by $p \circ \widehat{\partial}_{E}$ where $p$ is the quotient map $\widehat{\fg_{k}} \to \fg_{k}$.
\end{definition}

\begin{proposition}
	$\partial_{E}$ is well-defined and $\partial_{E}^2 = 0$.
\end{proposition}

\begin{proof}
	
\end{proof}

The \emph{forested graph complex} is thus defined as the sequence $\fg_{k}$ with boundary map $\partial_{E}$ and is well-defined by the above proposition.

\end{document}
