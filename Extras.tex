%\input{../header}

\section{Draftparts}

\begin{definition}
	A \emph{graph} $G$ is a finite $1$-dimensional CW complex. The set of edges is denoted by $E(G)$, the set of vertices by  $V(G)$ and the set of half edges by  $H(G)$.
	We call a graph \emph{connected} if the CW complex is connected in the topological sense.
	A graph is said to be \emph{$n$-valent} if every vertex has valency $n$ i.e. for every vertex the number of edges incident is  $n$.

	Lastly a \emph{tree} is a graph which contains no loops and a \emph{forest} is a collection of disjoint trees.
\end{definition}

On trivalent connected graphs we call an \emph{orientation} a choice of cyclic orders of all vertices up to an even number of changes.

\begin{definition}
	A \emph{forested graph} is a pair $(G, \Phi)$, where $G$ is a finite connected trivalent graph and $\Phi$ is an oriented forest which contains all vertices of $G$. 
\end{definition}

\begin{definition}
Let $(G,\Phi)$ be a forested graph and let  $e \in \Phi$. Moreover let $(G_{e},\Phi_{e})$ be the graph where $e$ has been collapsed.
Then there exist exactly two other graphs whose edge collapse results in $(G_{e},\Phi_{e})$. This is visualised in the figure below.
Where  $1,2,3,4$ represent the rest of the graph.

\ctikzfig{./tikzit/IHXRelator}
Now the vector 
\[
	(G,\Phi) + (G',\Phi') + (G'',\Phi'')
\]
is called the \emph{basic IHX relator} associated to $(G,\Phi,e)$.
\end{definition}

Denote by $\widehat{\fg}_{k}$ the vector space spanned by all forested graphs containing $k$ trees modulo the relations $(G,\Phi) = -(G,-\Phi)$.
Moreover let $\fg_{k}$ be the quotient of $\widehat{\fg}_{k}$ modulo the subspace spanned by all basic IHX relators.


\begin{definition}
	Let $\widehat{\partial}_E(G,\Phi): \widehat{\fg}_{k} \to \widehat{\fg}_{k-1}$ be given by
\begin{align*}
	\widehat{\partial}_{E}(G,\Phi) = \sum (G, \Phi \cup e)
.\end{align*}
where the sum is over all edges $e$ of $G \setminus \Phi$ such that $\Phi \cup e$ is still a forest.
Notice that this only happens if the two vertices of  $e$ lie in different trees of $\Phi$. Thus $\Phi \cup e$ has $k-1$ components.
The orientation of $\Phi \cup e$ is determined by ordering the edges of $\Phi$ with labels  $1,\ldots,k$ consistent with its orientation
and then labeling the new edge $e$ with  $k+1$.

Now let the boundary map $\partial_{E}: \fg_{k} \to \fg_{k-1}$ be the map induced by $p \circ \widehat{\partial}_{E}$ where $p$ is the quotient map $\widehat{\fg_{k}} \to \fg_{k}$.
\end{definition}

\begin{proposition}
	$\partial_{E}$ is well-defined and $\partial_{E}^2 = 0$.
\end{proposition}

\begin{proof}
	
\end{proof}

The \emph{forested graph complex} is thus defined as the sequence $\fg_{k}$ with boundary map $\partial_{E}$ and is well-defined by the above proposition.

\subsection{The Outer space}
We closely follow Vogtmann's definition from \cite[p. 2 ff.]{vogtmann16}
\begin{definition}
	By a \emph{metric graph} we mean a finite connected graph with positive real edge lengths, equipped with the path metric.
	We fix a model rose $R_{n}$ (a graph with one vertex and $n$ petals), and identify the petals of $R_{n}$ with the generators
	of the free group $F_{n}$. A point in $\mathcal{X}_{n}$ is then a metric graph $G$ together with a homotopy
	equivalence $g: R_{n} \to  G$ called a \emph{marking}; the marking serves to identify the fundamental group of $G$ with $F_{n}$.
	Marked graphs $(g,G)$ and $(g',G')$ are considered the same if there is an isometry  $f: G \to G'$ with $f \circ g$ homotopic to $g'$.

	To get a finite dimensional space we assume $G$ has no uni- and bivalent vertices (see Theorem \ref{thm:finGenCn}).
	Moreover we normalize our objects i.e. we assume the sum of edge lengths to be $1$ and assume that $G$ is $2$-connected.
\end{definition}
To make  $\mathcal{X}_{n}$ a space we need to define a topology. We proceed as follows: 
For every marked graph $(g,G)$ we define the open simplex $\sigma(g,G)$ as the set obtained by varying the edge lengths of $G$,
keeping their sum equal to $1$.
The simplex  $\sigma(g',G')$ is then a \emph{face} of $\sigma(g,G)$ if $(g',G')$ can be obtained from $(g,G)$ by collapsing some edges to points.

Finally $\mathcal{X}_{n}$ is the quotient space obtained from the disjoint union of the open simplices $\sigma(g,G)$ by face identification.

However not all faces of these simplices are in $\mathcal{X}_{n}$.
To rectify this we replace each open simplex $\sigma(g,G)$ by a closed simplex $\overline{\sigma}(g,G)$ and take
the quotient as before. This new space, denoted by $\mathcal{X}^{*}_{n}$, is a simplical complex and called the \emph{simplical closure} of Outer space.
The points in $\mathcal{X}^{*}_{n}$ which are not in $\mathcal{X}_{n}$ are said to be at infinity.

 Now the group $\out(F_{n})$ acts on $\mathcal{X}_{n}$ by changing the marking in particular
 any $\varphi \in \out(F_{n})$ can be realized by a homotopy equivalence $f: R_{n} \to R_{n}$
 by mapping petals to each other according to their identification with generators of $F_{n}$.
 The group action by $\varphi$ on $(g,G)$ is then defined by
 \[
	 (g,G) \varphi = (g \circ f, G)
 .\] 

 Finally $\mathcal{X}_{n}$ contains an equivariant deformation retract $K_{n}$, the spine of Outer space.
 It is a subcomplex of the barycentric subdivision of the simplical closure $\mathcal{X}_{n}^{*}$, consisting of simplices spanned by vertices which are not at infinity.

 In other language, $K_{n}$ is the geometric realisation of the partially ordered set of open simplices $\sigma(g,G)$ in $\mathcal{X}_{n}$, where 
 the partial order is given by the face relation.

 We have the following vital result providing the relationship between $\mathcal{X}_{n}$ and $\out(F_{n})$. This was proved by Culler and Vogtmann in \cite{vogtmann86}.
 \begin{theorem}
	 $\mathcal{X}_{n}$ is contractible and the action of $\out(F_{n})$ is proper.
	 The spine $K_{n}$ is an equivariant deformation retract of dimension $2 n - 3$ with compact quotient
 \end{theorem}
 
 \section{Introduction\footnotemark}
\footnotetext{This section follows closely Vogtmann's survey article \cite{vogtmann16}}
Free groups are one of the most basic examples of infinite finitely-generated groups.
The automorphism groups of free groups are of particular interest since they are tied to many other areas of mathematics.
The automorphism group $\aut(G)$ of a group $G$ can be separated into the inner automorphism group $\on{Inn}(G)$,
the group of automorphisms that arise from conjugation, and the outer automorphism group $\out(G)$, the quotient of  $\aut(G)$ by $\on{Inn}(G)$.

Most often the inner automorphism groups are well understood.  To understand the outer automorphism groups
the standard approach has been to study it via its action on some topological space.
For the outer automorphism group of the free group on $n$ generators $\out(F_{n})$ the space $\mathcal{X}_{n}$ known as "Outer space"
has been introduced by Culler and Vogtmann in \cite{vogtmann86}. As the points of the Outer space correspond to finite graphs with fundamental group $F_{n}$,
the structure of Outer space is connected to such diverse areas as the study of Lie algebras of derivations, degenerations of algebraic varieties,
the computation of Feynman integrals, and the statistics of phylogenetic trees.

A central concept of the Outer space is its spine $K_{n}$ which is an equivariant deformation retract of $\mathcal{X}_{n}$.
By being able to compute the rational homology of $K_{n} / \out(F_{n})$ one can also compute $\out(F_{n})$.
By identifying the spine $K_{n}$ with a cube complex one arrives at the forested graph complex
which has been introduced by Conant and Vogtmann in \cite{conant03}.

The forested graph complex is also the main focus of this paper.
We will begin by defining basic concepts of graphs, the Outer space and the forested graph complex as well as show the connection via the cube complex.

Afterwards we will focus on Morita's classes which are an infinite sequence of cocycles representing potentially nontrivial cohomology classes $\mu_{k} \in H^{4k}(\out(F_{2k+2})$.




