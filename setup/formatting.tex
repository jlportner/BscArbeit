%%% HEADER, FOOTER %%%%%%%%%%%%%%%%%%%%%%%%%%%%%%%%%%%%%%%%%%%%%%%%%%%%%%%%%%%%%

\pagestyle{fancy}
\renewcommand{\headrule}{}
%\renewcommand{\chaptermark}[1]{\markboth{#1}{}}
\fancyhf{} % clear everything
%\fancyhead[L]{\rightmark}
%\chead{\bfseries Zusammenfassung Algebra I/II}
%\rhead{Seite \thepage /\pageref*{LastPage}}
%\lfoot{}
\fancyfoot[C]{\thepage}
%\fancyfoot[R]{\thepage}

%%% TITLE FORMAT %%%%%%%%%%%%%%%%%%%%%%%%%%%%%%%%%%%%%%%%%%%%%%%%%%%%%%%%%%%%%%%

\setcounter{secnumdepth}{2}

%\titleformat{\chapter}[hang]
%{\normalfont\huge\bfseries}{\chaptertitlename\ \thechapter:}{20pt}{\Huge}
\titleformat{\section}
{\normalfont\LARGE\bfseries}{\thesection}{1em}{}
\titleformat{\subsection}
{\normalfont\large\bfseries}{\thesubsection}{1em}{}
\titleformat{\subsubsection}
{\normalfont\normalsize\bfseries}{\thesubsubsection}{1em}{}
\titleformat{\paragraph}[runin]
{\normalfont\normalsize\bfseries}{\theparagraph}{1em}{}
\titleformat{\subparagraph}[runin]
{\normalfont\normalsize\bfseries}{\thesubparagraph}{1em}{}

%%% SPACING %%%%%%%%%%%%%%%%%%%%%%%%%%%%%%%%%%%%%%%%%%%%%%%%%%%%%%%%%%%%%%

% Titles

%\titlespacing*{\chapter}{0pt}{0pt}{15pt}
\titlespacing*{\section}{0pt}{3.5ex plus 1ex minus .2ex}{2.3ex plus .2ex}
\titlespacing*{\subsection}{0pt}{3.25ex plus 1ex minus .2ex}{1.5ex plus .2ex}
\titlespacing*{\subsubsection}{0pt}{3.25ex plus 1ex minus .2ex}{1.5ex plus .2ex}
\titlespacing*{\paragraph}{0pt}{1.25ex plus 1ex minus .2ex}{1em}
\titlespacing*{\subparagraph}{\parindent}{3.25ex plus 1ex minus .2ex}{1em}

% Text, Paragraphs

\setstretch{1.05} % scaling of space between lines
\setlength{\parindent}{0pt} % indentation of paragraphs
\setlength{\parskip}{4.0pt plus 1.0pt minus 1.0pt} % space between paragraphs
%\setlength{\parskip}{0pt}

%%% SYMBOLS USED BY NUMBERINGS, ENVIRONMENTS, ... %%%%%%%%%%%%%%%%%%%%%%%%%%%%%%

% \renewcommand*\qedsymbol{$\blacksquare$} % alternative QED symbol
%\renewcommand{\thefootnote}{\arabic{footnote}} % normal footnotes on page
%\renewcommand{\thempfootnote}{\fnsymbol{mpfootnote}} % footnotes on minipages, e.g. in mdframed environments

%%% LISTS, ENUMERATIONS %%%%%%%%%%%%%%%%%%%%%%%%%%%%%%%%%%%%%%%%%%%%%%%%%%%%%%%%

% 'itemize'

\setlist[itemize]{noitemsep, topsep=0pt}

% 'enumerate'

\setlist[enumerate]{noitemsep, topsep=0pt}
% no special settings at the moment

% 'description'

% no special settings at the moment

% 'axioms'

%\newlist{axioms}{enumerate}{2}
%\setlist[axioms]{itemsep=0pt,label*=\arabic*.}

%%% GENERAL SYMBOLS %%%%%%%%%%%%%%%%%%%%%%%%%%%%%%%%%%%%%%%%%%%%%%%%%%%%%%%%%%%%
\newcommand\danger{\raisebox{\depth}{{\fontencoding{U}\fontfamily{futs}\selectfont\char 66\relax}}}
\newcommand\contra{\scalebox{1.5}{$\lightning$}}

%%% MDFRAMED PATCH %%%%%%%%%%%%%%%%%%%%%%%%%%%%%%%%%%%%%%%%%%%%%%%%%%%%%%%%%%%%%

\usepackage{xpatch}

\makeatletter
\xpatchcmd{\endmdframed}
  {\aftergroup\endmdf@trivlist\color@endgroup}
  {\endmdf@trivlist\color@endgroup\@doendpe}
  {}{}
\makeatother

%%% MDFRAMED STYLES %%%%%%%%%%%%%%%%%%%%%%%%%%%%%%%%%%%%%%%%%%%%%%%%%%%%%%%%%%%%

% thick frame and bar for title

%\mdfdefinestyle{style-box}{
%  skipabove=1.5ex plus .5ex minus .2ex,
%  skipbelow=1ex plus .2ex minus .2ex,
%  linewidth=2pt,
%  linecolor=Gray!20,
%   roundcorner=3pt,
%  innerleftmargin=0.5\baselineskip,
%  innerrightmargin=0.5\baselineskip,
%  innertopmargin=0.4\baselineskip,
%  innerbottommargin=0.4\baselineskip,
%  frametitlebackgroundcolor=Gray!20,
%  frametitleaboveskip=0.3pt,
%  frametitlebelowskip=0.3pt,
%  theoremseparator=,
%  theoremspace=\hfill,
%  theoremtitlefont=\mdseries\scshape,
%  nobreak=true
%}

% highlighted background

%\mdfdefinestyle{style-background}{
%  skipabove=1.5ex plus .5ex minus .2ex,
%  skipbelow=1ex plus .2ex minus .2ex,
%  hidealllines=true,
%  backgroundcolor=Gray!5,
%  innerleftmargin=0.5\baselineskip,
%  innerrightmargin=0.5\baselineskip,
%  innertopmargin=0.4\baselineskip,
%  innerbottommargin=0.4\baselineskip,
%}

% thin frame

%\mdfdefinestyle{style-leftline}{
%  skipabove=1.5ex plus .5ex minus .2ex,
%  skipbelow=1ex plus .2ex minus .2ex,
%  linewidth=1pt,
%  linecolor=Gray!50,
%  topline=false,
%  bottomline=false,
%  rightline=false,
%  innerleftmargin=0.5\baselineskip,
%  innerrightmargin=0,
%  innertopmargin=0.2\baselineskip,
%  innerbottommargin=0.0\baselineskip,
%}

%%% ENVIRONMENTS %%%%%%%%%%%%%%%%%%%%%%%%%%%%%%%%%%%%%%%%%%%%%%%%%%%%%%%%%%%%%%%



% Definition

\theoremstyle{definition}
\newtheorem{definition}{\definitionname}[section]
\newtheorem{attention}[definition]{\danger\ \attentionname}
\newtheorem{eg}[definition]{\examplename}

\theoremstyle{plain}
\newtheorem{theorem}[definition]{\theoremname}
\newtheorem{proposition}[definition]{\propositionname}
\newtheorem{corollary}[definition]{\corollaryname}
\newtheorem{lemma}[definition]{\lemmaname}

\theoremstyle{remark}
\newtheorem{remark}[definition]{\remarkname}
\newtheorem{claim}[definition]{\claimname}
\newtheorem{notation}[definition]{\notationname}
\newtheorem{convention}[definition]{\conventionname}



%\mdtheorem[
%  style=style-box,
%  linecolor=color-definition,
%  frametitlebackgroundcolor=color-definition
%]{definition}{\definitionname}[section]

% Theorem

%\mdtheorem[
%  style=style-box,
%  linecolor=color-theorem,
%  frametitlebackgroundcolor=color-theorem,
%  font=\itshape
%]{theorem}{\theoremname}[section]

% Proposition

%\mdtheorem[
%  style=style-box,
%  linecolor=color-proposition,
%  frametitlebackgroundcolor=color-proposition,
%  font=\itshape
%]{proposition}[theorem]{\propositionname}

% Corollary

%\mdtheorem[
%  style=style-box,
%  linecolor=color-corollary,
%  frametitlebackgroundcolor=color-corollary,
%  font=\itshape
%]{corollary}[theorem]{\corollaryname}

% Lemma

%\mdtheorem[
%  style=style-box,
%  linecolor=color-lemma,
%  frametitlebackgroundcolor=color-lemma,
%  font=\itshape
%]{lemma}[theorem]{\lemmaname}

%\mdtheorem[
%  style=style-box,
%  linecolor=color-attention,
%  frametitlebackgroundcolor=color-attention,
%  font=\itshape
%]{attention}[theorem]{\danger \attentionname}

%\theoremstyle{remark}

% Remark

%\newtheorem*{remark}{\remarkname}
%\surroundwithmdframed[
%  style=style-background,
%  backgroundcolor=color-remark
%]{remark}

% Example

%\newtheorem*{eg}{\examplename}
%\surroundwithmdframed[
%  style=style-background,
%  backgroundcolor=color-example
%]{eg}

%Notation

%\newtheorem*{notation}{\notationname}
%\surroundwithmdframed[
%  style=style-background,
%  backgroundcolor=color-notation
%]{notation}

%Convention

%\newtheorem*{convention}{\conventionname}
%\surroundwithmdframed[
%  style=style-background,
%  backgroundcolor=color-convention
%]{convention}

%Claim

%\newtheorem*{claim}{\claimname}
%\surroundwithmdframed[
%  style=style-background,
%  backgroundcolor=color-claim
%]{claim}

% Proof

%\surroundwithmdframed[
%  style=style-leftline
%]{proof}

%%% TEXT FORMATTING %%%%%%%%%%%%%%%%%%%%%%%%%%%%%%%%%%%%%%%%%%%%%%%%%%%%%%%%%%%%

% definitions
\let\epsilon\varepsilon
\renewcommand\emptyset{\varnothing}
\let\implies\Rightarrow
\let\impliedby\Leftarrow
\let\ForAll\forall
\renewcommand\forall{\;\ForAll}
\let\Exists\exists
\renewcommand\exists{\;\Exists}

\newcommand*{\df}[1]{\colorbox{color-definition}{\emph{#1}}}
