%\input{../header}
\input{./setup/MAIN.tex}
\usepackage{todonotes}
\usepackage{./tikzit/tikzit}
\usepackage{biblatex}
\input{./tikzit/test.tikzstyles}
\addbibresource{./references.bib}
\usepackage[stable]{footmisc}
\usepackage{caption}
\usepackage{subcaption}

\begin{document}	
\begin{titlepage}
   \begin{center}
       \vspace*{4cm}

       \textbf{The forested graph complex}

       \vspace{0.5cm}
            
       \vspace{1.5cm}

       \textbf{Jean-Luc Portner}

	   \vspace{0.8cm}
            
       Bachelor Thesis
       
       \vspace{0.5cm}
       
       Supervised by:\\
	   Thomas Hans Willwacher, Advisor\\
	   Michael Borinsky, Co-advisor
            
       \vfill
     
       Department of Mathematics\\
       ETH Zürich\\
       June 2022
            
   \end{center}
\end{titlepage}

\tableofcontents
\newpage
\section{Introduction}
As groups are a fundamental concept of mathematics understanding them has been a central aspect ever since their discovery.
Free groups are then the class of groups where no relation exists between their generators: 
The elements are words of generators and their inverses. The group relation is given by concatenating words
and reducing them i.e. if two adjacent elements are inverse to each other they will be omitted.
The free group on $n$ generators will be denoted by $F_{n}$.
The importance of free groups now arises from the fact that every group is isomorphic to a quotient group of a free group
an in particular any finitely generated group is isomorphic to a quotient group of $F_{n}$ for some $n$.

Another way to approach a group $G$ is to understand its automorphism group $\aut(G)$ as it describes $G$'s symmetries.
Now $\aut(G)$ can be separated into the inner automorphism group $\on{Inn}(G)$,
the group of automorphisms that arise from conjugation, and the outer automorphism group $\out(G)$, the quotient of  $\aut(G)$ by $\on{Inn}(G)$.

Combining those two aspects it is only natural that mathematicians study the automorphism group of $F_{n}$ since the beginning of the subject.
The inner automorphism group is well understood and isomorphic to $F_{n}$ itself.
The outer automorphism group $\out(F_{n})$ however still leaves many open questions.
As $F_{1} \cong \Z$ ,
\[
	\out(F_1) = \out(\Z) = \on{GL}_{1}(\Z)
.\] 
For $F_{2}$ Nielsen proved in \cite{nielsen17} that $\out(F_2) \cong \on{GL}_{2}(\Z)$.
For higher groups only the existence of a surjection $\out(F_{n}) \to \on{GL}_{n}(\Z)$ can be guaranteed.

Historically $\on{GL}_{n}(\Z)$ has been studied by by its action on the symmetric space $\on{SL}_{n}(\R) / \on{SO}_{n}(\R)$.
Due to the relation between $\out(F_{n})$ and $\on{GL}_{n}(\Z)$ early attempts of the study of the outer automorphism group
examined its action on $\on{SL}_{n}(\R) / \on{SO}_{n}(\R)$ induced by the above surjection.
However this action is not proper meaning that inverse of compact sets under the action are not necessarily compact.
Hence it behaves quite badly and a different approach was needed.

Therefore Culler and Vogtmann introduced a new space $\mathcal{X}_{n}$ known as "Outer space" in \cite{vogtmann86}.
To understand this space however we first have to introduce basic concepts of graph theory.

\subsection{Graphs}
\begin{definition}
	A \emph{graph} $G$ is a finite $1$-dimensional CW complex. The set of edges is denoted by $E(G)$, the set of vertices by  $V(G)$.
	We call an edge having the same start and end vertex a loop.

	We call a graph \emph{connected} if the CW complex is connected in the topological sense.
	A graph is $n$-edge-connected if it remains connected after removing  $n-1$ arbitrary edges.

	A graph is said to be \emph{$n$-regular} if every vertex has valency $n$ i.e. for every vertex the number of incident edges is $n$.
	The valency of a vertex $v \in G$ is often also called degree of $v$ and denoted by $\deg(v)$.

	For a subset of edges $\Phi$ of $G$ we denote by $G / \Phi$ the graph quotient, which is the quotient space of the CW complex $G$ over its topological subspace $\Phi$.
\end{definition}

\begin{remark}
	Note that sometimes these types of graphs are called multigraphs, as they are allowed to have multiple edges between vertices as well as loops.
	The word graph there normally refers to simple graphs which do not allow multi-edges and loops.

	In the context of algebraic topology however multigraphs are needed and thus the word graph here denotes multigraphs.
\end{remark}

\begin{definition}
	A subgraph $G'$ of a graph $G$ is a subcomplex of the CW-complex $G$. As a subcomplex is itself a CW-complex of dimension smaller or equal to the original complex,
	 $G'$ is itself a graph.

	A cycle in a graph $G$ is a subgraph that is homeomorphic to $S^1$. A tree is a
	connected graph containing no cycles. A forest is a collection of disjoint trees.
\end{definition}

\begin{theorem}\label{thm:fg_graph}
	Let $G$ be a graph. Then its fundamental group $\pi_{1}(G)$ is isomorphic to a free group.
\end{theorem}
By the theorem it makes sense to define the \emph{rank} of a graph $\rank(G)$ as the rank of its fundamental group.

A proof of this theorem will be given later at the beginning of section \ref{sec:RankGraph}.

Finally a \emph{metric graph} is a finite connected graph where each edge is assigned a positive real value 
value, its length, and on these lengths we define the path metric i.e.
the distance between two points is the length of the shortest path between them.
Here the length of the path is the sum of the lengths of the edges that are (partially) traversed.

\subsection{Outer space\footnote{The figures in this section are taken from \cite{vogtmann02}}}
We can now come back to the study of $\out(F_{n})$ and the aforementioned Outer space. 
For this we follow Vogtmann's survey in \cite{vogtmann02}.

To define Outer space we first consider the graph $R_{n}$ given by one vertex and $n$ edges each forming a loop.
By Theorem $\ref{thm:fg_graph}$ we can identify $\pi_1(R_{n})$ with the free group $F_{n}$ by
identifying the generators of $F_{n}$ $x_1,\ldots,x_{n}$ with oriented edges of $R_{n}$.
Then every reduced word in $F_{n}$ corresponds to a reduced edge-path loop starting and ending at the base-point of $R_{n}$.
An automorphism $\phi$ on $F_{n}$ is now a homotopy equivalence sending the loop corresponding to $x_{i}$
to the one identified with $\phi(x_{i})$.

Outer space can now be defined as the set of marked metric graphs $(g,G)$ under an equivalence relation.
The graphs $(g,G)$ have to satisfy
\begin{itemize}
	\item $G$ is a finite graph with vertex valency at least $3$,
	\item $g: R_{n} \to G$ is a homotopy equivalence called the marking of $G$, and
	\item each edge has a positive length such that the sum of all
		edge lengths is one. This turns $G$ into a metric space via the path metric.
\end{itemize}
The equivalence relation is given by $(g,G) \equiv (g',G')$ if and only if 
there exists an isometry $h: G \to G'$ such that $g \circ h$ is homotopic to $g'$.

Intuitively each point $(g,G)$ can be represented by drawing the graph $G$,
finding a maximal forest $T$ on $G$ and labelling all edges not in $T$ with an orientation
and an element of $F_{n}$.
Now the labels determine a map $f: G \to R_{n}$ given by sending $T$ to the vertex of $R_{n}$ and sending
each edge in $G \setminus T$ to the corresponding loop in $R_{n}$ given by the labelling.
The labelling is chose in such a way that $f$ is a homotopy inverse to $g$.
An example is given in the figure below.
\begin{figure}[h]
	\centering
	\includegraphics[width=0.2\textwidth]{./Images/pointOfOuterSpace.pdf}
	\caption{A point of Outer Space}
\end{figure}

To define a topology on $\mathcal{X}_{n}$ we look at the set $\mathcal{C}$ of conjugacy classes of $F_{n}$, the cyclic reduced words.
Now we can define a map from Outer space $\mathcal{X}_{n}$ to the infinite projective space $\mathbb{RP}^{\mathcal{C}}$.
This map assigns to every metric marked graph $(g,G)$
For every metric marked graph $(g,G) \in \mathcal{X}_{n}$ we assign the cyclically reduced word $w$
the length of the unique cyclically reduced path loop in $G$ homotopic to $g(w)$.
The topology is then the subspace topology obtained from RP.

This definition might seem quite ad hoc, however under it $X_{n}$ decomposes nicely into
a disjoint union of open simplices: Every marked graph $(g,G)$ belongs to the open simplex
containing all graphs that can be reached by varying the non-zero lengths of edges such that
the total edge length remains one. The faces of the simplex are then the marked graphs where one edge of $(g,G)$
has been fully contracted. Of course a contracted edge can be extended again in multiple ways thus connecting
the different simplices. An example is given in the figure below.

\begin{figure}[h]
	\centering
	\begin{subfigure}{0.3\textwidth}
		\includegraphics[width=\textwidth]{./Images/outerSpaceD2.pdf}
		\caption{Outer space $\mathcal{X}_{2}$}
	\end{subfigure}
	\hspace{0.1\textwidth}
	\begin{subfigure}{0.3\textwidth}
		\includegraphics[width=\textwidth]{./Images/outerSpaceFaces.pdf}
		\caption{Simplices in $\mathcal{X}_{2}$}
	\end{subfigure}
\end{figure}

For a marked graph with $k+1$ edges the corresponding simplex is $k$ dimensional.
Moreover the identification works the other way round to i.e. every open simplex in $X_{n}$ 
is a face of a maximal simplex which corresponds to a trivalent marked graph.
Taking the argument from the proof of Theorem \label{thm:finGenCn} we see that the dimension of $\mathcal{X}_{n}$ is equal to $3n -4$.

Now with the Outer space defined we can look at the group action of $\out(F_{n})$ on $\mathcal{X}_{n}$.
Here $\out(F)_{n}$ acts on $X_{n}$ on the right as follows:
Every $\phi \in \out(F)_{n}$ induces a map $f: R_{n} \to R_{n}$ by mapping the edge labelled
by $x$ to the edge labelled by $\phi(x)$.
Then the right group action is defined by $(g,G) \phi = (g \circ f, G)$.

An inconvenience that arises is that the quotient of $\mathcal{X}_{n}$ by $\out(F)_{n}$ is not compact.
Resolving this leads us to the next construction.

\subsection{Spine of Outer Space}
The beginning of this section follows Culler and Vogtmanns original definition from \cite{vogtmann86}.

In a first step we define a more convenient somewhat simpler version of $\mathcal{X}_{n}$ known as reduced Outer space and denote it by $Y_{n}$.
The points in the subspace $Y_{n}$ are the marked graphs $(g,G)$ not containing any separating edges, i.e.
no edges $e$ such tat $G \setminus e$ is disconnected.
An equivariant deformation retraction from $\mathcal{X}_{n}$ to $Y_{n}$ is given by shrinking the lengths of
the separating edges to zero while uniformly extending the lengths of the other edges in order to preserve the total edge length of one.

We can now define the spine of Outer space $K_{n}$ as follows:
$K_{n}$ is the maximal full subcomplex of the barycentric subdivision of $Y$ which is disjoint from the boundaries of the open simplices in $Y$.
The vertices of $K_{n}$ are the barycenters of simplices in $Y_{n}$ i.e. the subspace of $Y_{n}$ containing marked graphs 
with equal edges length. The deformation retraction from $Y_{n}$ to $K_{n}$ is given by
collapsing every simplex $\tau$ of the barycentric subdivision of $Y_{n}$ to the face of $\tau$ contained in $K_{n}$.
This can be done equivariantly.
Thus $K_{n}$ can be thought of as ignoring the metric structure of $Y_{n}$ and only focusing on its combinatorial structure.

Going the other way if we have to vertices $(g,G)$ and $(g',G')$ in $K_{n}$. Then the open simplex in $X_{n}$ 
determined by $(g,G)$ is a face of  the one determined by $(g',G')$ exactly when $G$ is obtained from $G'$ 
by collapsing a forest of edges in $G'$ and  $g$ is homotopic to the composition of $g'$ with the collapsing map.
This collapsing is also called a forest collapse.

Through this we can determine the dimension of $K_{n}$ as collapsing an edge decreases the number of vertices by $1$.
Thus with a similar argument as in the proof of Theorem \label{thm:finGenCn} we see that $\dim(K) = 2n -3$.
An example of a part of the spine of $\mathcal{X}_{2}$ is given below\footnote{This figure is taken from \cite{vogtmann02}}. 

\begin{figure}[h]
	\centering
	\includegraphics[width=0.3\textwidth]{./Images/spineOfOuterSpace.pdf}
	\caption{A section of the spine of Outer space $\mathcal{X}_{2}$}
\end{figure}

From our construction above $K_{n}$ has the structure of a simplical complex where a $k$-simplex
is a chain of $k$ forest collapses. Coming back to $\out(F_{n})$ acting on spaces; its action on $\mathcal{X}_{n}$ extends to a simplical action on $K_{n}$.
Culler and Vogtmann proved in \cite{vogtmann86} that this action has finite stabilizers.
Thus the rational homology of $\out(F_{n})$ can be computed as the quotient of $K_{n}$ by $\out(F_{n})$
\[
	H_{\bullet}(\out(F_{n},\Q) \cong H_{\bullet}(K_{n}/ \out(F_{n}),\Q)
.\] 

In order to calculate this homology we turn $K_{n}$ into a cube complex i.e. a CW-complex where the cells are homeomorphic
to Euclidean cubes and the attaching maps identify faces with lower dimensional cubes via homeomorphisms.
For a forest $\Phi$ in $G$ with $k$ edges we can now define the $k$-cube:
From $\Phi$ we get a chain of $k$-forest collapses by collapsing each edge in $\Phi$ at a time.
This yields a $k$-simplex. Collapsing the edges in another order yields another $k$-simplex.
All these different $k$-simplices can now be fit together to triangulate an $k$-dimensional cube.
Thus every $k$-cube is given by a graph $G$ and a forest $\Phi$ of size $k$.
The faces of dimension $k-1$ are the graphs obtained from $G$ where one edge in $\Phi$ has been collapsed.
An example is given in the figure below.

Viewing $K_{n}$ as a cube complex with one cube for every triple $(g,G,\Phi)$ we can define an orientation up to even permutation
via ordering the edges of $\Phi$. Then the rational homology of $K_{n} / \out(F_{n})$ can be
computed from a chain complex with one generator for each pair $(G,\Phi)$ that has no orientation
reversing-automorphism.

\subsection{The Forested graph complex}
This chain complex is generally known as the forested graph complex and has been
introduced by Conant and Vogtmann in \cite{conant03}. In this introductory section
we will describe the original construction. In the later chapter we are going to
introduce a more practical hands-on but equivalent definition from \cite{conant08}.

\begin{definition}
	A forested graph is a pair $(G,\Phi)$ of a finite conntected trivalent graph $G$ and an oriented forest $\Phi$ containing all vertices of $G$.
	The orientation on the forest is an ordering of its edges where interchanging any two edges reverses the orientation.
\end{definition}

We now denote by $\widehat{f\mathcal{G}}_{k}$ the vector space spanned by forested graphs with forest size $k$ modulo the relations
$(G,\Phi) = -(G,-\Phi)$.

Now if we consider a forested graph $(G,\Phi)$ and we collapse an edge $e$ in $\Phi$ then the graph obtained $(G_{e},\Phi_{e})$ 
has exactly one $4$ valent vertex. As the image below shows there are exactly two other graphs
whose edge collapse leads to the graph $(G_{e},\Phi_{e})$.
\ctikzfig{./tikzit/IHXRelator}
If we denote those two by $(G',\Phi')$ and  $(G'',\Phi'')$ then we call the vector
\[
	(G,\Phi) + (G',\Phi') + (G'',\Phi'')
.\] 
the basic IHX relator associated to $(G,\Phi,e)$. 
We denote be $IHX_{k}$ the subspace of $\widehat{f\mathcal{G}}_{k}$ spanned by all basic IHX relators
and define $f\mathcal{G}_{k}$ as the quotient space $\widehat{f\mathcal{G}}_{k} / IHX_{k}$.

Finally we can define a boundary map  $\partial: f\mathcal{G}_{k} \to f\mathcal{G}_{k-1}$ induced by the map on $\widehat{f \mathcal{G}}_{k}$ given by
\[
	\partial_{E}(G,\Phi) = \sum (G,\Phi \cup e)
\] 
where we sum over all edges in $G \setminus \Phi$ such that $\Phi \cup e$ is still a forest and 
$e$ gets the label $k+1$ in the orientation.

One can check that $\partial_{E}$ is really a boundary map i.e. that $\partial_{E}^2 = 0$ and thus
we get a chain complex $f\mathcal{G}_{\bullet}$ with boundary map $\partial_{E}$.
The rational homology of this complex now computes the rational homology of $\out(F_{n})$ as
explained at the end of the previous section.


\subsection{Known homology of \boldmath$\out(F_{n})$}
To conclude this introduction we try to summarize the known homology of $\out(F_{n})$.

\newpage


\section{The rank of a graph}\label{sec:RankGraph}
We start this section of by proving Theorem \ref{thm:fg_graph} which we state again below.
Afterwards we expand on the concept of rank and prove several identities and relations.

\begin{theorem}
	Let $G$ be a graph. Then its fundamental group $\pi_{1}(G)$ is isomorphic to a free group.
\end{theorem}

\begin{proof}
	The following proof is from \cite[p. 43f]{hatcher00}
	Let $G$ be a graph. W.l.o.g. $G$ is connected as else we consider each connected component separately. 
	Then let $T$ be a spanning tree on $G$ i.e. $T$ is a tree containing every vertex of $G$.
	Then $T$ is contractible.
	Now choose for every $e_{\alpha} \in E \setminus T$ an open neighborhood $A_{\alpha}$ of $T \cup e_{\alpha}$ that deformation retracts onto $T \cup e_{\alpha}$.
	The intersection of such $A_{\alpha}$ is $T$ and thus contractible. Moreover as $G$ is connected as a graph $A_{\alpha}$ and $T$ are path connected.
	Now the $A_{\alpha}$ form an open cover of $G$ and as $T$ is simply connected by Van Kampen's Theorem we get that $\pi_{1}(G) = *_{\alpha} \pi_{1}(A_{\alpha})$.
	Finally $A_{\alpha}$ deformation retracts onto $S^{1}$ and thus $\pi_{1}(A_{\alpha}) = \Z$. Now there are exactly $\abs{E} - \abs{T}$ many $A_{\alpha}$,
	which as $T$ is a spanning tree results in $\pi_1(G)$ being free on $\abs{E} - \abs{V} + 1$ generators.
\end{proof}

To understand the rank better we will need the following definitions
\begin{definition}
	For a finite CW-complex $X$ the \emph{Euler Characteristic} is defined as the alternating sum
	\[
		\chi(X) = k_0 - k_1 + k_2 - \ldots
	\] 
	where $k_{i}$ denotes the number of cells of dimension $i$ in the complex $X$.
\end{definition}
For graphs we thus get $\chi (G) = k_0 - k_1$, as they are $1$-dimensional which is equal to $\chi(G) = \abs{V} - \abs{E}$.

\begin{definition}
	Let $G$ be a graph. Then its cycle space is the set of even-degree subgraphs of $G$. 
	This space forms a vector space over $\F_2$ where the vector addition is given by the symmetric difference of two or more subgraphs.
	A basis of this space is called cycle basis and two cycles are independent if they are linearly independent in the vector space.
\end{definition}

\begin{remark}
	The cycle space is equal to the one homology group of $G$ with coefficients in $\F_2$ i.e. $H_1(G,\F_2)$.
\end{remark}

The following proposition 
\begin{proposition}\label{prop:rank}
	Let $G$ be a connected graph. Then the following are equal:
	\begin{enumerate}
		\item The rank of $G$.
		\item The number of independent cycles in $G$ i.e. the size of the cycle basis of $G$.
		\item The first Betti number i.e. the rank of $H_{1}(G)$.
		\item $1 - \chi(G) = \abs{E} - \abs{V} + 1$.
	\end{enumerate}
\end{proposition}

For the proof we will need the following Lemma:
\begin{lemma}
	Let $A$ be a set. Then the abelianization of the free group on $A$ is isomorphic the free abelian group of $A$.
\end{lemma}

\begin{proof}
	Consider the space $X = \bigvee_{a \in A} S^{1}$. Then by Van Kampen's Theorem $\pi_1(X) = *_{a \in A} \Z$.
	On the other hand we have $H_1(X) = \bigoplus_{a \in A} H_1(S_1) = \bigoplus_{a \in A} \Z$ which follows from the relative Homeomorphism Theorem.
	Now using Hurewicz Theorem we get that the abelianization of $\pi_1(X)$ is isomorphic to $H_1(X)$ and thus the desired statement
\end{proof}

\begin{proof}[Proof of the Theorem]
	(1) = (4): This was shown in the proof of Theorem \ref{thm:fg_graph}.\\
	(1) = (3): From Hurewicz Theorem we get that the abelianization of $\pi_{1}(G)$ is equal to $H_{1}(G)$
	and thus by the previous Lemma that the rank of $\pi_1(G)$ is equal to the rank of $H_1(G)$ which is the first Betti number.\\
	(4) = (2)\footnote{This proof is based on Harary's proof in \cite[p. 37-40]{harary69}.}:
		Consider again the sets $A_{\alpha}$ from the proof of Theorem \ref{thm:fg_graph}. Then each of them deformation retracts onto a cycle in $G$.
	Let $Z(T)$ be the set of cycles obtained in this way. Then  $Z(T)$ is independent as each cycle contains an edge not contained in any other cycle.
	Moreover every cycle $Z$ in $G$ can be written as the symmetric difference over the cycles corresponding to the edges in $(E \setminus T) \cap Z$.
	Thus $Z(T)$ spans the cycle space and consequently is a cycle basis. Now the size of $Z(T)$ is given by $\abs{E} - \abs{V} + 1$ and thus we conclude.
\end{proof}

\begin{definition}
	Let $G,H$ be two graphs. A map $f: V(G) \to V(H)$ is said to be a graph isomorphism if $f$ is a bijection such that
	\[
		(u,v) \in E(G) \Leftrightarrow (f(u),f(v)) \in E(H)
	.\] 
\end{definition}

\begin{eg}\label{ex:gAuto}
	Consider the following graphs.

	\ctikzfig{./tikzit/graph_automorphism}
	Then $G$ is $3$-connected and $3$-regular. Moreover $G$ is isomorphic to $H$.
	An isomorphism between them is given by mapping the same colored nodes to each other.
	The graph quotient $G / \{e,f,g\}$ is given by $J$.
\end{eg}

Finally we introduce the notion of degree of a graph also sometimes called excess.
\begin{definition}
	Let $G$ be a connected graph of rank $n$ with vertex-valency $\geq 3$. Its \emph{degree} is defined by
	\[
		\deg(G) := \sum_{v \in V(G)} (\deg(v) - 3)
	.\] 
\end{definition}

\begin{proposition}
	Let $G = (V,E)$ be a graph of rank $n$. Then we have the following identities:
	\begin{enumerate}
		\item $\deg(G) = 2 \abs{E} - 3 \abs{V}$
		\item $\abs{V} = 2n -2 - \deg(G)$
		\item $\abs{E} = 3n - 3 - \deg(G)$
		\item $G$ is $3$-regular $\Leftrightarrow \deg(G) = 0$.
	\end{enumerate}	
\end{proposition}

\begin{proof}
	By counting half edges we get $2 \abs{E} = \sum_{v \in V} \deg(V)$.
	Combining this with the definition of degree we directly get the first identity.
	Using Proposition \ref{prop:rank} and the first identity we have
	\begin{align*}
		2 n -2 - \deg(G) &= 2 \abs{E} - 2 \abs{V} + 2 - 2 - 2\abs{E} + 3 \abs{V} = \abs{V}\\
		3 n -3 - \deg(G) &= 3 \abs{E} - 3 \abs{V} + 3 - 3 - 2\abs{E} + 3 \abs{V} = \abs{E}
	\end{align*}
	which proves the second and third. The last statement follows as every element in the sum of the degree is non-negative.
	Thus $\deg(G) = 0$ if and only if every term is $0$ and thus if and only if every vertex has valency $3$.
\end{proof}

\section{The forested graph complex}
As mentioned before the forested graph complex has first been introduced by Conant and Vogtmann in \cite{conant03}.
Here however we will introduce the simplified construction and definition of the forested graph complex given by Conant and Vogtmann in \cite{conant08} 
which the author first found in \cite{bartholdi16}.
Very useful in the general understanding of  what a graph complex is and how the boundary map acts was Bar-Natan's and McKay's draft \cite{natan01}.
Inspired by this similar examples for the forested graph complex are presented.

Let $F_{n}$ denote the free group of rank $n$. We also denote by $\mathbb{S}_{n}$ the symmetric group of degree $n$.
\begin{definition}
	An \emph{admissible graph of rank $n$} is a $2$-edge-connected graph $G$ with fundamental group isomorphic to $F_{n}$ and with vertex-valency $\geq 3$.
\end{definition}

We often just write admissible graph for an admissible graph of rank $n$.

\begin{definition}
	Let $G = (V,E)$ be a graph. An ordering on its edges is a bijective function $\sigma$ from $E$ to $\{1,\ldots,\abs{E(G)}\}$.
	Notice that $\mathbb{S}_{\abs{E}}$ acts on $\sigma$ by $\pi \circ \sigma$ for $\pi \in \mathbb{S}_{n}$.
	We call the tuple $(G,\sigma)$ an ordered graph and note that $\mathbb{S}_{\abs{E}}$ acts on $(G,\sigma)$ by $\pi (G,\sigma) = (G,\pi \sigma)$ for $\pi \in \mathbb{S}_{\abs{E}}$.

	A \emph{forested graph} is a triple $(G,\Phi,\sigma)$ where $G$ is an admissible graph $\Phi$ is a subset of edges that spans a forest on $G$ and 
	$\sigma$ is an ordering on $\Phi$ i.e. $\sigma : \Phi \to \{1,\ldots,\abs{\Phi}\}$.

	A map $f$ between two forested graphs $(G,\Phi, \sigma) \to (H,\Psi, \tau)$ is said to be a forested graph isomorphism if 
	$f$ is a graph isomorphism on $G$,  $f(\Phi) = \Psi$ and $\sigma = \tau \circ f $
\end{definition}

We know want to construct our graph complex. For this we remember the notion of a graded vector space:
\begin{definition}
	A graded vector space, is a vector space $V$ and with a decomposition $\left(V_{n}\right)^{\infty}_{n=0} $ such that
	\[
		V = \bigoplus_{k=0}^{\infty} V_{k}
	.\] 
\end{definition}

We know consider the $\Q$-vector space $C$ spanned by isomorphism classes of forested graphs, subject to the relation
\[
	(G,\Phi, \pi \circ \sigma) = \sgn{\pi} \cdot (G,\Phi, \sigma) \qq{for all} \pi \in \mathbb{S}_{n}(\abs{\Phi})
.\]
Under this relation we call $\sigma$ an orientation.
Observe that if $(G,\Phi, \sigma) \simeq (G,\Phi, \pi \circ \sigma)$ for an odd permutation $\pi$ then 
$(G,\Phi, \sigma) \simeq (G,\Phi, \pi \circ \sigma) = - (G,\Phi, \sigma)$ and thus $(G,\Phi, \sigma) = 0$ in  $C_{k}$.

We can define the following three gradings on $C$:
 \begin{itemize}
	\item Let $C^{n} \subseteq C$ be the subspace spanned by forested graphs of rank $n$. Then clearly  $C^{n} \cap C^{m} = \emptyset$ for $n \neq m$ and
		as every graph has a rank, we get that the $C^{n}$ form a grading on $C$.
	\item Let $C_{k} \subseteq C$ be the subspace spanned by forested graphs $(G,\Phi,\sigma)$ with $k = \abs{\Phi}$. Clearly this also yields a decomposition of $C$ into a direct sum
		and thus yields another grading on $C$.
	\item Let  $C_{d} \subseteq C$ be the subspace spanned by forested graphs of degree $d$. Once again this yields a grading on $C$.
\end{itemize}
In the following we will mostly be concerned with the first two gradings. In particular we will consider the double-grading $C_{k}^{n}$,
where $k$ denotes the forest size and $n$ the rank. 

\begin{eg}
	Consider the graphs $G, J$ as in Example \ref{ex:gAuto}. Then $G$ and $J$ are admissible graphs of rank $4$ and $3$. Thus if we equip them with ordered forests 
	$(\Phi,\sigma), (\Psi,\tau)$ as below 
	(where the red edges represent the forest and the numbers the orientation) we get forested graphs in $C^{4}_{4}$ and $C^{3}_{2}$ respectively.

	\ctikzfig{./tikzit/forested_graph_automorphism}

	Observe, that $(J,\Psi) = 0$ in $C_{2}^{3}$, since $ (1 2)$ is an odd permutation and  $(1 2) (J,\Psi)$ is isomorphic to $(J,\Psi)$ 
	via the isomorphism mirroring along the vertical passing through the blue and purple vertex.

	$(G,\Phi)$ however is not trivial as the automorphism group is given by the identity,
	mirroring along the vertical, exchanging inner and outer vertices and their composition.
	None of these automorphisms induce an odd permutation and hence $(G,\Phi)$ does not vanish.
\end{eg}

Before we construct the chain complex we show that the $C^{n}$ are finitely generated and thus so are the $C_{k}^{n}$.

\begin{theorem}\label{thm:finGenCn}
	For all $n$, $C^{n}$ is finitely generated and $C_{k}^{n} = 0$ for all $k > 2n-3$. 
\end{theorem}

To prove this theorem we first need the following lemma:
\begin{lemma}
	For $n,m \in \N$ There are only finitely many admissible graphs $G = (V,E)$ with $\abs{V} \leq n$ vertices and $ \abs{E} \leq m$ edges.
\end{lemma}

\begin{proof}
	Every graph on $n$ vertices can be written as incidence matrix and each entry is $\leq m$ if the graph has maximally $m$ edges. 
	Thus there are maximally $m^{n^2}$ many different incidence matrices for graphs with  $n$ vertices and maximally $m$ edges.
	As every graph corresponds to an incidence matrix this also gives an upper bound on the number of different graphs with $n$ vertices.

	Thus the maximal possible number of admissible graphs with $\leq n$ vertices and $\leq m$ edges is bounded by
	\[
		\sum_{k=1}^{n} m^{k^2} 
	\]
	which is finite.
\end{proof}

\begin{proof}[Proof of Theorem \ref{thm:finGenCn}]
	By counting half edges we have
	\[
		2 \abs{E} = \sum_{v \in V} \deg{v}
	.\] 
	Using that admissible graphs have vertex-valency $\geq 3$ and rearranging yields $\abs{E} \geq \frac{3}{2} \abs{V}$.
	From Proposition \ref{prop:rank} we get that $\abs{E} = \abs{V} + n-1$.
	Combining yields
	\[
		\abs{V} + n -1 \geq \frac{3}{2} \abs{V} \Leftrightarrow 2 (n-1) \geq \abs{V}
	\] 
	and plugging in the above in the identity from the Proposition yields $\abs{E} \leq 3 (n-1)$.

	Thus by the above lemma we have that there are only finitely many graphs with $\abs{V} \leq 2 (n-1)$ and $\abs{E} \leq 3(n-1)$.
	As each graph also only has a finite number of forests and each of them has a finite number of orientations we get that
	$C^{n}$ is finitely generated.

	That $C_{k}^{n} = 0 \forall k > 2n -3$ follows from the bound on the number of vertices and the fact that a forest in a graph has maximally $\abs{V} - 1$ edges.
\end{proof}

\begin{remark}
	Notice that the constraint of vertex-valency $\geq 3$ in the definition of admissible graphs is necessary for $C^{n}$ being finitely generated.
	As else we can consider the following family of graphs:
	\ctikzfig{./tikzit/infinite_admissible_graphs}
	They all have rank $n$ (which can be checked via the Euler characteristic), are $2$-edge-connected and not isomorphic as they have different number of vertices/edges.
\end{remark}

\begin{remark}
	The bound on the $C_{k}^{n}$ can not be improved as the graph $J$ from the example above with the tree extended by the edge between purple and green has rank $3$ and
	tree size $3$ which equals $2 \cdot 3 -3$.
\end{remark}

To construct our chain complex we fix the rank $n$ and define a differential as follows:
\begin{definition}
	Let $(G,\Phi,\sigma) = (G, \{e_1,\ldots,e_{k}\},\sigma)$ be a forested graph. Then let $\partial_{C}, \partial_{R}: C_{k}^{n} \to C_{k-1}^{n}$ be given by
	\begin{align*}
		\partial_{C}(G,\Phi,\sigma) &= \sum_{i = 1}^{k} (-1)^{i} (G / e_{i}, \Phi \setminus \{e_{i}\}, \sigma_{e_{i}}),\\
		\partial _{R}(G,\Phi,\sigma) &= \sum_{i = 1}^{k} (-1)^{i} (G,\Phi \setminus \{e_{i}\}, \sigma_{e_{i}}) 
	\end{align*}
	where $\sigma_{e_{i}}: \Phi \setminus \{e_{i}\} \to \{1,\ldots,k-1\}$ is given by
	\[
		\sigma_{e_{i}}(e) = \begin{cases}
			\sigma(e) & \text{ if }\sigma(e) < i\\
			\sigma(e) - 1 & \text{ if } \sigma(e) > i
		\end{cases}
	.\]
	Notice that the case $\sigma(e) = i$ cannot happen as $e_{i}$ is not contained in $\Phi \setminus \{e_{i}\}$. 
	Finally define the boundary map $\partial = \partial_{C} - \partial_{R}$.
\end{definition}

\begin{proposition}
	$\partial$ is well-defined and $\partial^2 = 0$.
\end{proposition}

For better readability we will omit the orientation $\sigma$ in the proof.
\begin{proof}


	We proof the result in three steps:\\
	\textbf{Step 1:} Contracting an edge of a graph does not change the Euler characteristic as both the vertex number and the edge number decreases by one.
	Thus $\partial_{C}$ preserves the rank of the graph. Moreover the vertex-valency stays $\geq 3$ and the graph continues to be $2$-edge-connected.
	Hence it is admissible.
	Moreover $\partial_{C}$ as well as $\partial_{R}$ remove one edge from each forest. Thus decreasing
	$k$ by $1$. Hence both maps are well-defined from $C^{n}_{k}$ to $C^{n}_{k-1}$ and thus so is $\partial$.

	Let $(G,\Phi) = (G, \{e_1,\ldots,e_{p}\})$ be a forested graph and denote the edges in $\Phi \setminus \{e_{i}\}$ by $\{e_1',\ldots,e_{p-1}'\}$.
	For the consecutive steps we need the following observations:
	\[
		(G / e_{i}) /  e_{j}' = \begin{cases}
			(G / e_{j}) / e_{i-1}' & \text{ if } i > j\\
			(G / e_{j+1}) / e_{i}' & \text{ if } i \leq j
		\end{cases}
		\qq{ and }
		(\Phi \setminus \{e_{i}\}) \setminus \{e_{j}'\}  = \begin{cases}	
			(\Phi \setminus \{e_{j}\}) \setminus \{e_{i-1}'\} & \text{ if } i > j\\
			(\Phi \setminus \{e_{j+i}\}) \setminus \{e_{i}'\} & \text{ if } i \leq j
		\end{cases}
	\]

	\textbf{Step 2:}\enskip
	\emph{Claim:}
		$\partial_{C}^2 = 0$ and $\partial_{R}^2 = 0$

	We compute:
	\begin{align*}
		\partial_{C}^2 &= \partial_{C} \sum_{i=1}^{p} (-1)^{i}(G / e_{i}, \Phi \setminus \{e_{i}\})
		=  \sum_{i=1}^{p} \sum_{j=1}^{p-1} (-1)^{i+j}((G / e_{i}) / e_{j}', (\Phi \setminus \{e_{i}\} ) \setminus \{e_{j}'\})  \\
					   &= \sum_{j < i} (-1)^{i+j} ((G / e_{i}) / e_{j}', (\Phi \setminus \{e_{i}\} ) \setminus \{e_{j}'\}) + \sum_{i \leq j} (-1)^{i+j}
					   ((G / e_{i}) / e_{j}', (\Phi \setminus \{e_{i}\} ) \setminus \{e_{j}'\}) 
	\end{align*}
	We claim that the right and left sum cancel. For this first apply the observations above to the left sum and then change variables by setting $l = j$ and  $m = i-1$ to obtain:
	\begin{align*}
		\sum_{j < i} (-1)^{i+j} ((G / e_{i}) / e_{j}', (\Phi \setminus \{e_{i}\}) \setminus \{e_{j}'\} ) &= 
		\sum_{j < i} (-1)^{i+j}((G / e_{j}) / e_{i-1}', (\Phi \setminus \{e_{j}\}) \setminus \{e_{i-1}'\} ) \\ 
		&= \sum_{l \leq m} (-1)^{l+m+1} ((G / e_{l}) / e_{m}', (\Phi \setminus \{e_{l}\}) \setminus \{e_{m}'\} ) 
	\end{align*}
	This last expression is the same as the left sum above but with opposite sign. Thus they cancel and we have shown $\partial_{C}^2 = 0$.
	The same argument shows that $\partial_{R}^2 = 0$.

	\textbf{Step 3:} \emph{Claim:} $\partial_{C} \partial_{R} - \partial_{R} \partial_{C} = 0$

	For the mixed terms we compute as follows
	\begin{align*}
		\partial_{R} \partial_{C} &=  \sum_{i=1}^{p} \sum_{j=1}^{p-1} (-1)^{i+j}(G / e_{i}, (\Phi \setminus \{e_{i}\} ) \setminus \{e_{j}'\})  \\
					   &= \sum_{j < i} (-1)^{i+j} (G / e_{i}, (\Phi \setminus \{e_{i}\} ) \setminus \{e_{j}'\}) + \sum_{i \leq j} (-1)^{i+j}
					   (G / e_{i}, (\Phi \setminus \{e_{i}\} ) \setminus \{e_{j}'\}) \tag{$*$}
	\end{align*}
	and
	\begin{align*}
		\partial_{C} \partial_{R} &=  \sum_{i=1}^{p} \sum_{j=1}^{p-1} (-1)^{i+j}(G / e_{j}', (\Phi \setminus \{e_{i}\} ) \setminus \{e_{j}'\})  \\
					   &= \sum_{j < i} (-1)^{i+j} (G / e_{j}', (\Phi \setminus \{e_{i}\} ) \setminus \{e_{j}'\}) + \sum_{i \leq j} (-1)^{i+j}
					   (G / e_{j}', (\Phi \setminus \{e_{i}\} ) \setminus \{e_{j}'\}) \\
					   &\stackrel{(\heartsuit)}{=} \sum_{j < i} (-1)^{i+j} (G / e_{j}, (\Phi \setminus \{e_{j}\} ) \setminus \{e_{i-1}'\}) + \sum_{i \leq j} (-1)^{i+j}
					   (G / e_{j+1}, (\Phi \setminus \{e_{j+1}\} ) \setminus \{e_{i}'\}) \\
					   &\stackrel{(\dagger)}{=} \sum_{l \leq m} (-1)^{m+l+1} (G / e_{l}, (\Phi \setminus \{e_{l}\} ) \setminus \{e_{m}'\}) + \sum_{k < n} (-1)^{n+k-1}
					   (G / e_{k}, (\Phi \setminus \{e_{k}\} ) \setminus \{e_{n}'\}) \tag{$* *$}
	\end{align*}
	Where in $(\heartsuit)$ we used that if $j < i$ then $e_{j} = e_{j}'$ and if $i \leq j$ then $e_{j+1} = e_{j}'$, as well as the observation above.
	In $(\dagger)$ we used the substitution  $m = i-1$,  $l = j$ on the left and  $m = i$,  $k = j+1$ on the right sum.
	Comparing the sums in $(*)$ and $(* *)$ we see that they differ by a sign and thus cancel. Hence  $\partial_{C} \partial_{R} - \partial_{R} \partial_{C} = 0$.

	Combining Step 2 and 3 we get:
	\[
		\partial^2 = (\partial_{C} - \partial_{R})^2 = \partial_{C}^2 - (\partial_{C} \partial_{R} + \partial_{R} \partial_{C}) + \partial_{R}^2 = 0\qedhere
	\]
\end{proof}

Thus the spaces $(C^{n}_{\bullet})$ with the differential $\partial_{\bullet}$ form a chain complex.

\begin{eg}
	Once again we consider the graph $(G,\Phi)$ from above and calculate its boundary operator:
	\ctikzfig{./tikzit/boundary_operator}
	Where we used that $-H_{1}$ is equal to $H_{2}$ by mirroring along the vertical and applying $(2 3)$,
	$-H_{3}$ is equal to $H_2$ by exchanging inner and outer vertices, mirroring along the vertical and applying $(1 3)$ and
	$H_{4}$ is equal to $H_{2}$ by exchanging inner and outer vertices and applying $(1 3)(2 3)$.
	Where we have used the permutation $(2 3)$ on the first,  $(1 3)$ on the third, $(1 3)(2 3)$ on the fourth graph to get the result.
	\ctikzfig{./tikzit/boundary_operator2}
	Where we have used that $-G_{1}$ is equal to $-G_{3}$ by exchanging inner and outer vertices and applying $(1 3)(1 2)$,
	$G_{2}$ is equal to $-G_{3}$ by exchanging, mirroring along the vertical and applying $(1 3)$ and
	$G_{4}$ is equal to $-G_{3}$ by mirroring along the vertical and applying $(1 2)$. 

	Thus we can conclude that $\partial_{\bullet} G = 4 H_2 - 4 G_3$. Moreover we have that $H_2 - G_3 \in \im \partial_{\bullet}$ and 
	as $\partial_{\bullet}^2=0$ also $H_2 - G_3 \in \ker \partial_{\bullet}$
\end{eg}

As we did with the spine of Outer space in the introduction this chain complex
can also be viewed as a cubical complex: The $k$-cubes are given by graphs $(G,\Phi,\sigma)$
with forest size $k$ and the faces are $k-1$-cubes obtained from collapsing an edge in $\Phi$ 
or removing an edge in $\Phi$ from the forest. Collapsing the edge $e in \Phi$ is
on the opposite side of the $k$-cube of removing $e$ from $\Phi$. 
An orientation on the cube is induced by the signs from the boundary maps $\partial_{C}$ and $\partial_{R}$.

To visualize this construction we consider the following example:
\begin{eg}
	Consider again the graph $J$ from example \ref{ex:gAuto} with the forest $\Phi$ given by an edge between the top and middle vertex and
	between the left and write vertex.
	Then its $2$-dimensional cube is given as below:
	\ctikzfig{./tikzit/cubicalCCEx}
\end{eg}
\newpage
\section{Morita-Cycles}
\todo[inline]{Formulate this whole section better and in an easier way}
The goal of this section is to show that there exist a cycle in every $C_{n}$.
For this we define the Morita-Cycle graphs and show that there exists a chain of those graphs that vanishes under the boundary $\partial$.

 \begin{definition}
	 A general \emph{Morita-Cycle} $M_{n}$, for $n \geq 3$ odd, is a forested graph $(G,\Phi,\tau)$ defined as follows:
	 $G$ has $2\cdot n$ vertices which form two separate cycles each of order $n$.
	 More over each vertex of one cycle is connected via an edge to exactly one vertex of the other cycle.
	 The forest $\Phi$ are $n-1$ edges in each of the two cycles.
	 The orientation $\tau$ is some numbering of the edges in $\Phi$. 

	 We call one of the cycles the "left" and the other one the "right" cycle.
	 Moreover the missing edge in the cycle from the left/right tree in $\Phi$ is called missing left/right edge.
\end{definition}
As the cycles have $n$ edges each. The forest $\Phi$ has exactly two trees of size $n-1$ of the form of a line.
For every even $n$ it can easily be seen that $M_{n}$ has an odd automorphism (mirroring between the two cycles) and thus vanishes.

\begin{definition}
	We say a Morita-Cycle $M_{n}$ is (drawn) in standard form if the edges in the forest in the left cycle are numbered in ascending order
	starting with $1$ on one edge adjacent to the missing left edge and ending with $n-1$ on the other edge adjacent to the missing left.
	For the right cycle the edges in $\Phi$ are numbered analogous just from $n$ to $2n-1$. 
	Moreover when we draw the graph we always draw the missing left/right to the outside of the cycles.

	To make this definition more intuitive it is best to view the illustration below of a Morita cycle $M_{5}$ in standard form.
\end{definition}

\begin{proposition}
	Every Morita-Cycle $M_{n}$ in standard form is fully determined by $n$ and a permutation $\sigma \in S_{n}$. Thus we denote 
	such a Graph by $M_{n}(\sigma)$.
	Moreover every general Morita-Cycle $M_{n}$ can be written in standard form.
\end{proposition}

\begin{proof}
	We show the second part of the statement first.
	By renumbering the edges in the forest we can always get to the desired ordering. Notice that this might introduce a factor of $-1$
	if the renumbering is an odd permutation.
	Now by rotating the left/right cycle in our drawing we can also get the missing left/right in its desired place.

	Now to the first part of the proposition.
	By the properties of the standard form the left and right cycle are fully fixed and equal for all Morita-Cycles of size $2 n$.
	Thus the only difference are the edges between the two cycles.
	For these we number the nodes in the left cycle from $1$ to $n$ from top to bottom
	and on the right from $n$ to  $2 n$ also from top to bottom. Then each edge can be identified with a pair $(u,v)$.
	We now define the permutation $\sigma: {1,\ldots,n} \to {1,n}$ via $\sigma(u) = v - n$.

	On the other side if we have a permutation $\sigma \in S_{n}$. We can define the edges between the left and right cycle
	by $(i,\sigma(i) +n)$ for $i \in {1,\ldots,n}$ and get a Morita-Cycle in standard form
\end{proof}

\begin{eg}
	Below is a Morita-Cycle of order $5$ in standard form defined by the permutation $(12)(345)$.
	\ctikzfig{./tikzit/morita_cycle_5}
\end{eg}

\begin{theorem}
	For all $n \in \N_{\geq 3}$ odd it holds that 
	\[
		\partial\left(\sum_{\sigma \in S_{n}} \sgn(\sigma) M_{n}(\sigma)\right) = 0
	.\] 
\end{theorem}

We prove this statement in two parts first for $\partial_{C}$ and then for $\partial_{R}$, from which the final result follows.
Moreover when we talk about a Morita-Cycle we will always talk about it drawn in its standard form.
\begin{proof}[Proof for $\partial_C$]
	Let $(H,\Psi,\eta)$ be in the sum $\partial_{C}\left(\sum_{\sigma \in S_{n}} \sgn(\sigma) M_{n}(\sigma)\right)$.
	Then as it is an element of the boundary of some Morita-Cycle it has to
	have precisely one vertex of degree 4, in either the left or right cycle and the same cycle has order $n-1$ and only $n-2$ edges in the tree.
	W.l.o.g. we assume that the edge is missing in the left tree as else one can mirror along the horizontal
	and apply the permutation $(1 \ldots 2n-1)^{n}$ to the orientation.

	Now assume that the vertex of degree $4$ is the $k$-th one from the top. An example is shown in the figure below on the left for $k = 2$.
	Then there exist exactly two Morita-Cycles in standard form in whose boundary $H$ lies. 
	One is obtained by splitting the vertex $k$ into two vertices where each new vertex contains one edge from the forest and one connecting to the right cycle.
	Moreover the two new nodes get connected by an edge which is also added to the forest and given the number $k$ in the ordering.
	All other edges with ordering number  $> k $ get increased by one. Lets denote this Morita-Cycle by $M_{n}(\sigma)$.
	The other one is obtained in the same way however the two edges connecting to the right cycle are permuted i.e.
	if the new node is numbered by $k+1$ then this Morita-Cycle equals $M_{n}((k\ k+1)\sigma)$.
	Examples for both are also shown in the figure below in the middle and right respectively.

	Thus we see that the two permutations $\sigma$ and $(k\ k+1) \sigma$ have opposite parity and thus $M_{n}(\sigma)$ and $M_{n}((k\ k+1) \sigma)$ have
	opposite sign in the sum from the theorem. Thus the elements in their boundary corresponding to $H$ also have opposite sign and hence cancel.

	As $H$ was an arbitrary element of the sum we get that every summand has coefficient $0$ and the sum vanishes.
	\ctikzfig{./tikzit/morita_cycle_dC}
\end{proof}

\begin{proof}[Proof for $\partial_{R}$]
	Again let $(H,\Psi,\eta)$ be in the sum $\partial_{R}\left(\sum_{\sigma \in S_{n}} \sgn(\sigma) M_{n}(\sigma)\right)$.
	Let $\sigma$ be the edge assignment between the left and right cycle as described in the standard form.
	Then $H$ is a Morita-Cycle in standard form missing one edge in either the left or right tree of $\Psi$.
	W.l.o.g. we assume that the edge is missing in the left tree as else one can mirror along the horizontal
	and apply the permutation $(1 \ldots 2n-1)^{n}$ to the orientation.

	Now assume the $k$-th edge from the top is missing from $\Psi$; an example is shown in the figure below on the left for $k = 2$.
	Then there are exactly two Morita-Cycles in standard form which have $H$ in their boundary.
	One where the $k$-th edge has been added to the forest with number $k$ in the ordering i.e. $M_{n}(\sigma)$.
	And the other where the left most edge has been added, the numbering in the left cycle been changed such that $1$ is the first edge 
	after the missing one, the left most edge has order $n-k$ and so on.
	Notice that the second graph is not drawn in standard form. To bring it in standard form one has to "rotate" the left cycle
	by $k$ nodes counter clockwise. Each of this rotation induces the permutation $(1 \ldots n)$ on the edges connecting the left and right cycle i.e.
	the resulting Morita cycle is given by $M_{n}(\tau) := M_{n}(\sigma (1 \ldots n)^{k})$.
	Examples for both are also shown in the figure below in the middle and right respectively.
	
	As $(1 \ldots n)$ for $n$ odd has even parity $\tau$ and $\sigma$ have the same parity and thus $M_{n}(\tau)$ and $M_{n}(\sigma)$ have the same sign
	in the sum from the theorem. 
	Finally taking the boundary the elements corresponding to $H$ have different orientations.
	The element in $\partial_{R}(M_{n}(\sigma))$ is exactly $H$ with sign $(-1)^{k}$. The element in $\partial_{R}(M_{n}(\tau))$ however 
	differs from $H$ by the permutation $(1 \ldots n-2)^{k - 1}$ with even parity as $n-2$ is odd and has sign $(-1)^{n-k}$.
	Now for $n$ odd if $k$ is even $n-k$ is odd and vice versa. Thus the elements corresponding to $H$ have opposite sign and cancel.

	As $H$ was an arbitrary element in the sum we get that every summand has coefficient $0$ and the sum vanishes.

	\ctikzfig{./tikzit/morita_cycle_dR}

	Thus we have shown the result for both $\partial_{C}$ and $\partial_{R}$ and as $\partial = \partial_{C} - \partial_{R}$ it also follows for $\partial$.
\end{proof}

\newpage
\printbibliography

%%\input{../header}

\section{Draftparts}

\begin{definition}
	A \emph{graph} $G$ is a finite $1$-dimensional CW complex. The set of edges is denoted by $E(G)$, the set of vertices by  $V(G)$ and the set of half edges by  $H(G)$.
	We call a graph \emph{connected} if the CW complex is connected in the topological sense.
	A graph is said to be \emph{$n$-valent} if every vertex has valency $n$ i.e. for every vertex the number of edges incident is  $n$.

	Lastly a \emph{tree} is a graph which contains no loops and a \emph{forest} is a collection of disjoint trees.
\end{definition}

On trivalent connected graphs we call an \emph{orientation} a choice of cyclic orders of all vertices up to an even number of changes.

\begin{definition}
	A \emph{forested graph} is a pair $(G, \Phi)$, where $G$ is a finite connected trivalent graph and $\Phi$ is an oriented forest which contains all vertices of $G$. 
\end{definition}

\begin{definition}
Let $(G,\Phi)$ be a forested graph and let  $e \in \Phi$. Moreover let $(G_{e},\Phi_{e})$ be the graph where $e$ has been collapsed.
Then there exist exactly two other graphs whose edge collapse results in $(G_{e},\Phi_{e})$. This is visualised in the figure below.
Where  $1,2,3,4$ represent the rest of the graph.

\ctikzfig{./tikzit/IHXRelator}
Now the vector 
\[
	(G,\Phi) + (G',\Phi') + (G'',\Phi'')
\]
is called the \emph{basic IHX relator} associated to $(G,\Phi,e)$.
\end{definition}

Denote by $\widehat{\fg}_{k}$ the vector space spanned by all forested graphs containing $k$ trees modulo the relations $(G,\Phi) = -(G,-\Phi)$.
Moreover let $\fg_{k}$ be the quotient of $\widehat{\fg}_{k}$ modulo the subspace spanned by all basic IHX relators.


\begin{definition}
	Let $\widehat{\partial}_E(G,\Phi): \widehat{\fg}_{k} \to \widehat{\fg}_{k-1}$ be given by
\begin{align*}
	\widehat{\partial}_{E}(G,\Phi) = \sum (G, \Phi \cup e)
.\end{align*}
where the sum is over all edges $e$ of $G \setminus \Phi$ such that $\Phi \cup e$ is still a forest.
Notice that this only happens if the two vertices of  $e$ lie in different trees of $\Phi$. Thus $\Phi \cup e$ has $k-1$ components.
The orientation of $\Phi \cup e$ is determined by ordering the edges of $\Phi$ with labels  $1,\ldots,k$ consistent with its orientation
and then labeling the new edge $e$ with  $k+1$.

Now let the boundary map $\partial_{E}: \fg_{k} \to \fg_{k-1}$ be the map induced by $p \circ \widehat{\partial}_{E}$ where $p$ is the quotient map $\widehat{\fg_{k}} \to \fg_{k}$.
\end{definition}

\begin{proposition}
	$\partial_{E}$ is well-defined and $\partial_{E}^2 = 0$.
\end{proposition}

\begin{proof}
	
\end{proof}

The \emph{forested graph complex} is thus defined as the sequence $\fg_{k}$ with boundary map $\partial_{E}$ and is well-defined by the above proposition.

\end{document}


\end{document}
