%\input{../header}
%%% DOCUMENT TYPE %%%%%%%%%%%%%%%%%%%%%%%%%%%%%%%%%%%%%%%%%%%%%%%%%%%%%%%%%%%%%%

\documentclass[11pt, a4paper]{article}

%%% PACKAGES %%%%%%%%%%%%%%%%%%%%%%%%%%%%%%%%%%%%%%%%%%%%%%%%%%%%%%%%%%%%%%%%%%%

% Encoding

\usepackage[utf8]{inputenc}
\usepackage[T1]{fontenc}
\usepackage{lmodern}

% Geometry

\usepackage{geometry} % edit margins of paper
\usepackage{setspace} % edit line spacing
\usepackage{fancyhdr} % header, footer
\usepackage{titlesec} % edit format of titles

% Images
\usepackage[pdftex]{graphicx} % image locations

% Visual

\usepackage[dvipsnames]{xcolor} % colors
\usepackage{pgf,tikz} % graphics
\usetikzlibrary{quotes}
\usetikzlibrary{cd}
\usetikzlibrary{babel}

\usepackage[framemethod=tikz]{mdframed} % frames, better theorems

% Math

\usepackage{amsmath} % math tools
\usepackage{amssymb} % math symbols
\usepackage{amsthm} % thereoms
\usepackage{mathtools} % math tools
\usepackage{xfrac} %More Fractions

% Referencing

\usepackage{nameref}
\usepackage{hyperref}
\usepackage{cleveref}

% Useful

\usepackage[shortlabels]{enumitem} % enumerations

% Other

\usepackage{lastpage} % get number of last page
\usepackage{physics}
%\usepackage{bbm}
\usepackage[makeroom]{cancel} %crossing stuff out
\usepackage{stmaryrd}
\usepackage{tabu}

%%% MARGINS %%%%%%%%%%%%%%%%%%%%%%%%%%%%%%%%%%%%%%%%%%%%%%%%%%%%%%%%%%%%%%%%%%%%

\geometry{a4paper, left=25mm, right=25mm, top=10mm, bottom=20mm, includehead}

%%% COLORS %%%%%%%%%%%%%%%%%%%%%%%%%%%%%%%%%%%%%%%%%%%%%%%%%%%%%%%%%%%%%%%%%%%%%

%%% TITLES %%%%%%%%%%%%%%%%%%%%%%%%%%%%%%%%%%%%%%%%%%%%%%%%%%%%%%%%%%%%%%%%%%%%%

\colorlet{color-section}                {BrickRed}
\colorlet{color-subsection}             {BrickRed}

%%% MATH BOXES %%%%%%%%%%%%%%%%%%%%%%%%%%%%%%%%%%%%%%%%%%%%%%%%%%%%%%%%%%%%%%%%%

\colorlet{color-definition}             {SpringGreen!20}
\colorlet{color-theorem}                {Apricot!13}
\colorlet{color-proposition}            {Apricot!13}
\colorlet{color-corollary}              {Apricot!13}
\colorlet{color-lemma}                  {Apricot!13}
\colorlet{color-attention}              {OrangeRed!13}
\colorlet{color-remark}                 {Gray!5}
\colorlet{color-example}                {Lavender!7}
\colorlet{color-notation}               {Gray!5}
\colorlet{color-convention}             {Gray!5}
\colorlet{color-claim}                  {SkyBlue!7}
% \colorlet{color-proof}                  {FILL COLOR HERE}


%%% CAPTIONS %%%%%%%%%%%%%%%%%%%%%%%%%%%%%%%%%%%%%%%%%%%%%%%%%%%%%%%%%%%%%%%%%%%

%%% CAPTION DEFINITION %%%%%%%%%%%%%%%%%%%%%%%%%%%%%%%%%%%%%%%%%%%%%%%%%%%%%%%%%

\newcommand*{\definitionname}{Definition}
\newcommand*{\theoremname}{Theorem}
\newcommand*{\propositionname}{Proposition}
\newcommand*{\corollaryname}{Corollary}
\newcommand*{\lemmaname}{Lemma}
\newcommand*{\remarkname}{Remark}
\newcommand*{\examplename}{Example}
\newcommand*{\attentionname}{Attention}
\newcommand*{\notationname}{Notation}
\newcommand*{\conventionname}{CConvention}
\newcommand*{\claimname}{Claim}


%%% SHORTCUTS %%%%%%%%%%%%%%%%%%%%%%%%%%%%%%%%%%%%%%%%%%%%%%%%%%%%%%%%%%%%%%%%%%

%%% SINGLE SYMBOLS %%%%%%%%%%%%%%%%%%%%%%%%%%%%%%%%%%%%%%%%%%%%%%%%%%%%%%%%%%%%

% Logic

% \forall exists
% \exists exists
% \lnot exists
% \lor exists
% \land exists
\newcommand*{\limp}{\rightarrow}
\newcommand*{\limps}{\; \limp \;} % \limp with some space around
\newcommand*{\leqv}{\leftrightarrow}
\newcommand*{\leqvs}{\; \leqvs \;} % \leqv with some space around
\newcommand*{\lTri}{\vartriangleleft}
\newcommand*{\on}[1]{\operatorname{#1}}
\newcommand*{\fg}{f\mathcal{G}}
% Meta Logic

% \implies exists
% \iff exists

% Colon Stuff

\newcommand*{\cl}{\colon}
\newcommand*{\cleq}{\coloneqq}
\newcommand*{\eqcl}{\eqqcolon}

% Sets

\newcommand*{\N}{\mathbb{N}} % natural numbers
\newcommand*{\Z}{\mathbb{Z}} % integers
\newcommand*{\Q}{\mathbb{Q}} % rational numbers
\newcommand*{\R}{\mathbb{R}} % real numbers
\newcommand*{\C}{\mathbb{C}} % complex numbers
\newcommand*{\F}{\mathbb{F}} % finite field

%%% MATH OPERATORS %%%%%%%%%%%%%%%%%%%%%%%%%%%%%%%%%%%%%%%%%%%%%%%%%%%%%%%%%%%%%

% General

\DeclareMathOperator{\id}{id}
\DeclareMathOperator{\sgn}{sgn}
%Analysis
\DeclareMathOperator{\vol}{vol}
\DeclareMathOperator{\supp}{supp}
\let\grad\relax
\DeclareMathOperator{\grad}{grad}
\DeclareMathOperator{\rot}{rot}
\let\div\relax
\DeclareMathOperator{\div}{div}
%LinAlg
\let\ker\relax
\DeclareMathOperator{\ker}{Ker}
\DeclareMathOperator{\eig}{Eig}
\DeclareMathOperator{\im}{Im}
\let\hom\relax
\DeclareMathOperator{\hom}{Hom}
\DeclareMathOperator{\End}{End}
\DeclareMathOperator{\mat}{Mat}
%Algebra
\let\gcd\relax
\DeclareMathOperator{\gcd}{ggT}
\let\ev\relax
\DeclareMathOperator{\ev}{ev}
\DeclareMathOperator{\charak}{char}
\DeclareMathOperator{\quot}{Quot}
\DeclareMathOperator{\sym}{Sym}
\DeclareMathOperator{\bij}{Bij}
\DeclareMathOperator{\GL}{GL}
\DeclareMathOperator{\SL}{SL}
\DeclareMathOperator{\gal}{Gal}
\DeclareMathOperator{\aut}{Aut}
\DeclareMathOperator{\out}{Out}
\DeclareMathOperator{\SO}{SO}
\DeclareMathOperator{\irr}{irr}
%%% TEMPLATES %%%%%%%%%%%%%%%%%%%%%%%%%%%%%%%%%%%%%%%%%%%%%%%%%%%%%%%%%%%%%%%%%%

% General

% write a set definition like: { #1 | #2 }
\newcommand*{\sdef}[2]{
  \{#1 \mid #2\}
}

% write a nice map definition
\newcommand*{\mdef}[5]{
  \begin{align*}
    #1 \cl #2 &\to     #3 \\
           #4 &\mapsto #5
  \end{align*}
}

\newcommand*{\nstack}[2]{
	\begin{array}{c}
		#1\\
		#2\\
	\end{array}
}


%%% FORMATTING %%%%%%%%%%%%%%%%%%%%%%%%%%%%%%%%%%%%%%%%%%%%%%%%%%%%%%%%%%%%%%%%%

%%% HEADER, FOOTER %%%%%%%%%%%%%%%%%%%%%%%%%%%%%%%%%%%%%%%%%%%%%%%%%%%%%%%%%%%%%

\pagestyle{fancy}
\renewcommand{\headrule}{}
%\renewcommand{\chaptermark}[1]{\markboth{#1}{}}
\fancyhf{} % clear everything
%\fancyhead[L]{\rightmark}
%\chead{\bfseries Zusammenfassung Algebra I/II}
%\rhead{Seite \thepage /\pageref*{LastPage}}
%\lfoot{}
\fancyfoot[C]{\thepage}
%\fancyfoot[R]{\thepage}

%%% TITLE FORMAT %%%%%%%%%%%%%%%%%%%%%%%%%%%%%%%%%%%%%%%%%%%%%%%%%%%%%%%%%%%%%%%

\setcounter{secnumdepth}{2}

%\titleformat{\chapter}[hang]
%{\normalfont\huge\bfseries}{\chaptertitlename\ \thechapter:}{20pt}{\Huge}
\titleformat{\section}
{\normalfont\LARGE\bfseries}{\thesection}{1em}{}
\titleformat{\subsection}
{\normalfont\large\bfseries}{\thesubsection}{1em}{}
\titleformat{\subsubsection}
{\normalfont\normalsize\bfseries}{\thesubsubsection}{1em}{}
\titleformat{\paragraph}[runin]
{\normalfont\normalsize\bfseries}{\theparagraph}{1em}{}
\titleformat{\subparagraph}[runin]
{\normalfont\normalsize\bfseries}{\thesubparagraph}{1em}{}

%%% SPACING %%%%%%%%%%%%%%%%%%%%%%%%%%%%%%%%%%%%%%%%%%%%%%%%%%%%%%%%%%%%%%

% Titles

%\titlespacing*{\chapter}{0pt}{0pt}{15pt}
\titlespacing*{\section}{0pt}{3.5ex plus 1ex minus .2ex}{2.3ex plus .2ex}
\titlespacing*{\subsection}{0pt}{3.25ex plus 1ex minus .2ex}{1.5ex plus .2ex}
\titlespacing*{\subsubsection}{0pt}{3.25ex plus 1ex minus .2ex}{1.5ex plus .2ex}
\titlespacing*{\paragraph}{0pt}{1.25ex plus 1ex minus .2ex}{1em}
\titlespacing*{\subparagraph}{\parindent}{3.25ex plus 1ex minus .2ex}{1em}

% Text, Paragraphs

\setstretch{1.05} % scaling of space between lines
\setlength{\parindent}{0pt} % indentation of paragraphs
\setlength{\parskip}{4.0pt plus 1.0pt minus 1.0pt} % space between paragraphs
%\setlength{\parskip}{0pt}

%%% SYMBOLS USED BY NUMBERINGS, ENVIRONMENTS, ... %%%%%%%%%%%%%%%%%%%%%%%%%%%%%%

% \renewcommand*\qedsymbol{$\blacksquare$} % alternative QED symbol
%\renewcommand{\thefootnote}{\arabic{footnote}} % normal footnotes on page
%\renewcommand{\thempfootnote}{\fnsymbol{mpfootnote}} % footnotes on minipages, e.g. in mdframed environments

%%% LISTS, ENUMERATIONS %%%%%%%%%%%%%%%%%%%%%%%%%%%%%%%%%%%%%%%%%%%%%%%%%%%%%%%%

% 'itemize'

\setlist[itemize]{noitemsep, topsep=0pt}

% 'enumerate'

\setlist[enumerate]{noitemsep, topsep=0pt}
% no special settings at the moment

% 'description'

% no special settings at the moment

% 'axioms'

%\newlist{axioms}{enumerate}{2}
%\setlist[axioms]{itemsep=0pt,label*=\arabic*.}

%%% GENERAL SYMBOLS %%%%%%%%%%%%%%%%%%%%%%%%%%%%%%%%%%%%%%%%%%%%%%%%%%%%%%%%%%%%
\newcommand\danger{\raisebox{\depth}{{\fontencoding{U}\fontfamily{futs}\selectfont\char 66\relax}}}
\newcommand\contra{\scalebox{1.5}{$\lightning$}}

%%% MDFRAMED PATCH %%%%%%%%%%%%%%%%%%%%%%%%%%%%%%%%%%%%%%%%%%%%%%%%%%%%%%%%%%%%%

\usepackage{xpatch}

\makeatletter
\xpatchcmd{\endmdframed}
  {\aftergroup\endmdf@trivlist\color@endgroup}
  {\endmdf@trivlist\color@endgroup\@doendpe}
  {}{}
\makeatother

%%% MDFRAMED STYLES %%%%%%%%%%%%%%%%%%%%%%%%%%%%%%%%%%%%%%%%%%%%%%%%%%%%%%%%%%%%

% thick frame and bar for title

%\mdfdefinestyle{style-box}{
%  skipabove=1.5ex plus .5ex minus .2ex,
%  skipbelow=1ex plus .2ex minus .2ex,
%  linewidth=2pt,
%  linecolor=Gray!20,
%   roundcorner=3pt,
%  innerleftmargin=0.5\baselineskip,
%  innerrightmargin=0.5\baselineskip,
%  innertopmargin=0.4\baselineskip,
%  innerbottommargin=0.4\baselineskip,
%  frametitlebackgroundcolor=Gray!20,
%  frametitleaboveskip=0.3pt,
%  frametitlebelowskip=0.3pt,
%  theoremseparator=,
%  theoremspace=\hfill,
%  theoremtitlefont=\mdseries\scshape,
%  nobreak=true
%}

% highlighted background

%\mdfdefinestyle{style-background}{
%  skipabove=1.5ex plus .5ex minus .2ex,
%  skipbelow=1ex plus .2ex minus .2ex,
%  hidealllines=true,
%  backgroundcolor=Gray!5,
%  innerleftmargin=0.5\baselineskip,
%  innerrightmargin=0.5\baselineskip,
%  innertopmargin=0.4\baselineskip,
%  innerbottommargin=0.4\baselineskip,
%}

% thin frame

%\mdfdefinestyle{style-leftline}{
%  skipabove=1.5ex plus .5ex minus .2ex,
%  skipbelow=1ex plus .2ex minus .2ex,
%  linewidth=1pt,
%  linecolor=Gray!50,
%  topline=false,
%  bottomline=false,
%  rightline=false,
%  innerleftmargin=0.5\baselineskip,
%  innerrightmargin=0,
%  innertopmargin=0.2\baselineskip,
%  innerbottommargin=0.0\baselineskip,
%}

%%% ENVIRONMENTS %%%%%%%%%%%%%%%%%%%%%%%%%%%%%%%%%%%%%%%%%%%%%%%%%%%%%%%%%%%%%%%



% Definition

\theoremstyle{definition}
\newtheorem*{definition}{\definitionname}
\newtheorem*{attention}{\danger\ \attentionname}
\newtheorem*{eg}{\examplename}

\theoremstyle{plain}
\newtheorem*{theorem}{\theoremname}
\newtheorem*{proposition}{\propositionname}
\newtheorem*{corollary}{\corollaryname}
\newtheorem*{lemma}{\lemmaname}

\theoremstyle{remark}
\newtheorem*{remark}{\remarkname}
\newtheorem*{claim}{\claimname}
\newtheorem*{notation}{\notationname}
\newtheorem*{convention}{\conventionname}



%\mdtheorem[
%  style=style-box,
%  linecolor=color-definition,
%  frametitlebackgroundcolor=color-definition
%]{definition}{\definitionname}[section]

% Theorem

%\mdtheorem[
%  style=style-box,
%  linecolor=color-theorem,
%  frametitlebackgroundcolor=color-theorem,
%  font=\itshape
%]{theorem}{\theoremname}[section]

% Proposition

%\mdtheorem[
%  style=style-box,
%  linecolor=color-proposition,
%  frametitlebackgroundcolor=color-proposition,
%  font=\itshape
%]{proposition}[theorem]{\propositionname}

% Corollary

%\mdtheorem[
%  style=style-box,
%  linecolor=color-corollary,
%  frametitlebackgroundcolor=color-corollary,
%  font=\itshape
%]{corollary}[theorem]{\corollaryname}

% Lemma

%\mdtheorem[
%  style=style-box,
%  linecolor=color-lemma,
%  frametitlebackgroundcolor=color-lemma,
%  font=\itshape
%]{lemma}[theorem]{\lemmaname}

%\mdtheorem[
%  style=style-box,
%  linecolor=color-attention,
%  frametitlebackgroundcolor=color-attention,
%  font=\itshape
%]{attention}[theorem]{\danger \attentionname}

%\theoremstyle{remark}

% Remark

%\newtheorem*{remark}{\remarkname}
%\surroundwithmdframed[
%  style=style-background,
%  backgroundcolor=color-remark
%]{remark}

% Example

%\newtheorem*{eg}{\examplename}
%\surroundwithmdframed[
%  style=style-background,
%  backgroundcolor=color-example
%]{eg}

%Notation

%\newtheorem*{notation}{\notationname}
%\surroundwithmdframed[
%  style=style-background,
%  backgroundcolor=color-notation
%]{notation}

%Convention

%\newtheorem*{convention}{\conventionname}
%\surroundwithmdframed[
%  style=style-background,
%  backgroundcolor=color-convention
%]{convention}

%Claim

%\newtheorem*{claim}{\claimname}
%\surroundwithmdframed[
%  style=style-background,
%  backgroundcolor=color-claim
%]{claim}

% Proof

%\surroundwithmdframed[
%  style=style-leftline
%]{proof}

%%% TEXT FORMATTING %%%%%%%%%%%%%%%%%%%%%%%%%%%%%%%%%%%%%%%%%%%%%%%%%%%%%%%%%%%%

% definitions
\let\epsilon\varepsilon
\renewcommand\emptyset{\varnothing}
\let\implies\Rightarrow
\let\impliedby\Leftarrow
\let\ForAll\forall
\renewcommand\forall{\;\ForAll}
\let\Exists\exists
\renewcommand\exists{\;\Exists}

\newcommand*{\df}[1]{\colorbox{color-definition}{\emph{#1}}}


%%% LANGUAGE %%%%%%%%%%%%%%%%%%%%%%%%%%%%%%%%%%%%%%%%%%%%%%%%%%%%%%%%%%%%%%%%%%%

%%%% SETUP %%%%%%%%%%%%%%%%%%%%%%%%%%%%%%%%%%%%%%%%%%%%%%%%%%%%%%%%%%%%%%%%%%%%%%

\usepackage[english]{babel}
\usetikzlibrary{english}
\usepackage[babel,english=quotes]{csquotes}

%%% CAPTION REDEFINITION %%%%%%%%%%%%%%%%%%%%%%%%%%%%%%%%%%%%%%%%%%%%%%%%%%%%%%%

\renewcommand{\figurename}{Figure} % not really necessary, since Babel already implements this
\renewcommand{\tablename}{Table} % not really necessary, since Babel already implements this
\renewcommand*{\proofname}{Proof} % not really necessary, since Babel already implements this
\renewcommand*{\definitionname}{Definition}
\renewcommand*{\theoremname}{Theorem}
\renewcommand*{\propositionname}{Proposition}
\renewcommand*{\corollaryname}{Corollary}
\renewcommand*{\lemmaname}{Lemma}
\renewcommand*{\remarkname}{Remark}
\renewcommand*{\examplename}{Example}
\renewcommand*{\attentionname}{Attention}
\renewcommand*{\notationname}{Notation}
\renewcommand*{\conventionname}{Convention}
\renewcommand*{\claimname}{Claim}

%%% HYPHENATION %%%%%%%%%%%%%%%%%%%%%%%%%%%%%%%%%%%%%%%%%%%%%%%%%%%%%%%%%%%%%%%%



\usepackage{todonotes}
\usepackage{./tikzit/tikzit}
\usepackage{biblatex}
\input{./tikzit/test.tikzstyles}
\addbibresource{./references.bib}
\usepackage[stable]{footmisc}

\begin{document}	

\section{Introduction\footnotemark}
\footnotetext{This section follows closely Vogtmann's survey article \cite{vogtmann16}}
Free groups are one of the most basic examples of infinite finitely-generated groups.
The automorphism groups of free groups are of particular interest since they are tied to many other areas of mathematics.
The automorphism group $\aut(G)$ of a group $G$ can be separated into the inner automorphism group $\on{Inn}(G)$,
the group of automorphisms that arise from conjugation, and the outer automorphism group $\out(G)$, the quotient of  $\aut(G)$ by $\on{Inn}(G)$.

The standard approach to study $\out(G)$ has been to study it via its action on some topological space.
For $\out(F_{n})$ the outer automorphism group of the free group on $n$ generators this space $\mathcal{X}_{n}$ known as "Outer space"
has been introduced by Culler and Vogtmann in []. As the points of the Outer space correspond to finite graphs with fundamental group $F_{n}$,
the structure of Outer Space is connected to such diverse areas as the study of Lie algebras of derivations, degenerations of algebraic varieties,
the computation of Feynman integrals, and the statistics of phylogenetic trees.

A central concept of the Outer space is its spine $K_{n}$ which is an equivariant deformation retract of $\mathcal{X}_{n}$.
By being able to compute the rational homology of $K_{n} / \out(F_{n})$ on can also compute $\out(F_{n})$.
By identifying the spine $K_{n}$ with a cube complex one arrives at the forested graph complex
which has been introduced by Conant and Vogtmann in \cite{conant03}.

The forested graph complex is also the main focus of this paper.
We will begin by defining basic concepts of graphs, the Outer space and the forested graph complex as well as show the connection via the cube complex.

Afterwards we will focus on Morita's classes which are an infinite sequence of cocycles representing potentially nontrivial cohomology classes $\mu_{k} \in H^{4k}(\out(F_{2k+2})$.



\section{Basic Definitions}
\subsection{Graphs}
\begin{definition}
	A \emph{graph} $G$ is a finite $1$-dimensional CW complex. The set of edges is denoted by $E(G)$, the set of vertices by  $V(G)$.
	We call an edge having the same start and end vertex a loop.

	We call a graph \emph{connected} if the CW complex is connected in the topological sense.
	A graph is $n$-edge-connected if it remains connected after removing  $n-1$ arbitrary edges.

	A graph is said to be \emph{$n$-regular} if every vertex has valency $n$ i.e. for every vertex the number of incident edges is $n$.
	The valency of a vertex $v \in G$ is often also called degree of $v$ and denoted by $\deg(v)$.

	For a subset of edges $\Phi$ of $G$ we denote by $G / \Phi$ the graph quotient, which is the quotient space of the CW complex $G$ over its topological subspace $\Phi$.
\end{definition}

\begin{remark}
	Note that in classical graph theory these types of graphs are called multigraphs, as they are allowed to have multiple edges between vertices as well as loops.
	The word graph there normally refers to simple graphs which don't allow multi-edges and loops.

	In the context of algebraic topology however multigraphs are needed and thus the word graph here denotes multigraphs.
\end{remark}

\begin{definition}
	A subgraph $G'$ of a graph $G$ is a subcomplex of the CW-complex $G$. As a subcomplex is itself a CW-complex of dimension smaller or equal to the original complex,
	 $G'$ is itself a graph.

	A cycle in a graph $G$ is a subgraph that is homeomorphic to $S^1$. A tree is a
	connected graph containing no cycles. A forest is a collection of disjoint trees.
\end{definition}

\begin{theorem}\label{thm:fg_graph}
	Let $G$ be a graph. Then its fundamental group $\pi_{1}(G)$ is isomorphic to a free group.
\end{theorem}
By the theorem it makes sense to define the \emph{rank} of a graph $\rank(G)$ as the rank of its fundamental group.

The following proof is from \cite[p. 43f]{hatcher00}
\begin{proof}
	Let $G$ be a graph. W.l.o.g. $G$ is connected as else we consider each connected component separately. 
	Then let $T$ be a spanning tree on $G$ i.e. $T$ is a tree containing every vertex of $G$.
	Then $T$ is contractible.
	Now choose for every $e_{\alpha} \in E \setminus T$ an open neighborhood $A_{\alpha}$ of $T \cup e_{\alpha}$ that deformation retracts onto $T \cup e_{\alpha}$.
	The intersection of such $A_{\alpha}$ is $T$ and thus contractible. Moreover as $G$ is connected as a graph $A_{\alpha}$ and $T$ are path connected.
	Now the $A_{\alpha}$ form an open cover of $G$ and as $T$ is simply connected by Van Kampen's Theorem we get that $\pi_{1}(G) = *_{\alpha} \pi_{1}(A_{\alpha})$.
	Finally $A_{\alpha}$ deformation retracts onto $S^{1}$ and thus $\pi_{1}(A_{\alpha}) = \Z$. Now there are exactly $\abs{E} - \abs{T}$ many $A_{\alpha}$,
	which as $T$ is a spanning tree results in $\pi_1(G)$ being free on $\abs{E} - \abs{V} + 1$ generators.
\end{proof}

To understand the rank better we will need the following definitions
\begin{definition}
	For a finite CW-complex $X$ the \emph{Euler Characteristic} is defined as the alternating sum
	\[
		\chi(X) = k_0 - k_1 + k_2 - \ldots
	\] 
	where $k_{i}$ denotes the number of cells of dimension $i$ in the complex $X$.
\end{definition}
For graphs we thus get $\chi (G) = k_0 - k_1$, as they are $1$-dimensional which is equal to $\chi(G) = \abs{V} - \abs{E}$.

\begin{definition}
	Let $G$ be a graph. Then its cycle space is the set of even-degree subgraphs of $G$. 
	This space forms a vector space over $\F_2$ where the vector addition is given by thesymmetric difference of two or more subgraphs.
	A basis of this space is called cycle basis and two cycles are independent if they are linearly independent in the vector space.
\end{definition}

The following proposition 
\begin{proposition}\label{prop:rank}
	Let $G$ be a connected graph. Then the following are equal:
	\begin{enumerate}
		\item The rank of $G$
		\item The number of independent cycles in $G$ i.e. the size of the cycle basis of $G$.
		\item The first Betti number i.e. the rank of $H_{1}(G)$
		\item $1 - \chi(G) = \abs{E} - \abs{V} + 1$
	\end{enumerate}
\end{proposition}

For the proof we will need the following Lemma:
\begin{lemma}
	Let $A$ be a set. Then the abelianization of the free group on $A$ is isomorphic the free abelian group of $A$.
\end{lemma}

\begin{proof}
	Consider the space $X = \bigvee_{a \in A} S^{1}$. Then by Van Kampen's Theorem $\pi_1(X) = *_{a \in A} \Z$.
	On the other hand we have $H_1(X) = \bigoplus_{a \in A} H_1(S_1) = \bigoplus_{a \in A} \Z$ which follows from the relative Homeomorphism Theorem.
	Now using Hurewicz Theorem we get that the abelianization of$\pi_1(X)$ is isomorphic to $H_1(X)$ and thus the desired statement
\end{proof}

\begin{proof}[Proof of the Theorem]
	(1) = (4): This was shown in the proof of Theorem \ref{thm:fg_graph}.\\
	(1) = (3): From Hurewicz Theorem we get that the abelianization of $(\pi_{1}(G)$ is equal to $H_{1}(G)$
	and thus by the previous Lemma that the rank of $\pi_1(G)$ is equal to the rank of $H_1(G)$ which is the first Betti number.\\
	(4)= (2)\footnote{This proof is based on Harary's proof in \cite[p. 37-40]{harary69}.}:
		Consider again the sets $A_{\alpha}$ from the proof of Theorem \ref{thm:fg_graph}. Then each of them deformation retracts onto a cycle in $G$.
	Let $Z(T)$ be the set of cycles obtained in this way. Then  $Z(T)$ is independent as each cycle contains an edge not contained in any other cycle.
	Moreover every cycle $Z$ in $G$ can be written as the symmetric difference over the cycles corresponding to the edges in $(E \setminus T) \cap Z$.
	Thus $Z(T)$ spans the cycle space and consequently is a cycle basis. Now the size of $Z(T)$ is given by $\abs{E} - \abs{V} + 1$ and thus we conclude.
\end{proof}

\begin{definition}
	Let $G,H$ be two graphs. A map $f: V(G) \to V(H)$ is said to be a graph isomorphism if $f$ is a bijection such that
	\[
		(u,v) \in E(G) \Leftrightarrow (f(u),f(v)) \in E(H)
	.\] 
\end{definition}

\begin{eg}\label{ex:gAuto}
	Consider the following graphs.

	\ctikzfig{./tikzit/graph_automorphism}
	Then $G$ is $3$-connected and $3$-regular. Moreover $G$ is isomorphic to $H$.
	An isomorphism between them is given by mapping the same colored nodes to each other.
	The graph quotient $G / \{e,f,g\}$ is given by $J$.
\end{eg}

Finally we introduce the notion of degree of a graph also sometimes called excess.
\begin{definition}
	Let $G$ be a connected graph of rank $n$ with vertex-valency $\geq 3$. Its \emph{degree} is given by
	\[
		\deg(G) := \sum_{v \in V(G)} (\deg(v) - 3)
	.\] 
\end{definition}

\begin{proposition}
	We have the following identities
	\begin{enumerate}
		\item $\deg(G) = 2 \abs{E} - 3 \abs{V}$
		\item $\abs{V} = 2n -2 - \deg(G)$
		\item $\abs{E} = 3n - 3 - \deg(G)$
		\item $G$ is $3$-regular $\Leftrightarrow \deg(G) = 0$.
	\end{enumerate}	
\end{proposition}

\begin{proof}
	A general fact in graph theory is that $2 \abs{E} = \sum_{v \in V} \deg(V)$.
	Combining this with the definition of degree we directly get the first identity.
	Using Proposition \ref{prop:rank} and the first identity we have
	\begin{align*}
		2 n -2 - \deg(G) &= 2 \abs{E} - 2 \abs{V} + 2 - 2 - 2\abs{E} + 3 \abs{V} = \abs{V}\\
		3 n -3 - \deg(G) &= 3 \abs{E} - 3 \abs{V} + 3 - 3 - 2\abs{E} + 3 \abs{V} = \abs{E}
	\end{align*}
	which proves the second and third. The last statement follows as every element in the sum of the degree is positive.
	Thus $\deg(G) = 0$ if and only if every term is $0$ and thus iff $\deg(v) = 0 \forall v \in V$.
\end{proof}

\subsection{The Outer space}
We closely follow Vogtmann's definition from \cite[p. 2 ff.]{vogtmann16}
\begin{definition}
	By a \emph{metric graph} we mean a finite connected graph with positive real edge lengths, equipped with the path metric.
	We fix a model rose $R_{n}$ (a graph with one vertex and $n$ petals), and identify the petals of $R_{n}$ with a homotopy
	equivalence $g: R_{n} \to  G$ called a \emph{marking}; the marking serves to identify the fundamental group of $G$ with $F_{n}$.
	Marked graphs $(g,G)$ and $(g',G')$ are considered the same if there is an isometry  $f: G \to G'$ with $g \circ g$ homotopic to $g'$.

	To get a finite dimensional space we assume $G$ has no uni- and bivalent vertices (see Theorem \ref{thm:finGenCn}).
	Moreover we normalize our objects i.e. we assume the sum of edge lengths to be $1$ and assume that $G$ is $2$-connected.
\end{definition}
To make  $\mathcal{X}_{n}$ a space we need to define a topology. We proceed as follows: 
For every marked graph $(g,G)$ we define the open simplex $\sigma(g,G)$ as the set obtained by varying the edge lenghts of $G$,
keeping their sum equal to $1$.
The simplex  $\sigma(g',G')$ is then a \emph{face} of $\sigma(g,G)$ if $(g',G')$ can be obtained from $(g,G)$ by collapsing some edges to points.

Finally $\mathcal{X}_{n}$ is the quotient space obtained from the disjoint uniion of the open simplices $\sigma(g,G)$ by face identification.

However not all faces of these simplices are in $\mathcal{X}_{n}$.
To rectify this we replace each open simplex $\sigma(g,G)$ by a closed simplex $\overline{\sigma}(g,G)$ and take
the quotient as before. This new space, denoted by $\mathcal{X}^{*}_{n}$, is a simplical complex and called the \emph{simplical closure} of Outer space.
The points in $\mathcal{X^{*}_{n}}$ which are not in $\mathcal{X_{n}}$ are said to be at infinity.

 Now the group $\out(F_{n})$ acts on $\mathcal{X_{n}}$ by changing the marking in particular
 any $\varphi \in \out(F_{n})$ can be realized by a homotopy equivalence $f: R_{n} \to R_{n}$
 by mapping petals to each other according to their identification with generators of $F_{n}$.
 The group action by $\varphi$ on $(g,G)$ is then defined by
 \[
	 (g,G) \varphi = (g \circ f, G)
 .\] 

 Finally $\mathcal{X}_{n}$ contains an equivariant deformation retract $K_{n}$, the spine of Outer space.
 It is a subcomplex of the barycentric subdivision of the simplical closure $\mathcal{X}_{n}^{*}$, consisting of simplices spanned by vertices which are not at infinity.

 I.e. it is the grometric realisation of the partially ordered set of open simplices $\sigma(g,G)$ in $\mathcal{X}_{n}$, where 
 the partial order is given by the face relation.
 \todo{Understand and rephrase}

 We have the following vital result proved by Culler and Vogtmann in []\todo{cite}
 \begin{theorem}
	 $\mathcal{X}_{n}$ is contractible and the action of $\out(F_{n})$ is proper.
	 The spine $K_{n}$ is an equivariant deformation retract of dimension $2 n - 3$ with compact quotient
 \end{theorem}

\subsection{Forested graph Complex}
This construction mainly follow Bartholdi's simplified definiton from \cite{bartholdi16}.
Very useful in the general understanding of  what a graph complex is and how the boundary map acts was Bar-Natan's draft [].
Inspired by this similar examples for the forested graph complex are presented.
\todo{fix ordering and citation}

Let $F_{n}$ denote the free group of rank $n.$
\begin{definition}
	An \emph{admissible graph of rank $n$} is a $2$-edge-connected graph $G$ with fundamental group isomorphic to $F_{n}$ and with vertex-valency $\geq 3$.
\end{definition}


We often just write admissible graph for an admissible graph of rank $n$.

\begin{definition}
	An orientation $\sigma$ of a graph $G$ is an ordering of the edges i.e. $\sigma$ is an injective function from $E(G)$ to $\{1,\ldots,\abs{E(G)}\}$.
	We call the tuple $(G,\sigma)$ an oriented graph and note that $\on{Sym}$ acts on $(G,\sigma)$ by $\pi (G,\sigma) = (G,\pi \sigma)$ for $\pi \in \on{Sym}$.
\end{definition}

\begin{definition}
	A \emph{forested graph} is a triple $(G,\Phi,\sigma)$ where $G$ is an admissible graph $\Phi$ is a subset of edges that spans a forest on $G$ and 
	$\sigma$ is an orientation on $\Phi$.

	A map $f$ between two forested graphs $(G,\Phi, \sigma) \to (H,\Psi, \tau)$ is said to be a forested graph isomorphism if 
	$f$ is a graph isomorphism on $G$,  $f(\Phi) = \Psi$ and $\sigma = \tau \circ f $
\end{definition}

We know want to construct our graph complex. For this we remember the notion of a graded vector space:
\begin{definition}
	A graded vector space, is a vector space $V$ and with a decomposition $\left(V_{n}\right)^{\infty}_{n=0} $ such that
	\[
		V = \bigoplus_{k=0}^{\infty} V_{n}
	\] 
\end{definition}

We know consider the $\Q$-vector space $C$ spanned by isomorphism classes of forested graphs, subject to the relation
\[
	(G,\pi \Phi) = \sgn{\pi} \cdot (G,\Phi) \qq{for all} \pi \in \on{Sym}(k)
.\]
Observe that if $(G,\Phi) \simeq (G,\pi \Phi)$ for an odd permutation $\pi$ then $(G,\Phi) \simeq (G,\pi \Phi) = - (G,\Phi)$ and thus $(G,\Phi) = 0$ in  $C_{k}$.

We can define the following three gradings on $C$:
 \begin{itemize}
	\item Let $C^{n} \subseteq C$ be the subspace spanned by forested graphs of rank $n$. Then clearly  $C^{n} \cap C^{m} = \emptyset$ for $n \neq m$ and
		as every graph has a rank, we get that the $C^{n}$ form a grading of $C$.
	\item Let $C_{k} \subseteq C$ be the subspace spanned by forested graphs with a forest of size $k$. Clearly this also yields a decomposition of $C$ into a direct sum
		and thus yields another grading of $C$
	\item Let  $C_{d} \subseteq C$ be the subspace spanned by forested graphs of degree $d$. Once again this yields a grading on $C$
\end{itemize}
In the following we will mostly be concerned with the first two gradings. In particular we will consider the double-grading $C_{k}^{n}$,
where $k$ denotes the forest size and $n$ the rank. 

\begin{eg}
	Consider the graphs $G, J$ as in Example \ref{ex:gAuto}. Then $G$ and $J$ are admissible graphs of rank $3$ and $2$. Thus if we equip them with oriented forests $\Phi, \Psi$ as below 
	(where the red edges represent the forest and the numbers the orientation) we get forested graphs.

	\ctikzfig{./tikzit/forested_graph_automorphism}

	Observe, that $(J,\Psi) = 0$ in $C_{2}^{3}$, since $ (1 2)$ is an odd permutation and  $(1 2) (J,\Psi)$ is isomorphic to $(J,\Psi)$ 
	via the isomorphism mirroring along the vertical passing through the blue and purple vertex.

	$(G,\Phi)$ however is not trivial which can be seen as follows : 
	\todo[inline]{show that}
\end{eg}

Before we construct the chain complex we show that the $C^{n}$ are finitely generated and thus so are the $C_{k}^{n}$.

\begin{theorem}\label{thm:finGenCn}
	For all $n$, $C^{n}$ is finitely generated and for $C_{k}^{n} = 0 \forall k > 2n-3$. 
\end{theorem}

\begin{proof}
	We have the following general fact
	\[
		2 \abs{E} = \sum_{v \in V} \deg{v}
	.\] 
	Using that admissible graphs have vertex-valency $\geq 3$ and rearranging yields $\abs{E} \geq \frac{3}{2} \abs{V}$.
	From Proposition \ref{prop:rank} we get that $\abs{E} = \abs{V} + n-1$.
	Combining yields
	\[
		\abs{V} + n -1 \geq \frac{3}{2} \abs{V} \Leftrightarrow 2 (n-1) \geq \abs{V}
	\] 
	and plugging in the above in the identity from the Proposition yields $\abs{E} \leq 3 (n-1)$.

	With this we can get a loose upper bound on the amount of admissible graphs.
	As every graph can be written as incidence matrix and each entry is $\leq \abs{E}$ we get that there are maximally
	$\abs{E}^{\abs{V}^2}$ many different incidence matrices for graphs with  $\abs{V}$ vertices.
	As every graph corresponds to an incidence matrix this also gives an upper bound on the number of different graphs with $\abs{V}$ vertices.

	Thus the maximal possible number of admissible graphs of rank $n$ is bounded by
	\[
		\sum_{k=1}^{2(n-1)} (3 (n-1))^{k^2} 
	\] 
	As on each graph there also only exists a finite number of forests and each of them has a finite number of orientations we get that
	$C^{n}$ is finitely generated.

	That $C_{k}^{n} = 0 \forall k > 2n -3$ follows from the bound on the number of vertices and the fact that a forest in a graph has maximally $\abs{V} - 1$ edges.
\end{proof}

\begin{remark}
	Notice that the constraint of vertex-valency $\geq 3$ in the definition of admissible graphs is necessary for $C^{n}$ being finitely generated.
	As else we can consider the following family of graphs:
	\ctikzfig{./tikzit/infinite_admissible_graphs}
	They all have rank $n$ (which can be checked via the Euler characteristic), are $2$-edge-connected and not isomorphic as they have different number of vertices/edges.
\end{remark}

\begin{remark}
	The bound on the $C_{k}^{n}$ can not be improved as the graph $J$ from the example above with the tree extended by the edge between purple and green has rank $3$ and
	tree size $3$ which equals $2 \cdot 3 -3$.
\end{remark}

To construct our chain complex we fix the rank $n$ and define a differential as follows:
\begin{definition}
	Let $(G,\Phi,\sigma) = (G, \{e_1,\ldots,e_{k}\},\sigma)$ be a forested graph. Then let $\partial_{C}, \partial_{R}: C_{k}^{n} \to C_{k-1}^{n}$ be given by
	\begin{align*}
		\partial_{C}(G,\Phi) &= \sum_{i = 1}^{k} (-1)^{k} (G / e_{i}, \Phi \setminus \{e_{i}\}, \sigma_{e_{i}}),\\
		\partial _{R}(G,\Phi) &= \sum_{k = 1}^{k} (-1)^{k} (G,\Phi \setminus \{e_{i}\}, \sigma_{e_{i}}) 
	\end{align*}
	where $\sigma_{e_{i}}: \Phi \ \{e\} \to \{1,\ldots,k-1\}$ is given by
	\[
		\sigma_{e_{i}}(e) = \begin{cases}
			\sigma(e) & \text{ if }\sigma(e) < i\\
			\sigma(e) - 1 & \text{ if } \sigma(e) > i
		\end{cases}
	.\]
	Notice that the case $\sigma(e) = i$ can't happen as $e_{i}$ is not contained in $\Phi \setminus \{e_{i}\}$. 
	Finally define the boundary map $\partial = \partial_{C} - \partial_{R}$.
	%\todo[inline]{Is this renumbering correct or do we have to move $e_{i}$ first to the end of the orientation and then remove it inducing a sign of $-1^{i-1}$}
\end{definition}

\begin{proposition}
	$\partial$ is well-defined and $\partial^2 = 0$.
\end{proposition}

For better readability we will omit the orientation $\sigma$ in the proof.
\begin{proof}


	We proof the result in three steps:\\
	\textbf{Step 1:} Contracting an edge of a graph doesn't change the Euler characteristic as both the vertex number and the edge number decreases by one.
	Thus $\partial_{C}$ preserves the rank of the graph. Moreover the vertex-valency stays $\geq 3$ and the graph continues to be $2$-edge-connected.
	Hence is admissible.
	Moreover $\partial_{C}$ as well as $\partial_{R}$ remove one edge from each graph. Thus decreasing
	$k$ by $1$. Hence both maps are well-defined from $C^{n}_{k}$ to $C^{n}_{k-1}$ and thus so is $\partial$.

	Let $(G,\Phi) = (G, \{e_1,\ldots,e_{p}\})$ be a forested graph and denote the edges in $\Phi \setminus \{e_{i}\}$ by $\{e_1',\ldots,e_{p-1}'\}$.
	For the consecutive steps we need the following observations:
	\[
		(G / e_{i}) /  e_{j}' = \begin{cases}
			(G / e_{j}) / e_{i-1}' & \text{ if } i > j\\
			(G / e_{j+1}) / e_{i}' & \text{ if } i \leq j
		\end{cases}
		\qq{ as well as }
		(\Phi \setminus \{e_{i}\}) \setminus \{e_{j}'\}  = \begin{cases}	
			(\Phi \setminus \{e_{j}\}) \setminus \{e_{i-1}'\} & \text{ if } i > j\\
			(\Phi \setminus \{e_{j+i}\}) \setminus \{e_{i}'\} & \text{ if } i \leq j
		\end{cases}
	\]

	\textbf{Step 2:}\enskip
	\emph{Claim:}
		$\partial_{C}^2 = 0$ and $\partial_{R}^2 = 0$

	We compute:
	\begin{align*}
		\partial_{C}^2 &= \partial_{C} \sum_{i=1}^{p} (-1)^{i}(G / e_{i}, \Phi \setminus \{e_{i}\})
		=  \sum_{i=1}^{p} \sum_{j=1}^{p-1} (-1)^{i+j}((G / e_{i}) / e_{j}', (\Phi \setminus \{e_{i}\} ) \setminus \{e_{j}'\})  \\
					   &= \sum_{j < i} (-1)^{i+j} ((G / e_{i}) / e_{j}', (\Phi \setminus \{e_{i}\} ) \setminus \{e_{j}'\}) + \sum_{i \leq j} (-1)^{i+j}
					   ((G / e_{i}) / e_{j}', (\Phi \setminus \{e_{i}\} ) \setminus \{e_{j}'\}) 
	\end{align*}
	We claim that the right and left sum cancel. For this first apply the observations above to the left sum and then change variables by setting $l = j$ and  $m = i-1$ to obtain:
	\begin{align*}
		\sum_{j < i} (-1)^{i+j} ((G / e_{i}) / e_{j}', (\Phi \setminus \{e_{i}\}) \setminus \{e_{j}'\} ) &= 
		\sum_{j < i} (-1)^{i+j}((G / e_{j}) / e_{i-1}', (\Phi \setminus \{e_{j}\}) \setminus \{e_{i-1}'\} ) \\ 
		&= \sum_{l \leq m} (-1)^{l+m+1} ((G / e_{l}) / e_{m}', (\Phi \setminus \{e_{l}\}) \setminus \{e_{m}'\} ) 
	\end{align*}
	This last expression is the same as the left sum above but with opposite sign. Thus they cancel and we have shown $\partial_{C}^2 = 0$.
	The same argument shows that $\partial_{R}^2 = 0$.

	\textbf{Step 3:} \emph{Claim:} $\partial_{C} \partial_{R} - \partial_{R} \partial_{C} = 0$

	For the mixed terms we compute as follows
	\begin{align*}
		\partial_{R} \partial_{C} &=  \sum_{i=1}^{p} \sum_{j=1}^{p-1} (-1)^{i+j}(G / e_{i}, (\Phi \setminus \{e_{i}\} ) \setminus \{e_{j}'\})  \\
					   &= \sum_{j < i} (-1)^{i+j} (G / e_{i}, (\Phi \setminus \{e_{i}\} ) \setminus \{e_{j}'\}) + \sum_{i \leq j} (-1)^{i+j}
					   (G / e_{i}, (\Phi \setminus \{e_{i}\} ) \setminus \{e_{j}'\}) \tag{$*$}
	\end{align*}
	and
	\begin{align*}
		\partial_{C} \partial_{R} &=  \sum_{i=1}^{p} \sum_{j=1}^{p-1} (-1)^{i+j}(G / e_{j}', (\Phi \setminus \{e_{i}\} ) \setminus \{e_{j}'\})  \\
					   &= \sum_{j < i} (-1)^{i+j} (G / e_{j}', (\Phi \setminus \{e_{i}\} ) \setminus \{e_{j}'\}) + \sum_{i \leq j} (-1)^{i+j}
					   (G / e_{j}', (\Phi \setminus \{e_{i}\} ) \setminus \{e_{j}'\}) \\
					   &\stackrel{(\heartsuit)}{=} \sum_{j < i} (-1)^{i+j} (G / e_{j}, (\Phi \setminus \{e_{j}\} ) \setminus \{e_{i-1}'\}) + \sum_{i \leq j} (-1)^{i+j}
					   (G / e_{j+1}, (\Phi \setminus \{e_{j+1}\} ) \setminus \{e_{i}'\}) \\
					   &\stackrel{(\dagger)}{=} \sum_{l \leq m} (-1)^{m+l+1} (G / e_{l}, (\Phi \setminus \{e_{l}\} ) \setminus \{e_{m}'\}) + \sum_{k < n} (-1)^{n+k-1}
					   (G / e_{k}, (\Phi \setminus \{e_{k}\} ) \setminus \{e_{n}'\}) \tag{$* *$}
	\end{align*}
	Where in $(\heartsuit)$ we used that if $j < i$ then $e_{j} = e_{j}'$ and if $i \leq j$ then $e_{j+1} = e_{j}'$, as well as the observation above.
	In $(\dagger)$ we used the substitution  $m = i-1$,  $l = j$ on the left and  $m = i$,  $k = j+1$ on the right sum.
	Comparing the sums in $(*)$ and $(* *)$ we see that they differ by a sign and thus cancel. Hence  $\partial_{C} \partial_{R} - \partial_{R} \partial_{C} = 0$.

	Combining Step 2 and 3 we get:
	\[
		\partial^2 = (\partial_{C} - \partial_{R})^2 = \partial_{C}^2 - (\partial_{C} \partial_{R} + \partial_{R} \partial_{C}) + \partial_{R}^2 = 0
	\]
\end{proof}

Thus the spaces $(C_{\bullet})$ with the differential $\partial_{\bullet}$ form a chain complex.

\begin{eg}
	Once again we consider the graph $(G,\Phi)$ from above and calculate its boundary operator:
	\ctikzfig{./tikzit/boundary_operator}
	Where we used that $-H_{1}$ is equal to $H_{2}$ by mirroring along the vertical and applying $(2 3)$,
	$-H_{3}$ is equal to $H_2$ by exchanging inner and outer vertices, mirroring along the vertical and applying $(1 3)$ and
	$H_{4}$ is equal to $H_{2}$ by exchanging inner and outer vertices and applying $(1 3)(2 3)$.
	Where we have used the permutation $(2 3)$ on the first,  $(1 3)$ on the third, $(1 3)(2 3)$ on the fourth graph to get the result.
	\ctikzfig{./tikzit/boundary_operator2}
	Where we have used that $-G_{1}$ is equal to $-G_{3}$ by exchanging inner and outer vertices and applying $(1 3)(1 2)$,
	$G_{2}$ is equal to $-G_{3}$ by exchanging, mirroring along the vertical and applying $(1 3)$ and
	$G_{4}$ is equal to $-G_{3}$ by mirroring along the vertical and applying $(1 2)$. 

	Thus we can conclude that $\partial_{\bullet} G = 4 H_2 - 4 G_3$. Moreover we have that $H_2 - G_3 \in \im \partial_{\bullet}$ and 
	as $\partial_{\bullet}^2=0$ also $H_2 - G_3 \in \ker \partial_{\bullet}$
\end{eg}

\subsection{Cubical Chain Complex}
The above constructed complex $C_{\bullet}$ can also be viewed as a cubical chain complex.
Here we think of a graph $(G,\Phi) = (G, \{e_1,\ldots,e_{k}\})  \in C_{k}$ as the $k$-dimensional $[0,1]$-cube embedded in $R^{k}$.
The Graph $G_{\Phi}$ where all edges in  $\Phi$ have been collapsed sits at the origin and the graph $G$ where all edges have been removed from  $\Phi$ but not collapsed
sits diagonally opposite at  $(1,\ldots,1)$.
We can assign a graph to every face in the following way:
Consider a face $F$ of dimension $n < k$, let $bF$ be its barycenter. Then $bF \in \{0,1 / 2, 1\}^{k}$ and denote by $bF_{i}$ its $i$-th coordinate.
Moreover let
\[
	C := \{e_{i} \in \Phi \mid bF_{i} = 0\} \qq{and} R := \{e_{i} \in \Phi \mid bF_{i} = 1\}  
.\] 
Then the graph associated to $F$ is given by $(G / C, \Phi \setminus (C \cup D)$.
Thus an edge gets contracted if $bF_{i} = 0$ and an edge gets removed from $\Phi$ but not contracted if $bF_{i} = 1$.
Now if a face is of dimension $n$ then $n$ coordinates of $bF$ equal  $1 / 2$ and thus the resulting graph has a forest of size $n$ and is hence in $C_{n}$.

The above description gives us a bijection between the reduced graphs of $G$ and the half integral points.
Denote the reduced graph of $G$ associated to $bF$ by $G_{bF}$. Then we can define the boundary operator via this bijection as follows:

\todo[inline]{Formulate this}

To visualize this construction we consider the following example:
\begin{eg}
	Consider again the graph $J$ from example \ref{ex:gAuto} with the forest $\Phi$ given by an edge between the top and middle vertex and
	between the left and write vertex.
	Then its $2$-dimensional cube is given as below:
	\ctikzfig{./tikzit/cubicalCCEx}
\end{eg}

\section{Morita-Cycles}
\todo[inline]{Formulate this whole section better and in an easier way}
The goal of this section is to show that there exist a cycle in ever $C_{n}$.
For this we define the Morita-Cycle graphs and show that there exists a chain of those graphs that vanishes under the boundary $\partial$.

 \begin{definition}
	 A general \emph{Morita-Cycle} $M_{n}$, for $n \geq 3$ odd, is a forested graph $(G,\Phi,\tau)$ defined as follows:
	 $G$ has $2\cdot n$ vertices which form two separate cycles each of order $n$.
	 More over each vertex of one cycle is connected via an edge to exactly one vertex of the other cycle.
	 The forest $\Phi$ are $n-1$ edges in each of the two cycles.
	 The orientation is some numbering of the edges in $\Phi$. 

	 We call one of the cycles the "left" and the other one the "right" cycle.
	 Moreover the missing edge in the cycle from the left/right tree in $\Phi$ is called missing left/right edge.
\end{definition}
As the cycles have $n$ edges each. The forest $\Phi$ has exactly two trees of size $n-1$ of the form of a line.
For every even $n$ it can easily be seen that $M_{n}$ has an odd automorphism and thus vanishes.

\begin{definition}
	We say a Morita-Cycle $M_{n}$ is (drawn) in standard form if the edges in the forest in the left cycle are numbered in ascending order
	starting with $1$ on one edge adjacent to the missing left edge and ending with $n-1$ on the other edge adjacent to the missing left.
	For the right cycle the edges in $\Phi$ are numbered analogous just from $n$ to $2n-1$. 
	Moreover when we draw the graph we always draw the missing left/right to the outside of the cycles.

	To make this definition more intuitive it is best to view the illustration below of a Morita cycle $M_{5}$ in standard form.
\end{definition}

\begin{proposition}
	Every Morita-Cycle $M_{n}$ in standard form is fully determined by $n$ and a permutation $\sigma \in S_{n}$. Thus we denote 
	such a Graph by $M_{n}(\sigma)$.
	Moreover every general Morita-Cycle $M_{n}$ can be written in standard form.
\end{proposition}

\begin{proof}
	We show the second part of the statement first.
	By renumbering the edges in the forest we can always get to the desired ordering. Notice that this might introduce a factor of $-1$
	if the renumbering is an odd permutation.
	Now by rotating the left/right cycle in our drawing we can also get the missing left/right in its desired place.

	Now to the first part of the proposition.
	By the properties of the standard form the left and right cycle are fully fixed and equal for all Morita-Cycles of size $2 n$.
	Thus the only difference are the edges between the two cycles.
	For these we number the nodes in the left cycle from $1$ to $n$ from top to bottom
	and on the right from $n$ to  $2 n$ also from top to bottom. Then each edge can be identified with a pair $(u,v)$.
	If we now define the permutation $\sigma: {1,\ldots,n} \to {1,n}$ via $\sigma(u) = v - n$.

	On the other side if we have a permutation $\sigma \in S_{n}$. We can define the edges between the left and right cycle
	by $(i,\sigma(i) +n)$ for $i \in {1,\ldots,n}$.
\end{proof}

\begin{eg}
	Below is a Morita-Cycle of order $5$ in standard form defined by the permutation $(12)(345)$.
	\ctikzfig{./tikzit/morita_cycle_5}
\end{eg}

\begin{theorem}
	For all $n \in \N_{\geq 3}$ odd it holds that 
	\[
		\partial\left(\sum_{\sigma \in S_{n}} \sgn(\sigma) M_{n}(\sigma)\right) = 0
	.\] 
\end{theorem}

We prove this statement in two parts first for $\partial_{R}$ and then for $\partial_{C}$, from which the final result follows.
Moreover when we talk about a Morita-Cycle we will always talk about it drawn in its standard form.
\begin{proof}[Proof for $\delta_C$]
	Let $(H,\Psi,\eta)$ be in the sum $\partial_{C}\left(\sum_{\sigma \in S_{n}} \sgn(\sigma) M_{n}(\sigma)\right)$.
	Then as it is an element of the boundary of some Morita-Cycle it has to
	have one vertex of degree 4, in either the left or right cycle and the same cycle has order $n-1$ and only $n-2$ edges in the tree.
	W.l.o.g. we assume that the edge is missing in the left tree as else one can mirror along the horizontal
	and apply the permutation $(1 \ldots 2n-1)^{n}$ to the orientation. An example of $H$ is given in the figure below on the left.

	Now assume that the vertex of degree $4$ is the $k$-th one from the top. An example is shown in the figure below on the left for $k = 2$.
	Then there exist exactly two Morita-Cycles in standard form in whose boundary $H$ lies. 
	One is obtained by splitting the vertex $k$ into two where each new vertex contains one edge from the forest and one connecting to the right cycle.
	Moreover the two new nodes get connected by an edge which is also added to the forest and given the number $k$ in the ordering.
	All other edges with ordering number  $> k $ get increased by one. Lets denote this Morita-Cycle by $M_{n}(\sigma)$.
	The other one is obtained in the same way however the two edges connecting to the right cycle are permuted i.e.
	if the new node is numbered by $k+1$ then this Morita-Cycle equals $M_{n}((k\ k+1)\sigma)$.
	Examples for both are also shown in the figure below in the middle and right respectively.

	Thus we see that the two permutations $\sigma$ and $(k\ k+1) \sigma$ have opposite parity and thus $M_{n}(\sigma)$ and $M_{n}((k\ k+1) \sigma)$ have
	opposite sign in the sum from the theorem. Thus the elements in their boundary corresponding to $H$ also have opposite sign and hence cancel.

	As $H$ was arbitrary an arbitrary element of the sum we get that every summand has coefficient $0$ and the sum vanishes.
	\ctikzfig{./tikzit/morita_cycle_dC}
\end{proof}

\begin{proof}[Proof for $\partial_{R}$]
	Again let $(H,\Psi,\eta)$ be in the sum $\partial_{R}\left(\sum_{\sigma \in S_{n}} \sgn(\sigma) M_{n}(\sigma)\right)$.
	Let $\sigma$ be the edge assignment between the left and right cycle as described in the standard form.
	Then $H$ is a Morita-Cycle in standard form missing one edge in either the left or right tree.
	W.l.o.g. we assume that the edge is missing in the left tree as else one can mirror along the horizontal
	and apply the permutation $(1 \ldots 2n-1)^{n}$ to the orientation.

	Now assume the $k$-th edge from the top is missing an example is shown in the figure below on the left for $k = 2$.
	Then there are exactly two Morita-Cycles in standard form which have $H$ in their boundary.
	One where the $k$-th edge has been added to the forest with number $k$ in the ordering i.e. $M_{n}(\sigma)$.
	And the other where the most left edge has been added, the numbering in the left cycle been changed such that $1$ is the first edge 
	after the missing one, the left most edge has order $n-k$ and so on.
	Notice that the second graph is not drawn in standard form. To bring it in standard form one has to "rotate" the left cycle
	by $k$ nodes counter clockwise. Each of this rotation induces the permutation $(1 \ldots n)$ on the edges i.e.
	the resulting Morita cycle is given by $M_{n}(\tau) := M_{n}(\sigma (1 \ldots n)^{k})$.
	Examples for both are also shown in the figure below in the middle and right respectively.
	
	As $(1 \ldots n)$ for $n$ odd has even parity $\tau$ and $\sigma$ have the same parity and thus $M_{n}(\tau)$ and $M_{n}(\sigma)$ have the same sign
	in the sum from the theorem. 
	Finally taking the boundary the elements corresponding to $H$ have different orientations.
	The element in $\partial_{R}(M_{n}(\sigma))$ is exactly $H$ with sign $(-1)^{k}$. The element in $\partial_{R}(M_{n}(\tau))$ however 
	differs from $H$ by the permutation $(1 \ldots n-2)^{k - 1}$ with even parity as $n-2$ is odd and has sign $(-1)^{n-k}$.
	Now for $n$ odd if $k$ is even $n-k$ is odd and vice versa. Thus the elements corresponding to $H$ have opposite sign and cancel.

	As $H$ was an arbitrary element in the sum we get that every summand has coefficient $0$ and the sum vanishes.

	\ctikzfig{./tikzit/morita_cycle_dR}

	Thus we have shown the result for both $\partial_{C}$ and $\partial_{R}$ and as $\partial = \partial_{C} - \partial_{R}$ it also follows for $\partial$.
\end{proof}

\newpage
\printbibliography

%\input{../header}

\section{Draftparts}

\begin{definition}
	A \emph{graph} $G$ is a finite $1$-dimensional CW complex. The set of edges is denoted by $E(G)$, the set of vertices by  $V(G)$ and the set of half edges by  $H(G)$.
	We call a graph \emph{connected} if the CW complex is connected in the topological sense.
	A graph is said to be \emph{$n$-valent} if every vertex has valency $n$ i.e. for every vertex the number of edges incident is  $n$.

	Lastly a \emph{tree} is a graph which contains no loops and a \emph{forest} is a collection of disjoint trees.
\end{definition}

On trivalent connected graphs we call an \emph{orientation} a choice of cyclic orders of all vertices up to an even number of changes.

\begin{definition}
	A \emph{forested graph} is a pair $(G, \Phi)$, where $G$ is a finite connected trivalent graph and $\Phi$ is an oriented forest which contains all vertices of $G$. 
\end{definition}

\begin{definition}
Let $(G,\Phi)$ be a forested graph and let  $e \in \Phi$. Moreover let $(G_{e},\Phi_{e})$ be the graph where $e$ has been collapsed.
Then there exist exactly two other graphs whose edge collapse results in $(G_{e},\Phi_{e})$. This is visualised in the figure below.
Where  $1,2,3,4$ represent the rest of the graph.

\ctikzfig{./tikzit/IHXRelator}
Now the vector 
\[
	(G,\Phi) + (G',\Phi') + (G'',\Phi'')
\]
is called the \emph{basic IHX relator} associated to $(G,\Phi,e)$.
\end{definition}

Denote by $\widehat{\fg}_{k}$ the vector space spanned by all forested graphs containing $k$ trees modulo the relations $(G,\Phi) = -(G,-\Phi)$.
Moreover let $\fg_{k}$ be the quotient of $\widehat{\fg}_{k}$ modulo the subspace spanned by all basic IHX relators.


\begin{definition}
	Let $\widehat{\partial}_E(G,\Phi): \widehat{\fg}_{k} \to \widehat{\fg}_{k-1}$ be given by
\begin{align*}
	\widehat{\partial}_{E}(G,\Phi) = \sum (G, \Phi \cup e)
.\end{align*}
where the sum is over all edges $e$ of $G \setminus \Phi$ such that $\Phi \cup e$ is still a forest.
Notice that this only happens if the two vertices of  $e$ lie in different trees of $\Phi$. Thus $\Phi \cup e$ has $k-1$ components.
The orientation of $\Phi \cup e$ is determined by ordering the edges of $\Phi$ with labels  $1,\ldots,k$ consistent with its orientation
and then labeling the new edge $e$ with  $k+1$.

Now let the boundary map $\partial_{E}: \fg_{k} \to \fg_{k-1}$ be the map induced by $p \circ \widehat{\partial}_{E}$ where $p$ is the quotient map $\widehat{\fg_{k}} \to \fg_{k}$.
\end{definition}

\begin{proposition}
	$\partial_{E}$ is well-defined and $\partial_{E}^2 = 0$.
\end{proposition}

\begin{proof}
	
\end{proof}

The \emph{forested graph complex} is thus defined as the sequence $\fg_{k}$ with boundary map $\partial_{E}$ and is well-defined by the above proposition.

\subsection{The Outer space}
We closely follow Vogtmann's definition from \cite[p. 2 ff.]{vogtmann16}
\begin{definition}
	By a \emph{metric graph} we mean a finite connected graph with positive real edge lengths, equipped with the path metric.
	We fix a model rose $R_{n}$ (a graph with one vertex and $n$ petals), and identify the petals of $R_{n}$ with the generators
	of the free group $F_{n}$. A point in $\mathcal{X}_{n}$ is then a metric graph $G$ together with a homotopy
	equivalence $g: R_{n} \to  G$ called a \emph{marking}; the marking serves to identify the fundamental group of $G$ with $F_{n}$.
	Marked graphs $(g,G)$ and $(g',G')$ are considered the same if there is an isometry  $f: G \to G'$ with $f \circ g$ homotopic to $g'$.

	To get a finite dimensional space we assume $G$ has no uni- and bivalent vertices (see Theorem \ref{thm:finGenCn}).
	Moreover we normalize our objects i.e. we assume the sum of edge lengths to be $1$ and assume that $G$ is $2$-connected.
\end{definition}
To make  $\mathcal{X}_{n}$ a space we need to define a topology. We proceed as follows: 
For every marked graph $(g,G)$ we define the open simplex $\sigma(g,G)$ as the set obtained by varying the edge lengths of $G$,
keeping their sum equal to $1$.
The simplex  $\sigma(g',G')$ is then a \emph{face} of $\sigma(g,G)$ if $(g',G')$ can be obtained from $(g,G)$ by collapsing some edges to points.

Finally $\mathcal{X}_{n}$ is the quotient space obtained from the disjoint union of the open simplices $\sigma(g,G)$ by face identification.

However not all faces of these simplices are in $\mathcal{X}_{n}$.
To rectify this we replace each open simplex $\sigma(g,G)$ by a closed simplex $\overline{\sigma}(g,G)$ and take
the quotient as before. This new space, denoted by $\mathcal{X}^{*}_{n}$, is a simplical complex and called the \emph{simplical closure} of Outer space.
The points in $\mathcal{X}^{*}_{n}$ which are not in $\mathcal{X}_{n}$ are said to be at infinity.

 Now the group $\out(F_{n})$ acts on $\mathcal{X}_{n}$ by changing the marking in particular
 any $\varphi \in \out(F_{n})$ can be realized by a homotopy equivalence $f: R_{n} \to R_{n}$
 by mapping petals to each other according to their identification with generators of $F_{n}$.
 The group action by $\varphi$ on $(g,G)$ is then defined by
 \[
	 (g,G) \varphi = (g \circ f, G)
 .\] 

 Finally $\mathcal{X}_{n}$ contains an equivariant deformation retract $K_{n}$, the spine of Outer space.
 It is a subcomplex of the barycentric subdivision of the simplical closure $\mathcal{X}_{n}^{*}$, consisting of simplices spanned by vertices which are not at infinity.

 In other language, $K_{n}$ is the geometric realisation of the partially ordered set of open simplices $\sigma(g,G)$ in $\mathcal{X}_{n}$, where 
 the partial order is given by the face relation.

 We have the following vital result providing the relationship between $\mathcal{X}_{n}$ and $\out(F_{n})$. This was proved by Culler and Vogtmann in \cite{vogtmann86}.
 \begin{theorem}
	 $\mathcal{X}_{n}$ is contractible and the action of $\out(F_{n})$ is proper.
	 The spine $K_{n}$ is an equivariant deformation retract of dimension $2 n - 3$ with compact quotient
 \end{theorem}
 
 \section{Introduction\footnotemark}
\footnotetext{This section follows closely Vogtmann's survey article \cite{vogtmann16}}
Free groups are one of the most basic examples of infinite finitely-generated groups.
The automorphism groups of free groups are of particular interest since they are tied to many other areas of mathematics.
The automorphism group $\aut(G)$ of a group $G$ can be separated into the inner automorphism group $\on{Inn}(G)$,
the group of automorphisms that arise from conjugation, and the outer automorphism group $\out(G)$, the quotient of  $\aut(G)$ by $\on{Inn}(G)$.

Most often the inner automorphism groups are well understood.  To understand the outer automorphism groups
the standard approach has been to study it via its action on some topological space.
For the outer automorphism group of the free group on $n$ generators $\out(F_{n})$ the space $\mathcal{X}_{n}$ known as "Outer space"
has been introduced by Culler and Vogtmann in \cite{vogtmann86}. As the points of the Outer space correspond to finite graphs with fundamental group $F_{n}$,
the structure of Outer space is connected to such diverse areas as the study of Lie algebras of derivations, degenerations of algebraic varieties,
the computation of Feynman integrals, and the statistics of phylogenetic trees.

A central concept of the Outer space is its spine $K_{n}$ which is an equivariant deformation retract of $\mathcal{X}_{n}$.
By being able to compute the rational homology of $K_{n} / \out(F_{n})$ one can also compute $\out(F_{n})$.
By identifying the spine $K_{n}$ with a cube complex one arrives at the forested graph complex
which has been introduced by Conant and Vogtmann in \cite{conant03}.

The forested graph complex is also the main focus of this paper.
We will begin by defining basic concepts of graphs, the Outer space and the forested graph complex as well as show the connection via the cube complex.

Afterwards we will focus on Morita's classes which are an infinite sequence of cocycles representing potentially nontrivial cohomology classes $\mu_{k} \in H^{4k}(\out(F_{2k+2})$.






\end{document}
