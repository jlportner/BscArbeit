%\input{../header}
%%% DOCUMENT TYPE %%%%%%%%%%%%%%%%%%%%%%%%%%%%%%%%%%%%%%%%%%%%%%%%%%%%%%%%%%%%%%

\documentclass[11pt, a4paper]{article}

%%% PACKAGES %%%%%%%%%%%%%%%%%%%%%%%%%%%%%%%%%%%%%%%%%%%%%%%%%%%%%%%%%%%%%%%%%%%

% Encoding

\usepackage[utf8]{inputenc}
\usepackage[T1]{fontenc}
\usepackage{lmodern}

% Geometry

\usepackage{geometry} % edit margins of paper
\usepackage{setspace} % edit line spacing
\usepackage{fancyhdr} % header, footer
\usepackage{titlesec} % edit format of titles

% Images
\usepackage[pdftex]{graphicx} % image locations

% Visual

\usepackage[dvipsnames]{xcolor} % colors
\usepackage{pgf,tikz} % graphics
\usetikzlibrary{quotes}
\usetikzlibrary{cd}
\usetikzlibrary{babel}

\usepackage[framemethod=tikz]{mdframed} % frames, better theorems

% Math

\usepackage{amsmath} % math tools
\usepackage{amssymb} % math symbols
\usepackage{amsthm} % thereoms
\usepackage{mathtools} % math tools
\usepackage{xfrac} %More Fractions

% Referencing

\usepackage{nameref}
\usepackage{hyperref}
\usepackage{cleveref}

% Useful

\usepackage[shortlabels]{enumitem} % enumerations

% Other

\usepackage{lastpage} % get number of last page
\usepackage{physics}
%\usepackage{bbm}
\usepackage[makeroom]{cancel} %crossing stuff out
\usepackage{stmaryrd}
\usepackage{tabu}

%%% MARGINS %%%%%%%%%%%%%%%%%%%%%%%%%%%%%%%%%%%%%%%%%%%%%%%%%%%%%%%%%%%%%%%%%%%%

\geometry{a4paper, left=25mm, right=25mm, top=10mm, bottom=20mm, includehead}

%%% COLORS %%%%%%%%%%%%%%%%%%%%%%%%%%%%%%%%%%%%%%%%%%%%%%%%%%%%%%%%%%%%%%%%%%%%%

%%% TITLES %%%%%%%%%%%%%%%%%%%%%%%%%%%%%%%%%%%%%%%%%%%%%%%%%%%%%%%%%%%%%%%%%%%%%

\colorlet{color-section}                {BrickRed}
\colorlet{color-subsection}             {BrickRed}

%%% MATH BOXES %%%%%%%%%%%%%%%%%%%%%%%%%%%%%%%%%%%%%%%%%%%%%%%%%%%%%%%%%%%%%%%%%

\colorlet{color-definition}             {SpringGreen!20}
\colorlet{color-theorem}                {Apricot!13}
\colorlet{color-proposition}            {Apricot!13}
\colorlet{color-corollary}              {Apricot!13}
\colorlet{color-lemma}                  {Apricot!13}
\colorlet{color-attention}              {OrangeRed!13}
\colorlet{color-remark}                 {Gray!5}
\colorlet{color-example}                {Lavender!7}
\colorlet{color-notation}               {Gray!5}
\colorlet{color-convention}             {Gray!5}
\colorlet{color-claim}                  {SkyBlue!7}
% \colorlet{color-proof}                  {FILL COLOR HERE}


%%% CAPTIONS %%%%%%%%%%%%%%%%%%%%%%%%%%%%%%%%%%%%%%%%%%%%%%%%%%%%%%%%%%%%%%%%%%%

%%% CAPTION DEFINITION %%%%%%%%%%%%%%%%%%%%%%%%%%%%%%%%%%%%%%%%%%%%%%%%%%%%%%%%%

\newcommand*{\definitionname}{Definition}
\newcommand*{\theoremname}{Theorem}
\newcommand*{\propositionname}{Proposition}
\newcommand*{\corollaryname}{Corollary}
\newcommand*{\lemmaname}{Lemma}
\newcommand*{\remarkname}{Remark}
\newcommand*{\examplename}{Example}
\newcommand*{\attentionname}{Attention}
\newcommand*{\notationname}{Notation}
\newcommand*{\conventionname}{CConvention}
\newcommand*{\claimname}{Claim}


%%% SHORTCUTS %%%%%%%%%%%%%%%%%%%%%%%%%%%%%%%%%%%%%%%%%%%%%%%%%%%%%%%%%%%%%%%%%%

%%% SINGLE SYMBOLS %%%%%%%%%%%%%%%%%%%%%%%%%%%%%%%%%%%%%%%%%%%%%%%%%%%%%%%%%%%%

% Logic

% \forall exists
% \exists exists
% \lnot exists
% \lor exists
% \land exists
\newcommand*{\limp}{\rightarrow}
\newcommand*{\limps}{\; \limp \;} % \limp with some space around
\newcommand*{\leqv}{\leftrightarrow}
\newcommand*{\leqvs}{\; \leqvs \;} % \leqv with some space around
\newcommand*{\lTri}{\vartriangleleft}
\newcommand*{\on}[1]{\operatorname{#1}}
\newcommand*{\fg}{f\mathcal{G}}
% Meta Logic

% \implies exists
% \iff exists

% Colon Stuff

\newcommand*{\cl}{\colon}
\newcommand*{\cleq}{\coloneqq}
\newcommand*{\eqcl}{\eqqcolon}

% Sets

\newcommand*{\N}{\mathbb{N}} % natural numbers
\newcommand*{\Z}{\mathbb{Z}} % integers
\newcommand*{\Q}{\mathbb{Q}} % rational numbers
\newcommand*{\R}{\mathbb{R}} % real numbers
\newcommand*{\C}{\mathbb{C}} % complex numbers
\newcommand*{\F}{\mathbb{F}} % finite field

%%% MATH OPERATORS %%%%%%%%%%%%%%%%%%%%%%%%%%%%%%%%%%%%%%%%%%%%%%%%%%%%%%%%%%%%%

% General

\DeclareMathOperator{\id}{id}
\DeclareMathOperator{\sgn}{sgn}
%Analysis
\DeclareMathOperator{\vol}{vol}
\DeclareMathOperator{\supp}{supp}
\let\grad\relax
\DeclareMathOperator{\grad}{grad}
\DeclareMathOperator{\rot}{rot}
\let\div\relax
\DeclareMathOperator{\div}{div}
%LinAlg
\let\ker\relax
\DeclareMathOperator{\ker}{Ker}
\DeclareMathOperator{\eig}{Eig}
\DeclareMathOperator{\im}{Im}
\let\hom\relax
\DeclareMathOperator{\hom}{Hom}
\DeclareMathOperator{\End}{End}
\DeclareMathOperator{\mat}{Mat}
%Algebra
\let\gcd\relax
\DeclareMathOperator{\gcd}{ggT}
\let\ev\relax
\DeclareMathOperator{\ev}{ev}
\DeclareMathOperator{\charak}{char}
\DeclareMathOperator{\quot}{Quot}
\DeclareMathOperator{\sym}{Sym}
\DeclareMathOperator{\bij}{Bij}
\DeclareMathOperator{\GL}{GL}
\DeclareMathOperator{\SL}{SL}
\DeclareMathOperator{\gal}{Gal}
\DeclareMathOperator{\aut}{Aut}
\DeclareMathOperator{\out}{Out}
\DeclareMathOperator{\SO}{SO}
\DeclareMathOperator{\irr}{irr}
%%% TEMPLATES %%%%%%%%%%%%%%%%%%%%%%%%%%%%%%%%%%%%%%%%%%%%%%%%%%%%%%%%%%%%%%%%%%

% General

% write a set definition like: { #1 | #2 }
\newcommand*{\sdef}[2]{
  \{#1 \mid #2\}
}

% write a nice map definition
\newcommand*{\mdef}[5]{
  \begin{align*}
    #1 \cl #2 &\to     #3 \\
           #4 &\mapsto #5
  \end{align*}
}

\newcommand*{\nstack}[2]{
	\begin{array}{c}
		#1\\
		#2\\
	\end{array}
}


%%% FORMATTING %%%%%%%%%%%%%%%%%%%%%%%%%%%%%%%%%%%%%%%%%%%%%%%%%%%%%%%%%%%%%%%%%

%%% HEADER, FOOTER %%%%%%%%%%%%%%%%%%%%%%%%%%%%%%%%%%%%%%%%%%%%%%%%%%%%%%%%%%%%%

\pagestyle{fancy}
\renewcommand{\headrule}{}
%\renewcommand{\chaptermark}[1]{\markboth{#1}{}}
\fancyhf{} % clear everything
%\fancyhead[L]{\rightmark}
%\chead{\bfseries Zusammenfassung Algebra I/II}
%\rhead{Seite \thepage /\pageref*{LastPage}}
%\lfoot{}
\fancyfoot[C]{\thepage}
%\fancyfoot[R]{\thepage}

%%% TITLE FORMAT %%%%%%%%%%%%%%%%%%%%%%%%%%%%%%%%%%%%%%%%%%%%%%%%%%%%%%%%%%%%%%%

\setcounter{secnumdepth}{2}

%\titleformat{\chapter}[hang]
%{\normalfont\huge\bfseries}{\chaptertitlename\ \thechapter:}{20pt}{\Huge}
\titleformat{\section}
{\normalfont\LARGE\bfseries}{\thesection}{1em}{}
\titleformat{\subsection}
{\normalfont\large\bfseries}{\thesubsection}{1em}{}
\titleformat{\subsubsection}
{\normalfont\normalsize\bfseries}{\thesubsubsection}{1em}{}
\titleformat{\paragraph}[runin]
{\normalfont\normalsize\bfseries}{\theparagraph}{1em}{}
\titleformat{\subparagraph}[runin]
{\normalfont\normalsize\bfseries}{\thesubparagraph}{1em}{}

%%% SPACING %%%%%%%%%%%%%%%%%%%%%%%%%%%%%%%%%%%%%%%%%%%%%%%%%%%%%%%%%%%%%%

% Titles

%\titlespacing*{\chapter}{0pt}{0pt}{15pt}
\titlespacing*{\section}{0pt}{3.5ex plus 1ex minus .2ex}{2.3ex plus .2ex}
\titlespacing*{\subsection}{0pt}{3.25ex plus 1ex minus .2ex}{1.5ex plus .2ex}
\titlespacing*{\subsubsection}{0pt}{3.25ex plus 1ex minus .2ex}{1.5ex plus .2ex}
\titlespacing*{\paragraph}{0pt}{1.25ex plus 1ex minus .2ex}{1em}
\titlespacing*{\subparagraph}{\parindent}{3.25ex plus 1ex minus .2ex}{1em}

% Text, Paragraphs

\setstretch{1.05} % scaling of space between lines
\setlength{\parindent}{0pt} % indentation of paragraphs
\setlength{\parskip}{4.0pt plus 1.0pt minus 1.0pt} % space between paragraphs
%\setlength{\parskip}{0pt}

%%% SYMBOLS USED BY NUMBERINGS, ENVIRONMENTS, ... %%%%%%%%%%%%%%%%%%%%%%%%%%%%%%

% \renewcommand*\qedsymbol{$\blacksquare$} % alternative QED symbol
%\renewcommand{\thefootnote}{\arabic{footnote}} % normal footnotes on page
%\renewcommand{\thempfootnote}{\fnsymbol{mpfootnote}} % footnotes on minipages, e.g. in mdframed environments

%%% LISTS, ENUMERATIONS %%%%%%%%%%%%%%%%%%%%%%%%%%%%%%%%%%%%%%%%%%%%%%%%%%%%%%%%

% 'itemize'

\setlist[itemize]{noitemsep, topsep=0pt}

% 'enumerate'

\setlist[enumerate]{noitemsep, topsep=0pt}
% no special settings at the moment

% 'description'

% no special settings at the moment

% 'axioms'

%\newlist{axioms}{enumerate}{2}
%\setlist[axioms]{itemsep=0pt,label*=\arabic*.}

%%% GENERAL SYMBOLS %%%%%%%%%%%%%%%%%%%%%%%%%%%%%%%%%%%%%%%%%%%%%%%%%%%%%%%%%%%%
\newcommand\danger{\raisebox{\depth}{{\fontencoding{U}\fontfamily{futs}\selectfont\char 66\relax}}}
\newcommand\contra{\scalebox{1.5}{$\lightning$}}

%%% MDFRAMED PATCH %%%%%%%%%%%%%%%%%%%%%%%%%%%%%%%%%%%%%%%%%%%%%%%%%%%%%%%%%%%%%

\usepackage{xpatch}

\makeatletter
\xpatchcmd{\endmdframed}
  {\aftergroup\endmdf@trivlist\color@endgroup}
  {\endmdf@trivlist\color@endgroup\@doendpe}
  {}{}
\makeatother

%%% MDFRAMED STYLES %%%%%%%%%%%%%%%%%%%%%%%%%%%%%%%%%%%%%%%%%%%%%%%%%%%%%%%%%%%%

% thick frame and bar for title

%\mdfdefinestyle{style-box}{
%  skipabove=1.5ex plus .5ex minus .2ex,
%  skipbelow=1ex plus .2ex minus .2ex,
%  linewidth=2pt,
%  linecolor=Gray!20,
%   roundcorner=3pt,
%  innerleftmargin=0.5\baselineskip,
%  innerrightmargin=0.5\baselineskip,
%  innertopmargin=0.4\baselineskip,
%  innerbottommargin=0.4\baselineskip,
%  frametitlebackgroundcolor=Gray!20,
%  frametitleaboveskip=0.3pt,
%  frametitlebelowskip=0.3pt,
%  theoremseparator=,
%  theoremspace=\hfill,
%  theoremtitlefont=\mdseries\scshape,
%  nobreak=true
%}

% highlighted background

%\mdfdefinestyle{style-background}{
%  skipabove=1.5ex plus .5ex minus .2ex,
%  skipbelow=1ex plus .2ex minus .2ex,
%  hidealllines=true,
%  backgroundcolor=Gray!5,
%  innerleftmargin=0.5\baselineskip,
%  innerrightmargin=0.5\baselineskip,
%  innertopmargin=0.4\baselineskip,
%  innerbottommargin=0.4\baselineskip,
%}

% thin frame

%\mdfdefinestyle{style-leftline}{
%  skipabove=1.5ex plus .5ex minus .2ex,
%  skipbelow=1ex plus .2ex minus .2ex,
%  linewidth=1pt,
%  linecolor=Gray!50,
%  topline=false,
%  bottomline=false,
%  rightline=false,
%  innerleftmargin=0.5\baselineskip,
%  innerrightmargin=0,
%  innertopmargin=0.2\baselineskip,
%  innerbottommargin=0.0\baselineskip,
%}

%%% ENVIRONMENTS %%%%%%%%%%%%%%%%%%%%%%%%%%%%%%%%%%%%%%%%%%%%%%%%%%%%%%%%%%%%%%%



% Definition

\theoremstyle{definition}
\newtheorem*{definition}{\definitionname}
\newtheorem*{attention}{\danger\ \attentionname}
\newtheorem*{eg}{\examplename}

\theoremstyle{plain}
\newtheorem*{theorem}{\theoremname}
\newtheorem*{proposition}{\propositionname}
\newtheorem*{corollary}{\corollaryname}
\newtheorem*{lemma}{\lemmaname}

\theoremstyle{remark}
\newtheorem*{remark}{\remarkname}
\newtheorem*{claim}{\claimname}
\newtheorem*{notation}{\notationname}
\newtheorem*{convention}{\conventionname}



%\mdtheorem[
%  style=style-box,
%  linecolor=color-definition,
%  frametitlebackgroundcolor=color-definition
%]{definition}{\definitionname}[section]

% Theorem

%\mdtheorem[
%  style=style-box,
%  linecolor=color-theorem,
%  frametitlebackgroundcolor=color-theorem,
%  font=\itshape
%]{theorem}{\theoremname}[section]

% Proposition

%\mdtheorem[
%  style=style-box,
%  linecolor=color-proposition,
%  frametitlebackgroundcolor=color-proposition,
%  font=\itshape
%]{proposition}[theorem]{\propositionname}

% Corollary

%\mdtheorem[
%  style=style-box,
%  linecolor=color-corollary,
%  frametitlebackgroundcolor=color-corollary,
%  font=\itshape
%]{corollary}[theorem]{\corollaryname}

% Lemma

%\mdtheorem[
%  style=style-box,
%  linecolor=color-lemma,
%  frametitlebackgroundcolor=color-lemma,
%  font=\itshape
%]{lemma}[theorem]{\lemmaname}

%\mdtheorem[
%  style=style-box,
%  linecolor=color-attention,
%  frametitlebackgroundcolor=color-attention,
%  font=\itshape
%]{attention}[theorem]{\danger \attentionname}

%\theoremstyle{remark}

% Remark

%\newtheorem*{remark}{\remarkname}
%\surroundwithmdframed[
%  style=style-background,
%  backgroundcolor=color-remark
%]{remark}

% Example

%\newtheorem*{eg}{\examplename}
%\surroundwithmdframed[
%  style=style-background,
%  backgroundcolor=color-example
%]{eg}

%Notation

%\newtheorem*{notation}{\notationname}
%\surroundwithmdframed[
%  style=style-background,
%  backgroundcolor=color-notation
%]{notation}

%Convention

%\newtheorem*{convention}{\conventionname}
%\surroundwithmdframed[
%  style=style-background,
%  backgroundcolor=color-convention
%]{convention}

%Claim

%\newtheorem*{claim}{\claimname}
%\surroundwithmdframed[
%  style=style-background,
%  backgroundcolor=color-claim
%]{claim}

% Proof

%\surroundwithmdframed[
%  style=style-leftline
%]{proof}

%%% TEXT FORMATTING %%%%%%%%%%%%%%%%%%%%%%%%%%%%%%%%%%%%%%%%%%%%%%%%%%%%%%%%%%%%

% definitions
\let\epsilon\varepsilon
\renewcommand\emptyset{\varnothing}
\let\implies\Rightarrow
\let\impliedby\Leftarrow
\let\ForAll\forall
\renewcommand\forall{\;\ForAll}
\let\Exists\exists
\renewcommand\exists{\;\Exists}

\newcommand*{\df}[1]{\colorbox{color-definition}{\emph{#1}}}


%%% LANGUAGE %%%%%%%%%%%%%%%%%%%%%%%%%%%%%%%%%%%%%%%%%%%%%%%%%%%%%%%%%%%%%%%%%%%

%%%% SETUP %%%%%%%%%%%%%%%%%%%%%%%%%%%%%%%%%%%%%%%%%%%%%%%%%%%%%%%%%%%%%%%%%%%%%%

\usepackage[english]{babel}
\usetikzlibrary{english}
\usepackage[babel,english=quotes]{csquotes}

%%% CAPTION REDEFINITION %%%%%%%%%%%%%%%%%%%%%%%%%%%%%%%%%%%%%%%%%%%%%%%%%%%%%%%

\renewcommand{\figurename}{Figure} % not really necessary, since Babel already implements this
\renewcommand{\tablename}{Table} % not really necessary, since Babel already implements this
\renewcommand*{\proofname}{Proof} % not really necessary, since Babel already implements this
\renewcommand*{\definitionname}{Definition}
\renewcommand*{\theoremname}{Theorem}
\renewcommand*{\propositionname}{Proposition}
\renewcommand*{\corollaryname}{Corollary}
\renewcommand*{\lemmaname}{Lemma}
\renewcommand*{\remarkname}{Remark}
\renewcommand*{\examplename}{Example}
\renewcommand*{\attentionname}{Attention}
\renewcommand*{\notationname}{Notation}
\renewcommand*{\conventionname}{Convention}
\renewcommand*{\claimname}{Claim}

%%% HYPHENATION %%%%%%%%%%%%%%%%%%%%%%%%%%%%%%%%%%%%%%%%%%%%%%%%%%%%%%%%%%%%%%%%



\usepackage{todonotes}
\usepackage{./tikzit/tikzit}
\usepackage{biblatex}
\input{./tikzit/test.tikzstyles}
\addbibresource{./references.bib}

\begin{document}	

\section{Einführung}
In the following we mainly follow Bartholdi's hands on approach from \cite{bartholdi16} to define the forested graph complex.
The more general definition from Conant and Vogtmann from \cite{conant03} has been postponed as it is too complex and unintuitive for the beginning.
Very useful in the general understanding of  what a graph complex is and how the boundary map acts was Bar-Natan's draft [].
Inspired by this similar examples for the forested graph complex are presented.

\begin{definition}
	A \emph{graph} $G$ is a finite $1$-dimensional CW complex. The set of edges is denoted by $E(G)$, the set of vertices by  $V(G)$.
	A loop is an edge having the same start and endpoint.
	We call a graph \emph{connected} if the CW complex is connected in the topological sense.
	A graph is $n$-connected if it remains connected after removing  $n-1$ arbitrary edges.
	A graph is said to be \emph{$n$-regular} if every vertex has valency $n$ i.e. for every vertex the number of incident edges is $n$.

	For a collection of edges $\Phi$ of $G$ we denote by $G / \Phi$ the graph quotient, which is the quotient space of the CW complex $G$ over its topological subspace $\Phi$.

	Lastly a \emph{tree} is a connected graph which contains no cycles and a \emph{forest} is a collection of disjoint trees.
\end{definition}

\begin{definition}
	As the fundamental group of a connected Graph is isomorphic to a free group, it makes sense to define the rank of a graph as the rank of its fundamental group $\pi_{1}(G)$.
\end{definition}

For a proof of this see for example \cite[p. 43f]{hatcher00}

\begin{proposition}
	The rank of a graph $G$ is equal to:
	\begin{enumerate}
		\item The number of independent cycles in $G$
		\item The first Betti number i.e. the rank of $H_{1}(G)$
		\item the Euler Characteristic of $G$ given by $\chi(G) = \abs{C(G)} - \abs{V(G)} + \abs{E(G)}$, 
			where $\abs{C(G)}$ is the number of connected components of $G$.
	\end{enumerate}
\end{proposition}
\begin{proof}[Proof Sketch]
	That the rank is equal to the first Betti number follows from Hurewicz Theorem and the fact that the abelianisation of the free group of rank $n$
	is the free abelian group of rank  $n$.

	The equality of $2)$ and $3)$ follows from the argument for the fundamental group in \cite[p. 43f]{hatcher00}.
	Lastly a proof of the equality of $1)$ and $3)$ can be found in \cite[p. 37-40]{harary69}.
\end{proof}

\begin{eg}\label{ex:gAuto}
	Consider the following graph $G$.

	Then $G$ is $3$-connected and $3$-regular. Moreover $G$ is isomorphic to $H$.
	An isomorphism between them is given by mapping the same colored nodes to each other.
	The graph quotient $G / \{e,f,g\}$ is given by $J$.

	\ctikzfig{./tikzit/graph_automorphism}
\end{eg}

\begin{definition}
	Let $G,H$ be two graphs. A map $f: V(G) \to V(H)$ is said to be a graph isomorphism if $f$ is a bijection such that
	\[
		(u,v) \in E(G) \Leftrightarrow (f(u),f(v)) \in E(H)
	.\] 
\end{definition}

Let $F_{n}$ denote the free group of rank $n.$
\begin{definition}
	An \emph{admissible graph of rank $n$} is a $2$-connected loop less graph $G$ with fundamental group isomorphic to $F_{n}$ and with vertex-valency $\geq 2$.
\end{definition}

We often just write admissible graph for an admissible graph of rank $n$.

\begin{definition}
	Let $G$ be an admissible graph. Its \emph{degree} is given by
	\[
		\deg(G) := \sum_{v \in V(G)} (\deg(v) - 3)
	.\] 
\end{definition}

In particular $G$ has $2n -2 - \deg(G)$ vertices and $3n -3 - \deg(G)$ edges and is $3$-regular iff $\deg(G) = 0$.

\begin{definition}
	An orientation $\sigma$ of a graph $G$ is an ordering of the edges i.e. $\sigma$ is an injective function from $E(G)$ to $\{1,\ldots,\abs{E(G)}\}$.
	We call the tuple $(G,\sigma)$ an oriented graph and note that $\on{Sym}$ acts on $(G,\sigma)$ by $\pi (G,\sigma) = (G,\pi \sigma)$ for $\pi \in \on{Sym}$.
\end{definition}

If it is clear that $(G,\sigma)$ is an oriented graph we often just write $G$.

\begin{definition}
	A \emph{forested graph} is a pair $(G,\Phi)$ where $G$ is an admissible graph and $\Phi$ is an oriented forest which contains all vertices of $G$.
	\todo[inline]{All vertices cant be either because then boundary doesnt work?}

	A map $f$ between two forested graphs $(G,\Phi), (H,\Psi)$ is said to be a forested graph isomorphism if $f$ is a graph isomorphism and $\sigma \circ f \mid_{\Phi} $  is
	the identity, where $\sigma$ is the orientation on $\Phi$.
	\todo{Is this correct?}
\end{definition}

For $k \in \N$ let $C_{k}$ denote the $\Q$-vector space spanned by isomorphism classes of forested graphs of rank $n$ with a forest of size $k$, subject to the relation
\[
	(G,\pi \Phi) = \sgn{\pi} \cdot (G,\Phi) \qq{for all} \pi \in \on{Sym}(k)
.\]
Observe that if $(G,\Phi) \simeq (G,\pi \Phi)$ for an odd permutation $\pi$ then $(G,\Phi) \simeq (G,\pi \Phi) = - (G,\Phi)$ and thus $(G,\Phi) = 0$ in  $C_{k}$.

\todo[inline]{If $C_{k}$ is only of rank $n$ how can the boundary map be well defined when it reduces edge number by $1$ thus reducing rank by $1$}
\begin{eg}
	Consider the graphs $G, J$ as in Example \ref{ex:gAuto}. Then $G$ and $J$ are admissible graphs of rank $3$ and $2$. Thus if we equip them with oriented forests $\Phi, \Psi$ as below 
	(where the red edges represent the forest and the numbers the orientation) we get forested graphs.

	\ctikzfig{./tikzit/forested_graph_automorphism}

	Observe, that $(J,\Psi) = 0$ in $C_{k}$, since $ (1 2)$ is an odd permutation and  $(1 2) (J,\Psi)$ is isomorphic to $(J,\Psi)$ 
	via the isomorphism mirroring along the vertical passing through the blue and purple vertex.

	$(G,\Phi)$ however is not trivial which can be seen as follows : 
	\todo[inline]{show that}
\end{eg}

To turn these spaces into a chain complex we define a differential as follows:
\begin{definition}
	Let $(G,\Phi) = (G, \{e_1,\ldots,e_{p}\} )$ be a forested graph. Then let
	\begin{align*}
		\partial_{C}(G,\Phi) = \sum_{i=1}^{p} (-1)^{i} (G / e_{i}, \Phi \setminus \{e_i\}),\\
		\partial _{R}(G,\Phi) = \sum_{i=1}^{p} (-1)^{i} (G,\Phi \setminus \{e_{i}\}) 
	\end{align*}
	where if $\sigma: E(\Phi) \to \{1,\ldots,p\} $ is the orientation on $\Phi$ the orientation $\tau: E(\Phi \setminus \{e_{i}\}) \to \{1,\ldots,p-1\} $ 
	on $\Phi \setminus \{e_{i}\}$ is given by 
	\[
		\tau(e) = \begin{cases}
			\sigma(e) & \text{ if }\sigma(e) < i\\
			\sigma(e) - 1 & \text{ if } \sigma(e) > i
		\end{cases}
	.\]
	Notice that the case $\sigma(e) = i$ can't happen as $e_{i}$ is not contained in $\Phi \setminus \{e_{i}\}$. 
	Finally define the boundary map $\partial = \partial_{C} - \partial_{R}$.
	\todo[inline]{Is this renumbering correct or do we have to move $e_{i}$ first to the end of the orientation and then remove it inducing a sign of $-1^{i-1}$}
\end{definition}

\begin{proposition}
	$\partial$ is well-defined and $\partial^2 = 0$.
\end{proposition}

\begin{proof}
	Let $(G,\Phi) = (G, \{e_1,\ldots,e_{p}\})$ be a forested graph and denote the edges in $\Phi \setminus \{e_{i}\}$ by $\{e_1',\ldots,e_{p-1}'\}$.
	Firstly we observe that 
	\[
		(G / e_{i}) /  e_{j}' = \begin{cases}
			(G / e_{j}) / e_{i-1}' & \text{ if } i > j\\
			(G / e_{j+1}) / e_{i}' & \text{ if } i \leq j
		\end{cases}
		\qq{ as well as }
		(\Phi \setminus \{e_{i}\}) \setminus \{e_{j}'\}  = \begin{cases}	
			(\Phi \setminus \{e_{j}\}) \setminus \{e_{i-1}'\} & \text{ if } i > j\\
			(\Phi \setminus \{e_{j+i}\}) \setminus \{e_{i}'\} & \text{ if } i \leq j
		\end{cases}
	\]
	Now we compute:
	\begin{align*}
		\partial_{C}^2 &= \partial_{C} \sum_{i=1}^{p} (-1)^{i}(G / e_{i}, \Phi \setminus \{e_{i}\})
		=  \sum_{i=1}^{p} \sum_{j=1}^{p-1} (-1)^{i+j}((G / e_{i}) / e_{j}', (\Phi \setminus \{e_{i}\} ) \setminus \{e_{j}'\})  \\
					   &= \sum_{j < i} (-1)^{i+j} ((G / e_{i}) / e_{j}', (\Phi \setminus \{e_{i}\} ) \setminus \{e_{j}'\}) + \sum_{i \leq j} (-1)^{i+j}
					   ((G / e_{i}) / e_{j}', (\Phi \setminus \{e_{i}\} ) \setminus \{e_{j}'\}) 
	\end{align*}
	We claim that the right and left sum cancel. For this first apply the observations above to the left sum and then change variables by setting $l = j$ and  $m = i-1$ to obtain:
	\begin{align*}
		\sum_{j < i} (-1)^{i+j} ((G / e_{i}) / e_{j}', (\Phi \setminus \{e_{i}\}) \setminus \{e_{j}'\} ) &= 
		\sum_{j < i} (-1)^{i+j}((G / e_{j}) / e_{i-1}', (\Phi \setminus \{e_{j}\}) \setminus \{e_{i-1}'\} ) \\ 
		&= \sum_{l \leq m} (-1)^{l+m+1} ((G / e_{l}) / e_{m}', (\Phi \setminus \{e_{l}\}) \setminus \{e_{m}'\} ) 
	\end{align*}
	This last expression is the same as the left sum above but with opposite sign. Thus they cancel and we have shown $\partial_{C}^2 = 0$.
	The same argument shows that $\partial_{R}^2 = 0$.

	For the mixed terms we compute as follows
	\begin{align*}
		\partial_{R} \partial_{C} &=  \sum_{i=1}^{p} \sum_{j=1}^{p-1} (-1)^{i+j}(G / e_{i}, (\Phi \setminus \{e_{i}\} ) \setminus \{e_{j}'\})  \\
					   &= \sum_{j < i} (-1)^{i+j} (G / e_{i}, (\Phi \setminus \{e_{i}\} ) \setminus \{e_{j}'\}) + \sum_{i \leq j} (-1)^{i+j}
					   (G / e_{i}, (\Phi \setminus \{e_{i}\} ) \setminus \{e_{j}'\}) \tag{$*$}
	\end{align*}
	and
	\begin{align*}
		\partial_{C} \partial_{R} &=  \sum_{i=1}^{p} \sum_{j=1}^{p-1} (-1)^{i+j}(G / e_{j}', (\Phi \setminus \{e_{i}\} ) \setminus \{e_{j}'\})  \\
					   &= \sum_{j < i} (-1)^{i+j} (G / e_{j}', (\Phi \setminus \{e_{i}\} ) \setminus \{e_{j}'\}) + \sum_{i \leq j} (-1)^{i+j}
					   (G / e_{j}', (\Phi \setminus \{e_{i}\} ) \setminus \{e_{j}'\}) \\
					   &\stackrel{(\heartsuit)}{=} \sum_{j < i} (-1)^{i+j} (G / e_{j}, (\Phi \setminus \{e_{j}\} ) \setminus \{e_{i-1}'\}) + \sum_{i \leq j} (-1)^{i+j}
					   (G / e_{j+1}, (\Phi \setminus \{e_{j+1}\} ) \setminus \{e_{i}'\}) \\
					   &\stackrel{(\dagger)}{=} \sum_{l \leq m} (-1)^{m+l+1} (G / e_{l}, (\Phi \setminus \{e_{l}\} ) \setminus \{e_{m}'\}) + \sum_{k < n} (-1)^{n+k-1}
					   (G / e_{k}, (\Phi \setminus \{e_{k}\} ) \setminus \{e_{n}'\}) \tag{$* *$}
	\end{align*}
	Where in $(\heartsuit)$ we used that if $j < i$ then $e_{j} = e_{j}'$ and if $i \leq j$ then $e_{j+1} = e_{j}'$, as well as the observation above.
	In $(\dagger)$ we used the substitution  $m = i-1$,  $l = j$ on the left and  $m = i$,  $k = j+1$ on the right sum.
	Comparing the sums in $(*)$ and $(* *)$ we see that they differ by a sign and thus cancel. Hence  $\partial_{C} \partial_{R} - \partial_{R} \partial_{C} = 0$.

	Combining the above we get:
	\[
		\partial^2 = (\partial_{C} - \partial_{R})^2 = \partial_{C}^2 - (\partial_{C} \partial_{R} + \partial_{R} \partial_{C}) + \partial_{R}^2 = 0
	\]
\end{proof}

Thus the spaces $(C_{\bullet})$ with the differential $\partial_{\bullet}$ form a chain complex.

\begin{eg}
	Once again we consider the graph $(G,\Phi)$ from above and calculate its boundary operator:
	\ctikzfig{./tikzit/boundary_operator}
	Where we used that $-H_{1}$ is equal to $H_{2}$ by mirroring along the vertical and applying $(2 3)$,
	$-H_{3}$ is equal to $H_2$ by exchanging inner and outer vertices, mirroring along the vertical and applying $(1 3)$ and
	$H_{4}$ is equal to $H_{2}$ by exchanging inner and outer vertices and applying $(1 3)(2 3)$.
	Where we have used the permutation $(2 3)$ on the first,  $(1 3)$ on the third, $(1 3)(2 3)$ on the fourth graph to get the result.
	\ctikzfig{./tikzit/boundary_operator2}
	Where we have used that $-G_{1}$ is equal to $-G_{3}$ by exchanging inner and outer vertices and applying $(1 3)(1 2)$,
	$G_{2}$ is equal to $-G_{3}$ by exchanging, mirroring along the vertical and applying $(1 3)$ and
	$G_{4}$ is equal to $-G_{3}$ by mirroring along the vertical and applying $(1 2)$. 

	Thus we can conclude that $\partial_{\bullet} G = 4 H_2 - 4 G_3$. Moreover we have that $H_2 - G_3 \in \im \partial_{\bullet}$ and 
	as $\partial_{\bullet}^2=0$ also $H_2 - G_3 \in \ker \partial_{\bullet}$
\end{eg}

\printbibliography

%\input{../header}

\section{Draftparts}

\begin{definition}
	A \emph{graph} $G$ is a finite $1$-dimensional CW complex. The set of edges is denoted by $E(G)$, the set of vertices by  $V(G)$ and the set of half edges by  $H(G)$.
	We call a graph \emph{connected} if the CW complex is connected in the topological sense.
	A graph is said to be \emph{$n$-valent} if every vertex has valency $n$ i.e. for every vertex the number of edges incident is  $n$.

	Lastly a \emph{tree} is a graph which contains no loops and a \emph{forest} is a collection of disjoint trees.
\end{definition}

On trivalent connected graphs we call an \emph{orientation} a choice of cyclic orders of all vertices up to an even number of changes.

\begin{definition}
	A \emph{forested graph} is a pair $(G, \Phi)$, where $G$ is a finite connected trivalent graph and $\Phi$ is an oriented forest which contains all vertices of $G$. 
\end{definition}

\begin{definition}
Let $(G,\Phi)$ be a forested graph and let  $e \in \Phi$. Moreover let $(G_{e},\Phi_{e})$ be the graph where $e$ has been collapsed.
Then there exist exactly two other graphs whose edge collapse results in $(G_{e},\Phi_{e})$. This is visualised in the figure below.
Where  $1,2,3,4$ represent the rest of the graph.

\ctikzfig{./tikzit/IHXRelator}
Now the vector 
\[
	(G,\Phi) + (G',\Phi') + (G'',\Phi'')
\]
is called the \emph{basic IHX relator} associated to $(G,\Phi,e)$.
\end{definition}

Denote by $\widehat{\fg}_{k}$ the vector space spanned by all forested graphs containing $k$ trees modulo the relations $(G,\Phi) = -(G,-\Phi)$.
Moreover let $\fg_{k}$ be the quotient of $\widehat{\fg}_{k}$ modulo the subspace spanned by all basic IHX relators.


\begin{definition}
	Let $\widehat{\partial}_E(G,\Phi): \widehat{\fg}_{k} \to \widehat{\fg}_{k-1}$ be given by
\begin{align*}
	\widehat{\partial}_{E}(G,\Phi) = \sum (G, \Phi \cup e)
.\end{align*}
where the sum is over all edges $e$ of $G \setminus \Phi$ such that $\Phi \cup e$ is still a forest.
Notice that this only happens if the two vertices of  $e$ lie in different trees of $\Phi$. Thus $\Phi \cup e$ has $k-1$ components.
The orientation of $\Phi \cup e$ is determined by ordering the edges of $\Phi$ with labels  $1,\ldots,k$ consistent with its orientation
and then labeling the new edge $e$ with  $k+1$.

Now let the boundary map $\partial_{E}: \fg_{k} \to \fg_{k-1}$ be the map induced by $p \circ \widehat{\partial}_{E}$ where $p$ is the quotient map $\widehat{\fg_{k}} \to \fg_{k}$.
\end{definition}

\begin{proposition}
	$\partial_{E}$ is well-defined and $\partial_{E}^2 = 0$.
\end{proposition}

\begin{proof}
	
\end{proof}

The \emph{forested graph complex} is thus defined as the sequence $\fg_{k}$ with boundary map $\partial_{E}$ and is well-defined by the above proposition.

\subsection{The Outer space}
We closely follow Vogtmann's definition from \cite[p. 2 ff.]{vogtmann16}
\begin{definition}
	By a \emph{metric graph} we mean a finite connected graph with positive real edge lengths, equipped with the path metric.
	We fix a model rose $R_{n}$ (a graph with one vertex and $n$ petals), and identify the petals of $R_{n}$ with the generators
	of the free group $F_{n}$. A point in $\mathcal{X}_{n}$ is then a metric graph $G$ together with a homotopy
	equivalence $g: R_{n} \to  G$ called a \emph{marking}; the marking serves to identify the fundamental group of $G$ with $F_{n}$.
	Marked graphs $(g,G)$ and $(g',G')$ are considered the same if there is an isometry  $f: G \to G'$ with $f \circ g$ homotopic to $g'$.

	To get a finite dimensional space we assume $G$ has no uni- and bivalent vertices (see Theorem \ref{thm:finGenCn}).
	Moreover we normalize our objects i.e. we assume the sum of edge lengths to be $1$ and assume that $G$ is $2$-connected.
\end{definition}
To make  $\mathcal{X}_{n}$ a space we need to define a topology. We proceed as follows: 
For every marked graph $(g,G)$ we define the open simplex $\sigma(g,G)$ as the set obtained by varying the edge lengths of $G$,
keeping their sum equal to $1$.
The simplex  $\sigma(g',G')$ is then a \emph{face} of $\sigma(g,G)$ if $(g',G')$ can be obtained from $(g,G)$ by collapsing some edges to points.

Finally $\mathcal{X}_{n}$ is the quotient space obtained from the disjoint union of the open simplices $\sigma(g,G)$ by face identification.

However not all faces of these simplices are in $\mathcal{X}_{n}$.
To rectify this we replace each open simplex $\sigma(g,G)$ by a closed simplex $\overline{\sigma}(g,G)$ and take
the quotient as before. This new space, denoted by $\mathcal{X}^{*}_{n}$, is a simplical complex and called the \emph{simplical closure} of Outer space.
The points in $\mathcal{X}^{*}_{n}$ which are not in $\mathcal{X}_{n}$ are said to be at infinity.

 Now the group $\out(F_{n})$ acts on $\mathcal{X}_{n}$ by changing the marking in particular
 any $\varphi \in \out(F_{n})$ can be realized by a homotopy equivalence $f: R_{n} \to R_{n}$
 by mapping petals to each other according to their identification with generators of $F_{n}$.
 The group action by $\varphi$ on $(g,G)$ is then defined by
 \[
	 (g,G) \varphi = (g \circ f, G)
 .\] 

 Finally $\mathcal{X}_{n}$ contains an equivariant deformation retract $K_{n}$, the spine of Outer space.
 It is a subcomplex of the barycentric subdivision of the simplical closure $\mathcal{X}_{n}^{*}$, consisting of simplices spanned by vertices which are not at infinity.

 In other language, $K_{n}$ is the geometric realisation of the partially ordered set of open simplices $\sigma(g,G)$ in $\mathcal{X}_{n}$, where 
 the partial order is given by the face relation.

 We have the following vital result providing the relationship between $\mathcal{X}_{n}$ and $\out(F_{n})$. This was proved by Culler and Vogtmann in \cite{vogtmann86}.
 \begin{theorem}
	 $\mathcal{X}_{n}$ is contractible and the action of $\out(F_{n})$ is proper.
	 The spine $K_{n}$ is an equivariant deformation retract of dimension $2 n - 3$ with compact quotient
 \end{theorem}
 
 \section{Introduction\footnotemark}
\footnotetext{This section follows closely Vogtmann's survey article \cite{vogtmann16}}
Free groups are one of the most basic examples of infinite finitely-generated groups.
The automorphism groups of free groups are of particular interest since they are tied to many other areas of mathematics.
The automorphism group $\aut(G)$ of a group $G$ can be separated into the inner automorphism group $\on{Inn}(G)$,
the group of automorphisms that arise from conjugation, and the outer automorphism group $\out(G)$, the quotient of  $\aut(G)$ by $\on{Inn}(G)$.

Most often the inner automorphism groups are well understood.  To understand the outer automorphism groups
the standard approach has been to study it via its action on some topological space.
For the outer automorphism group of the free group on $n$ generators $\out(F_{n})$ the space $\mathcal{X}_{n}$ known as "Outer space"
has been introduced by Culler and Vogtmann in \cite{vogtmann86}. As the points of the Outer space correspond to finite graphs with fundamental group $F_{n}$,
the structure of Outer space is connected to such diverse areas as the study of Lie algebras of derivations, degenerations of algebraic varieties,
the computation of Feynman integrals, and the statistics of phylogenetic trees.

A central concept of the Outer space is its spine $K_{n}$ which is an equivariant deformation retract of $\mathcal{X}_{n}$.
By being able to compute the rational homology of $K_{n} / \out(F_{n})$ one can also compute $\out(F_{n})$.
By identifying the spine $K_{n}$ with a cube complex one arrives at the forested graph complex
which has been introduced by Conant and Vogtmann in \cite{conant03}.

The forested graph complex is also the main focus of this paper.
We will begin by defining basic concepts of graphs, the Outer space and the forested graph complex as well as show the connection via the cube complex.

Afterwards we will focus on Morita's classes which are an infinite sequence of cocycles representing potentially nontrivial cohomology classes $\mu_{k} \in H^{4k}(\out(F_{2k+2})$.






\end{document}
