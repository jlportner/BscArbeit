%\input{../header}
\input{./setup/MAIN.tex}
\usepackage{todonotes}
\usepackage{./tikzit/tikzit}
\usepackage{biblatex}
\input{./tikzit/test.tikzstyles}
\addbibresource{./references.bib}

\begin{document}	

\section{Introduction}
In the following we mainly follow Bartholdi's hands on approach from \cite{bartholdi16} to define the forested graph complex.
The more general definition by Conant and Vogtmann from \cite{conant03} has been postponed as it is too complex and unintuitive for the beginning.
Very useful in the general understanding of  what a graph complex is and how the boundary map acts was Bar-Natan's draft [].
Inspired by this similar examples for the forested graph complex are presented.

\section{Basic Definitions}
\subsection{Graphs}
\begin{definition}
	A \emph{graph} $G$ is a finite $1$-dimensional CW complex. The set of edges is denoted by $E(G)$, the set of vertices by  $V(G)$.
	We call an edge having the same start and end vertex a loop.

	We call a graph \emph{connected} if the CW complex is connected in the topological sense.
	A graph is $n$-edge-connected if it remains connected after removing  $n-1$ arbitrary edges.

	A graph is said to be \emph{$n$-regular} if every vertex has valency $n$ i.e. for every vertex the number of incident edges is $n$.

	For a subset of edges $\Phi$ of $G$ we denote by $G / \Phi$ the graph quotient, which is the quotient space of the CW complex $G$ over its topological subspace $\Phi$.
\end{definition}

\begin{remark}
	Note that in classical graph theory these types of graphs are called multigraphs, as they are allowed to have multiple edges between vertices as well as loops.
	The word graph there normally refers to simple graphs which don't allow multi-edges and loops.

	In the context of algebraic topology however multigraphs are needed and thus the word graph here denotes multigraphs.
\end{remark}

\begin{definition}
	A sub graph $G'$ of a graph $G$ is a subcomplex of the CW complex $G$. As a subcomplex is itself a CW complex of dimension smaller or equal to the original complex,
	 $G'$ is itself a graph.
\end{definition}

\begin{theorem}
	Let $G$ be a graph. Then its fundamental group $\pi_{1}(G)$ is isomorphic to a free group.
\end{theorem}
By the theorem it makes sense to define the \emph{rank} of a graph $\rank(G)$ as the rank of its fundamental group.

\begin{proof}
	
\end{proof}

For a proof of this see for example \cite[p. 43f]{hatcher00}

To understand the rank better we will need the following definition
\begin{definition}
	For a finite Cw-complex $X$ the \emph{Euler Characteristic} is defined as the alternating sum
	\[
		\chi(X) = k_0 - k_1 + k_2 - \ldots
	\] 
	where $k_{i}$ denotes the number of cells of dimension $i$ in the complex $X$.
\end{definition}
For graphs we thus get $\chi (G) = k_0 - k_1$, as they are $1$-dimensional which is equal to $\chi(G) = \abs{V} - \abs{E}$.

The following proposition 
\begin{proposition}\label{prop:rank}
	The rank of a graph $G$ is equal to:
	\begin{enumerate}
		\item The number of independent cycles in $G$
		\item The first Betti number i.e. the rank of $H_{1}(G)$
		\item $1 - \chi(G) = \abs{E} - \abs{V} + 1$
	\end{enumerate}
\end{proposition}

\begin{proof}[Proof Sketch]
	That the rank is equal to the first Betti number follows from Hurewicz Theorem and the fact that the abelianisation of the free group of rank $n$
	is the free abelian group of rank  $n$.

	The equality of $2)$ and $3)$ follows from the argument for the fundamental group in \cite[p. 43f]{hatcher00}.
	Lastly a proof of the equality of $1)$ and $3)$ can be found in \cite[p. 37-40]{harary69}.
\end{proof}

Lastly a \emph{tree} is a connected graph which contains no cycles and a \emph{forest} is a collection of disjoint trees.

\begin{definition}
	Let $G,H$ be two graphs. A map $f: V(G) \to V(H)$ is said to be a graph isomorphism if $f$ is a bijection such that
	\[
		(u,v) \in E(G) \Leftrightarrow (f(u),f(v)) \in E(H)
	.\] 
\end{definition}

\begin{eg}\label{ex:gAuto}
	Consider the following graphs.

	\ctikzfig{./tikzit/graph_automorphism}
	Then $G$ is $3$-connected and $3$-regular. Moreover $G$ is isomorphic to $H$.
	An isomorphism between them is given by mapping the same colored nodes to each other.
	The graph quotient $G / \{e,f,g\}$ is given by $J$.
\end{eg}

\subsection{Forested graph Complex}
Let $F_{n}$ denote the free group of rank $n.$
\begin{definition}
	An \emph{admissible graph of rank $n$} is a $2$-edge-connected graph $G$ with fundamental group isomorphic to $F_{n}$ and with vertex-valency $\geq 3$.
\end{definition}


We often just write admissible graph for an admissible graph of rank $n$.

\begin{definition}
	Let $G$ be an admissible graph. Its \emph{degree} is given by
	\[
		\deg(G) := \sum_{v \in V(G)} (\deg(v) - 3)
	.\] 
\end{definition}

In particular $G$ has $2n -2 - \deg(G)$ vertices and $3n -3 - \deg(G)$ edges and is $3$-regular iff $\deg(G) = 0$.

\begin{definition}
	An orientation $\sigma$ of a graph $G$ is an ordering of the edges i.e. $\sigma$ is an injective function from $E(G)$ to $\{1,\ldots,\abs{E(G)}\}$.
	We call the tuple $(G,\sigma)$ an oriented graph and note that $\on{Sym}$ acts on $(G,\sigma)$ by $\pi (G,\sigma) = (G,\pi \sigma)$ for $\pi \in \on{Sym}$.
\end{definition}

If it is clear that $(G,\sigma)$ is an oriented graph we often just write $G$.

\begin{definition}
	A \emph{forested graph} is a tripple $(G,\Phi,\sigma)$ where $G$ is an admissible graph $\Phi$ is a forest on $G$ and  $\sigma$ is an orientation on $\Phi$.

	A map $f$ between two forested graphs $(G,\Phi, \sigma) \to (H,\Psi, \tau)$ is said to be a forested graph isomorphism if 
	$f$ is a graph isomorphism on $G$,  $f(\Phi) = \Psi$ and $\sigma = \tau \circ f $
\end{definition}

We know want to construct our graph complex. For this we remember the notion of a graded vector space:
\begin{definition}
	A graded vector space, is a vector space $V$ and with a decomposition $\left(V_{n}\right)^{\infty}_{n=0} $ such that
	\[
		V = \bigoplus_{k=0}^{\infty} V_{n}
	\] 
\end{definition}

We know consider the $\Q$-vector space $C$ spanned by isomorphism classes of forested graphs, subject to the relation
\[
	(G,\pi \Phi) = \sgn{\pi} \cdot (G,\Phi) \qq{for all} \pi \in \on{Sym}(k)
.\]
Observe that if $(G,\Phi) \simeq (G,\pi \Phi)$ for an odd permutation $\pi$ then $(G,\Phi) \simeq (G,\pi \Phi) = - (G,\Phi)$ and thus $(G,\Phi) = 0$ in  $C_{k}$.

We can define the following three gradings on $C$:
 \begin{itemize}
	\item Let $C^{n} \subseteq C$ be the subspace spanned by forested graphs of rank $n$. Then clearly  $C^{n} \cap C^{m} = \emptyset$ for $n \neq m$ and
		as every graph has a rank, we get that the $C^{n}$ form a grading of $C$.
	\item Let $C_{k} \subseteq C$ be the subspace spanned by forested graphs with a forest of size $k$. Clearly this also yields a decomposition of $C$ into a direct sum
		and thus yields another grading of $C$
	\item Let  $C_{d} \subseteq C$ be the subspace spanned by forested graphs of degree $d$. Once again this yields a grading on $C$
\end{itemize}
In the following we will mostly be concerned with the first two gradings. In particular we will consider the double-grading $C_{k}^{n}$,
where $k$ denotes the forest size and $n$ the rank. 

\begin{eg}
	Consider the graphs $G, J$ as in Example \ref{ex:gAuto}. Then $G$ and $J$ are admissible graphs of rank $3$ and $2$. Thus if we equip them with oriented forests $\Phi, \Psi$ as below 
	(where the red edges represent the forest and the numbers the orientation) we get forested graphs.

	\ctikzfig{./tikzit/forested_graph_automorphism}

	Observe, that $(J,\Psi) = 0$ in $C_{2}^{3}$, since $ (1 2)$ is an odd permutation and  $(1 2) (J,\Psi)$ is isomorphic to $(J,\Psi)$ 
	via the isomorphism mirroring along the vertical passing through the blue and purple vertex.

	$(G,\Phi)$ however is not trivial which can be seen as follows : 
	\todo[inline]{show that}
\end{eg}

Before we construct the chain complex we show that the $C^{n}$ are finitely generated and thus so are the $C_{k}^{n}$.

\begin{theorem}
	For all $n$, $C^{n}$ is finitely generated and for $C_{k}^{n} = 0 \forall k > 2n-3$. 
\end{theorem}

\begin{proof}
	We have the following general fact
	\[
		2 \abs{E} = \sum_{v \in V} \deg{v}
	.\] 
	Using that admissible graphs have vertex-valency $\geq 3$ and rearranging yields $\abs{E} \geq \frac{3}{2} \abs{V}$.
	From Proposition \ref{prop:rank} we get that $\abs{E} = \abs{V} + n-1$.
	Combining yields
	\[
		\abs{V} + n -1 \geq \frac{3}{2} \abs{V} \Leftrightarrow 2 (n-1) \geq \abs{V}
	\] 
	and plugging in the above in the identity from the Proposition yields $\abs{E} \leq 3 (n-1)$.

	With this we can get a loose upper bound on the amount of admissible graphs.
	As every graph can be written as incidence matrix and each entry is $\leq \abs{E}$ we get that there are maximally
	$\abs{E}^{\abs{V}^2}$ many different incidence matrices for graphs with  $\abs{V}$ vertices.
	As every graph corresponds to an incidence matrix this also gives an upper bound on the number of different graphs with $\abs{V}$ vertices.

	Thus the maximal possible number of admissible graphs of rank $n$ is bounded by
	\[
		\sum_{k=1}^{2(n-1)} (3 (n-1))^{k^2} 
	\] 
	As on each graph there also only exists a finite number of forests and each of them has a finite number of orientations we get that
	$C^{n}$ is finitely generated.

	That $C_{k}^{n} = 0 \forall k > 2n -3$ follows from the bound on the number of vertices and the fact that a forest in a graph has maximally $\abs{V} - 1$ edges.
\end{proof}

\begin{remark}
	Notice that the constraint of vertex-valency $\geq 3$ in the definition of admissible graphs is necessary for $C^{n}$ being finitely generated.
	As else we can consider the following family of graphs:
	\ctikzfig{./tikzit/infinite_admissible_graphs}
	They all have rank $n$ (which can be checked via the Euler characteristic), are $2$-edge-connected and not isomorphic as they have different number of vertices/edges.
\end{remark}

\begin{remark}
	The bound on the $C_{k}^{n}$ can not be improved as the graph $J$ from the example above with the tree extended by the edge between purple and green has rank $3$ and
	tree size $3$ which equals $2 \cdot 3 -3$.
\end{remark}

To construct our chain complex we fix the rank $n$ and define a differential as follows:
\begin{definition}
	Let $(G,\Phi,\sigma) = (G, \{e_1,\ldots,e_{k}\},\sigma)$ be a forested graph. Then let $\partial_{C}, \partial_{R}: C_{k}^{n} \to C_{k-1}^{n}$ be given by
	\begin{align*}
		\partial_{C}(G,\Phi) &= \sum_{i = 1}^{k} (-1)^{k} (G / e_{i}, \Phi \setminus \{e_{i}\}, \sigma_{e_{i}}),\\
		\partial _{R}(G,\Phi) &= \sum_{k = 1}^{k} (-1)^{k} (G,\Phi \setminus \{e_{i}\}, \sigma_{e_{i}}) 
	\end{align*}
	where $\sigma_{e_{i}}: \Phi \ \{e\} \to \{1,\ldots,k-1\}$ is given by
	\[
		\sigma_{e_{i}}(e) = \begin{cases}
			\sigma(e) & \text{ if }\sigma(e) < i\\
			\sigma(e) - 1 & \text{ if } \sigma(e) > i
		\end{cases}
	.\]
	Notice that the case $\sigma(e) = i$ can't happen as $e_{i}$ is not contained in $\Phi \setminus \{e_{i}\}$. 
	Finally define the boundary map $\partial = \partial_{C} - \partial_{R}$.
	%\todo[inline]{Is this renumbering correct or do we have to move $e_{i}$ first to the end of the orientation and then remove it inducing a sign of $-1^{i-1}$}
\end{definition}

\begin{proposition}
	$\partial$ is well-defined and $\partial^2 = 0$.
\end{proposition}

For better readability we will omit the orientation $\sigma$ in the proof.
\begin{proof}


	We proof the result in three steps:\\
	\textbf{Step 1:} Contracting an edge of a graph doesn't change the Euler characteristic as both the vertex number and the edge number decreases by one.
	Thus $\partial_{C}$ preserves the rank of the graph. Moreover the vertex-valency stays $\geq 3$ and the graph continues to be $2$-edge-connected.
	Hence is admissible.
	Moreover $\partial_{C}$ as well as $\partial_{R}$ remove one edge from each graph. Thus decreasing
	$k$ by $1$. Hence both maps are well-defined from $C^{n}_{k}$ to $C^{n}_{k-1}$ and thus so is $\partial$.

	Let $(G,\Phi) = (G, \{e_1,\ldots,e_{p}\})$ be a forested graph and denote the edges in $\Phi \setminus \{e_{i}\}$ by $\{e_1',\ldots,e_{p-1}'\}$.
	For the consecutive steps we need the following observations:
	\[
		(G / e_{i}) /  e_{j}' = \begin{cases}
			(G / e_{j}) / e_{i-1}' & \text{ if } i > j\\
			(G / e_{j+1}) / e_{i}' & \text{ if } i \leq j
		\end{cases}
		\qq{ as well as }
		(\Phi \setminus \{e_{i}\}) \setminus \{e_{j}'\}  = \begin{cases}	
			(\Phi \setminus \{e_{j}\}) \setminus \{e_{i-1}'\} & \text{ if } i > j\\
			(\Phi \setminus \{e_{j+i}\}) \setminus \{e_{i}'\} & \text{ if } i \leq j
		\end{cases}
	\]

	\textbf{Step 2:}\enskip
	\emph{Claim:}
		$\partial_{C}^2 = 0$ and $\partial_{R}^2 = 0$

	We compute:
	\begin{align*}
		\partial_{C}^2 &= \partial_{C} \sum_{i=1}^{p} (-1)^{i}(G / e_{i}, \Phi \setminus \{e_{i}\})
		=  \sum_{i=1}^{p} \sum_{j=1}^{p-1} (-1)^{i+j}((G / e_{i}) / e_{j}', (\Phi \setminus \{e_{i}\} ) \setminus \{e_{j}'\})  \\
					   &= \sum_{j < i} (-1)^{i+j} ((G / e_{i}) / e_{j}', (\Phi \setminus \{e_{i}\} ) \setminus \{e_{j}'\}) + \sum_{i \leq j} (-1)^{i+j}
					   ((G / e_{i}) / e_{j}', (\Phi \setminus \{e_{i}\} ) \setminus \{e_{j}'\}) 
	\end{align*}
	We claim that the right and left sum cancel. For this first apply the observations above to the left sum and then change variables by setting $l = j$ and  $m = i-1$ to obtain:
	\begin{align*}
		\sum_{j < i} (-1)^{i+j} ((G / e_{i}) / e_{j}', (\Phi \setminus \{e_{i}\}) \setminus \{e_{j}'\} ) &= 
		\sum_{j < i} (-1)^{i+j}((G / e_{j}) / e_{i-1}', (\Phi \setminus \{e_{j}\}) \setminus \{e_{i-1}'\} ) \\ 
		&= \sum_{l \leq m} (-1)^{l+m+1} ((G / e_{l}) / e_{m}', (\Phi \setminus \{e_{l}\}) \setminus \{e_{m}'\} ) 
	\end{align*}
	This last expression is the same as the left sum above but with opposite sign. Thus they cancel and we have shown $\partial_{C}^2 = 0$.
	The same argument shows that $\partial_{R}^2 = 0$.

	\textbf{Step 3:} \emph{Claim:} $\partial_{C} \partial_{R} - \partial_{R} \partial_{C} = 0$

	For the mixed terms we compute as follows
	\begin{align*}
		\partial_{R} \partial_{C} &=  \sum_{i=1}^{p} \sum_{j=1}^{p-1} (-1)^{i+j}(G / e_{i}, (\Phi \setminus \{e_{i}\} ) \setminus \{e_{j}'\})  \\
					   &= \sum_{j < i} (-1)^{i+j} (G / e_{i}, (\Phi \setminus \{e_{i}\} ) \setminus \{e_{j}'\}) + \sum_{i \leq j} (-1)^{i+j}
					   (G / e_{i}, (\Phi \setminus \{e_{i}\} ) \setminus \{e_{j}'\}) \tag{$*$}
	\end{align*}
	and
	\begin{align*}
		\partial_{C} \partial_{R} &=  \sum_{i=1}^{p} \sum_{j=1}^{p-1} (-1)^{i+j}(G / e_{j}', (\Phi \setminus \{e_{i}\} ) \setminus \{e_{j}'\})  \\
					   &= \sum_{j < i} (-1)^{i+j} (G / e_{j}', (\Phi \setminus \{e_{i}\} ) \setminus \{e_{j}'\}) + \sum_{i \leq j} (-1)^{i+j}
					   (G / e_{j}', (\Phi \setminus \{e_{i}\} ) \setminus \{e_{j}'\}) \\
					   &\stackrel{(\heartsuit)}{=} \sum_{j < i} (-1)^{i+j} (G / e_{j}, (\Phi \setminus \{e_{j}\} ) \setminus \{e_{i-1}'\}) + \sum_{i \leq j} (-1)^{i+j}
					   (G / e_{j+1}, (\Phi \setminus \{e_{j+1}\} ) \setminus \{e_{i}'\}) \\
					   &\stackrel{(\dagger)}{=} \sum_{l \leq m} (-1)^{m+l+1} (G / e_{l}, (\Phi \setminus \{e_{l}\} ) \setminus \{e_{m}'\}) + \sum_{k < n} (-1)^{n+k-1}
					   (G / e_{k}, (\Phi \setminus \{e_{k}\} ) \setminus \{e_{n}'\}) \tag{$* *$}
	\end{align*}
	Where in $(\heartsuit)$ we used that if $j < i$ then $e_{j} = e_{j}'$ and if $i \leq j$ then $e_{j+1} = e_{j}'$, as well as the observation above.
	In $(\dagger)$ we used the substitution  $m = i-1$,  $l = j$ on the left and  $m = i$,  $k = j+1$ on the right sum.
	Comparing the sums in $(*)$ and $(* *)$ we see that they differ by a sign and thus cancel. Hence  $\partial_{C} \partial_{R} - \partial_{R} \partial_{C} = 0$.

	Combining Step 2 and 3 we get:
	\[
		\partial^2 = (\partial_{C} - \partial_{R})^2 = \partial_{C}^2 - (\partial_{C} \partial_{R} + \partial_{R} \partial_{C}) + \partial_{R}^2 = 0
	\]
\end{proof}

Thus the spaces $(C_{\bullet})$ with the differential $\partial_{\bullet}$ form a chain complex.

\begin{eg}
	Once again we consider the graph $(G,\Phi)$ from above and calculate its boundary operator:
	\ctikzfig{./tikzit/boundary_operator}
	Where we used that $-H_{1}$ is equal to $H_{2}$ by mirroring along the vertical and applying $(2 3)$,
	$-H_{3}$ is equal to $H_2$ by exchanging inner and outer vertices, mirroring along the vertical and applying $(1 3)$ and
	$H_{4}$ is equal to $H_{2}$ by exchanging inner and outer vertices and applying $(1 3)(2 3)$.
	Where we have used the permutation $(2 3)$ on the first,  $(1 3)$ on the third, $(1 3)(2 3)$ on the fourth graph to get the result.
	\ctikzfig{./tikzit/boundary_operator2}
	Where we have used that $-G_{1}$ is equal to $-G_{3}$ by exchanging inner and outer vertices and applying $(1 3)(1 2)$,
	$G_{2}$ is equal to $-G_{3}$ by exchanging, mirroring along the vertical and applying $(1 3)$ and
	$G_{4}$ is equal to $-G_{3}$ by mirroring along the vertical and applying $(1 2)$. 

	Thus we can conclude that $\partial_{\bullet} G = 4 H_2 - 4 G_3$. Moreover we have that $H_2 - G_3 \in \im \partial_{\bullet}$ and 
	as $\partial_{\bullet}^2=0$ also $H_2 - G_3 \in \ker \partial_{\bullet}$
\end{eg}

\subsection{Cubical Chain Complex}
The above constructed complex $C_{\bullet}$ can also be viewed as a cubical chain complex.
Here we think of a graph $(G,\Phi) = (G, \{e_1,\ldots,e_{k}\})  \in C_{k}$ as the $k$-dimensional $[0,1]$-cube embedded in $R^{k}$.
The Graph $G_{\Phi}$ where all edges in  $\Phi$ have been collapsed sits at the origin and the graph $G$ where all edges have been removed from  $\Phi$ but not collapsed
sits diagonally opposite at  $(1,\ldots,1)$.
We can assign a graph to every face in the following way:
Consider a face $F$ of dimension $n < k$, let $bF$ be its barycenter. Then $bF \in \{0,1 / 2, 1\}^{k}$ and denote by $bF_{i}$ its $i$-th coordinate.
Moreover let
\[
	C := \{e_{i} \in \Phi \mid bF_{i} = 0\} \qq{and} R := \{e_{i} \in \Phi \mid bF_{i} = 1\}  
.\] 
Then the graph associated to $F$ is given by $(G / C, \Phi \setminus (C \cup D)$.
Thus an edge gets contracted if $bF_{i} = 0$ and an edge gets removed from $\Phi$ but not contracted if $bF_{i} = 1$.
Now if a face is of dimension $n$ then $n$ coordinates of $bF$ equal  $1 / 2$ and thus the resulting graph has a forest of size $n$ and is hence in $C_{n}$.

The above description gives us a bijection between the reduced graphs of $G$ and the half integral points.
Denote the reduced graph of $G$ associated to $bF$ by $G_{bF}$. Then we can define the boundary operator via this bijection as follows:

\todo[inline]{Formulate this}

To visualize this construction we consider the following example:
\begin{eg}
	Consider again the graph $J$ from example \ref{ex:gAuto} with the forest $\Phi$ given by an edge between the top and middle vertex and
	between the left and write vertex.
	Then its $2$-dimensional cube is given as below:
	\ctikzfig{./tikzit/cubicalCCEx}
\end{eg}

\newpage
\printbibliography

%\input{../header}

\section{Draftparts}

\begin{definition}
	A \emph{graph} $G$ is a finite $1$-dimensional CW complex. The set of edges is denoted by $E(G)$, the set of vertices by  $V(G)$ and the set of half edges by  $H(G)$.
	We call a graph \emph{connected} if the CW complex is connected in the topological sense.
	A graph is said to be \emph{$n$-valent} if every vertex has valency $n$ i.e. for every vertex the number of edges incident is  $n$.

	Lastly a \emph{tree} is a graph which contains no loops and a \emph{forest} is a collection of disjoint trees.
\end{definition}

On trivalent connected graphs we call an \emph{orientation} a choice of cyclic orders of all vertices up to an even number of changes.

\begin{definition}
	A \emph{forested graph} is a pair $(G, \Phi)$, where $G$ is a finite connected trivalent graph and $\Phi$ is an oriented forest which contains all vertices of $G$. 
\end{definition}

\begin{definition}
Let $(G,\Phi)$ be a forested graph and let  $e \in \Phi$. Moreover let $(G_{e},\Phi_{e})$ be the graph where $e$ has been collapsed.
Then there exist exactly two other graphs whose edge collapse results in $(G_{e},\Phi_{e})$. This is visualised in the figure below.
Where  $1,2,3,4$ represent the rest of the graph.

\ctikzfig{./tikzit/IHXRelator}
Now the vector 
\[
	(G,\Phi) + (G',\Phi') + (G'',\Phi'')
\]
is called the \emph{basic IHX relator} associated to $(G,\Phi,e)$.
\end{definition}

Denote by $\widehat{\fg}_{k}$ the vector space spanned by all forested graphs containing $k$ trees modulo the relations $(G,\Phi) = -(G,-\Phi)$.
Moreover let $\fg_{k}$ be the quotient of $\widehat{\fg}_{k}$ modulo the subspace spanned by all basic IHX relators.


\begin{definition}
	Let $\widehat{\partial}_E(G,\Phi): \widehat{\fg}_{k} \to \widehat{\fg}_{k-1}$ be given by
\begin{align*}
	\widehat{\partial}_{E}(G,\Phi) = \sum (G, \Phi \cup e)
.\end{align*}
where the sum is over all edges $e$ of $G \setminus \Phi$ such that $\Phi \cup e$ is still a forest.
Notice that this only happens if the two vertices of  $e$ lie in different trees of $\Phi$. Thus $\Phi \cup e$ has $k-1$ components.
The orientation of $\Phi \cup e$ is determined by ordering the edges of $\Phi$ with labels  $1,\ldots,k$ consistent with its orientation
and then labeling the new edge $e$ with  $k+1$.

Now let the boundary map $\partial_{E}: \fg_{k} \to \fg_{k-1}$ be the map induced by $p \circ \widehat{\partial}_{E}$ where $p$ is the quotient map $\widehat{\fg_{k}} \to \fg_{k}$.
\end{definition}

\begin{proposition}
	$\partial_{E}$ is well-defined and $\partial_{E}^2 = 0$.
\end{proposition}

\begin{proof}
	
\end{proof}

The \emph{forested graph complex} is thus defined as the sequence $\fg_{k}$ with boundary map $\partial_{E}$ and is well-defined by the above proposition.

\end{document}


\end{document}
